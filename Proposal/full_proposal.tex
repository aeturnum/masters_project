%
% ______          _           _    ______                                _ 
% | ___ \        (_)         | |   | ___ \                              | |
% | |_/ / __ ___  _  ___  ___| |_  | |_/ / __ ___  _ __   ___  ___  __ _| |
% |  __/ '__/ _ \| |/ _ \/ __| __| |  __/ '__/ _ \| '_ \ / _ \/ __|/ _` | |
% | |  | | | (_) | |  __/ (__| |_  | |  | | | (_) | |_) | (_) \__ \ (_| | |
% \_|  |_|  \___/| |\___|\___|\__| \_|  |_|  \___/| .__/ \___/|___/\__,_|_|
%               _/ |                              | |                      
%              |__/                               |_|                      
% -------------------------------------------------------------------------
% It is generally worthwhile, I think, to justfiy the use of LaTeX. It's obtuse 
% and it's difficult and it's unclear that it's any better than using Word for
% most projects. However, it has one advantage: it is generated from plain text 
% and this project is using Git, a tool that talks plain text. 
% 
% Using LaTeX will let me create a relaibly formatted document and also have an
% intelligable history of changes. It will also let me work on this document (and
% presumably my final paper which will also use LaTeX) in any environment. It's also
% nice to have a language with comments that won't render into the final paper for
% items like this.
% 
% Docx conversion is achieved through Adobe Acrobat DC. Pandoc was investigated, but it
% ignores much of the tex style and formatting


% Use APA module
% Based on template from https://www.overleaf.com/latex/templates/your-apa6-style-manuscript/kngbbqpypjcq
\documentclass[a4paper,man,natbib]{apa6}
% Imports natbib
% Ref Sheet: https://gking.harvard.edu/files/natnotes2.pdf

% Packages
\usepackage[english]{babel}
\usepackage[utf8x]{inputenc}
\usepackage{amsmath}
\usepackage{graphicx}
% set single spacing
\usepackage{setspace}
% glossary
\usepackage[acronym]{glossaries}
% fancy quotes
\usepackage{epigraph, varwidth}
% For formatting of potential bibliography
\usepackage{enumitem}
% html links
\usepackage{hyperref}
%\usepackage[colorinlistoftodos]{todonotes}

% This section was found in a stack overflow comment about making the epigraph length different and I include it here as an incantation against bad formatting
\renewcommand{\epigraphsize}{\small}
\setlength{\epigraphwidth}{0.6\textwidth}
\renewcommand{\textflush}{flushright}
\renewcommand{\sourceflush}{flushright}
% A useful addition
\newcommand{\epitextfont}{\itshape}
\newcommand{\episourcefont}{\scshape}

\makeatletter
\newsavebox{\epi@textbox}
\newsavebox{\epi@sourcebox}
\newlength\epi@finalwidth
\renewcommand{\epigraph}[2]{%
  \vspace{\beforeepigraphskip}
  {\epigraphsize\begin{\epigraphflush}
   \epi@finalwidth=\z@
   \sbox\epi@textbox{%
     \varwidth{\epigraphwidth}
     \begin{\textflush}\epitextfont#1\end{\textflush}
     \endvarwidth
   }%
   \epi@finalwidth=\wd\epi@textbox
   \sbox\epi@sourcebox{%
     \varwidth{\epigraphwidth}
     \begin{\sourceflush}\episourcefont#2\end{\sourceflush}%
     \endvarwidth
   }%
   \ifdim\wd\epi@sourcebox>\epi@finalwidth 
     \epi@finalwidth=\wd\epi@sourcebox
   \fi
   \leavevmode\vbox{
     \hb@xt@\epi@finalwidth{\hfil\box\epi@textbox}
     \vskip1.75ex
     \hrule height \epigraphrule
     \vskip.75ex
     \hb@xt@\epi@finalwidth{\hfil\box\epi@sourcebox}
   }%
   \end{\epigraphflush}
   \vspace{\afterepigraphskip}}}
\makeatother
% End epigraph modifications

\makenoidxglossaries

\singlespacing

\loadglsentries{glossary}

\renewcommand{\bibsection}{\section*}

\title{Mapping EDEN: Revealing the territory created by society and software}
\shorttitle{Mapping EDEN}
\author{Daniel "Drex" Drexler}
\affiliation{Center for Science, Technology and Society at Drexel University}
\date{October 2019}
\begin{document}
   \maketitle
   % The October revision to this document is based on a Thesis proposal that my Graduate Advisor
   % Vincent Duclos suggested as a good example: The one written by Jim Gall. Though his thesis 
   % porposal is not included here, an outline is reproduced below:
   % 
   % Overview: sucscinct, opinionated description of the project and the problem it is looking to 
   %    address. Ending with a few sentances on literatures engaged and research questions
   %
   % Conceptual Framework: Citation-heavy section citing the specific theories you expect to be 
   %    using in your work, continue to describe the situation as it is useful 
   %
   % Methods & Timeline: How you will do the work and the time in which you suggest to do it.
   %
   % Perparation of Researcher: Brief explanation about why you would be good at this and any previous
   %    experiences that are relevant to this work
   %
   % Preliminary List of Sources: Sources that are expected to be used in the project including all 
   %    sources cited in the proposal.
   % 
   \section*{Overview}
   % Overview (0.5-1 page)
   % A summary of other sections of the proposal
   \epigraph{[T]he "writing of technology" is by by no means universal; the opaque and stubborn places do not lie simply beneath technology, but are wrapped around and in it}{\textit{Adrian Mackenzie \\ Cutting Code: Software and Sociality p. 181}}

   This project will examine \acrfull{eden}, a \acrfull{foss} project developed to manage disaster response. \acrshort{eden}'s makers, the Sahana Foundation, lists their mission as "Saving lives through information management solutions." They present \acrshort{eden} as bringing the fruits of \acrfull{ict} to humanitarian projects, part of a wider "\acrfull{ict4d}" movement. Sahana lists "community engagement" as a goal and want to be seen as a 'user centered' project that gives users representation in digital spaces \citep{Sahana_Foundation_undated-hl}. 

   \acrshort{ict4d} and 'community driven' software attempts to focus on user agency, but like all software they emerge out of a social process in which the needs of the project's engineers and managers are most visible. Each \acrshort{eden} release is a snapshot of the team's expectations about their users and the project's future \citep{Mackenzie2006-hb}. Analyzing the material qualities of the tools used to create software can help us spread positive technological impacts more widely than they are currently felt.

   This project will look at the interface and internals of \acrshort{eden}. What ways of being and doing are supported by the design of \acrshort{eden}? What qualities might be impossible to represent? How are these standpoints impacted by the material qualities of the constitutive tools used to create \acrshort{eden}? 

   % \medskip
   Critical engagement with technoscience must have a second act. Does understanding standpoints within something help us change it? This project is a \textit{situated intervention} into the code and design of \acrshort{eden} \citep{Zuiderent-Jerak2015-go}. What can we learn by trying to change \acrshort{eden} guided by the standpoints that we discover? What lessons can we draw about how to critique better by engaging with material realities?

   % \medskip
   This project hopes to realize the goals of \citet{Sismondo2008-xz}'s \textit{engaged program} by bringing feminist epistemological critiques into conversation with the process-tool we call "software." The goal is to produce new knowledge about \textit{how software can} change and \textit{how critique can guide that change}.
   \par
   \begin{flushleft}
   I focus on three questions:
   \end{flushleft}
   \begin{enumerate}
     \item What standpoints can be found within \acrshort{eden}?

     \item How are \acrshort{eden}'s standpoints driven by its constitutive technologies?

     \item Through modifying \acrshort{eden}, can we identify useful qualities of an applied standpoint critique?  
   \end{enumerate}

   
   % \pagebreak
   \section*{Conceptual Framework}
   % Background	description	of	problem	and	context (1 page)
   % Intellectual	Merits (1-2 pages)
   % 2 to 3 scholarly conversations your research contributes to
   \epigraph{A model is worked, and it does work}{\textit{Donna Haraway \\ Staying with the Trouble: Making Kin in the Chthulucene p. 63}}
   
   Building software is as much about imagining the future and accounting for the past as it is about creating a particular product for a particular time. Though software projects produce executables (discrete tools that can be moved and sold), it is better understood as a process-tool which exists and develops within a social and commercial context whose concerns and constraints extend far beyond a given executable. Much of the work in software involves managing the relationships between the other process-tools on which it relies (databases, programming languages and so on), as well as the relationships the software has with its users, its makers, and its imagined future \citep{Mackenzie2006-hb}. These relationships are necessarily managed from a perspective within the project. Software developers exist in a heterogeneous world of software creation and formation cultures which also drive decisions around the nature and form of the project \citep{Ensmenger2012-kz,Gabriella_Coleman2012-lq, Kelty2008-jm}. The experience from outside is difficult to predict, especially as the same executable in different environments will create different code/spaces that are specific and unpredictable \citep{Kitchin2011-af}. The ways in which software can shape how people see the world \textit{through} it and how things are represented \textit{within} its vision of the world can depart dramatically from the designers' intent.

   Software strives to show a "view from nowhere" when guiding its users. The reduction of the complexities of life into a manageable series of \gls{capta} is why we use tools to make intractable situations tractable. These simplifications also represent a concern: what information is being thrown away and how was it chosen? Each time a piece of software shows someone a view of the world, we can understand that view as being from a particular standpoint. As having particular concerns and goals in showing \textit{this} but not \textit{that} \citep{Harding1992-od}. Such perspectives allow tools to fulfill their function, but allow power relationships to operate unobserved. To combat this, we can use on a \textit{situated epistemological} approach which re-contextualizes the view through noting how information is hidden and analyzing the normally invisible structural elements of the view from nowhere \citep{Haraway1988-nh}. To ignore this quality in tools can give rise to inadvertently selecting tools that render natural variation as racial difference or environmental factors as measurements of native purity \citep{Subramaniam2014-wg}.

   If standpoint analysis were to give simple, uncomplicated guidance, then the task of improving problematic software would be simple. But software has material qualities that aren't impacted by descriptions of their undesirable consequences. To change the impact of particular software on particular people in particular situations we must grapple with those material qualities \citep{Bivens2017-tc}. \citet{Zuiderent-Jerak2015-go}'s concept of \textit{situated intervention} offers a powerful tool for learning how we can change real programs. Only by trying to apply the judgments from theory to artifacts that emerge out of the the process-tool of software can we learn to shape those judgments to be more useful to those who are building the original software.

   The design of the project draws on Science, Technology \& Society practices of making \& doing and specifically on \citet{Zuiderent-Jerak2015-go}'s concept of \textit{situated intervention}. It is deeply indebted to \citet{Bivens2017-tc} demonstration of the impact of the material nature of software and \citet{Mackenzie2006-hb}'s work on the Sociality of software.

   \section*{Project Structure}
   % Study	Design (1-2 pages)
   
   % Description and justification of data sources or fieldsites
   \epigraph{That virtual worlds are places means they can be fieldsites;}{\textit{Tom Boellstorff \\ Coming of Age in Second Life p. 107}}

   \subsection*{Outcomes}
   The project will produce three artifacts: a paper detailing some perspectives within \acrshort{eden} and reflecting on the attempt to engage them, New \acrshort{eden} (a modified version of \acrshort{eden}) and a git repository that will have a history of the project..
   % Broader   Impacts (.5-.75 page)
   % Output (e.g., conference presentations, results to informants, white paper, thesis, etc.)
   \subsubsection*{The Git Repository}
   The two primary artifacts and all the other ephemera (including this proposal) involved in producing them will be publicly available in a git repository.

   The Git repository for this project is available at \url{https://github.com/aeturnum/masters_project}.

   \subsection*{Phases}
   \subsubsection*{One}
   The first phase of this project will be a survey of literature. Focusing on Science, Technology and Society works that analyze the standpoint and biases of other technical tools and relevant work on digital humanitarianism and disaster recovery. I will also be familiarizing myself with the implementation and operation of EDEN itself. These theoretical tools are what I will engage the "text" of EDEN with in the second phase. 
   \begin{flushleft}
   The git repository will begin having content added to it immediately.
   \end{flushleft}
   \subsubsection*{Two}
   The second phase will be be a co-constitutive process between social critique and technical analysis. I'll search for perspectives within \acrshort{eden}, analyzing how that perspective emerges from \acrshort{eden} and impacts the world around \acrshort{eden}. My focus will be perspectives that seem driven by the material technical realities of \acrshort{eden}. These findings will be the center of the project. This will be a filtering process - some number of perspectives will need to be investigated and found uninteresting. 
   \begin{flushleft}
   This phase will produce the first half of the paper.
   \end{flushleft}
   \subsubsection*{Three}
   The perspectives found in phase two will guide modifications to the \acrshort{eden} software package itself. These modifications will be made with the goal of engaging with \acrshort{eden} on a material level and understanding the practicalities of trying to address encoded perspectives. The modified software, called "New \acrshort{eden}", will be the second product of this project.
   \begin{flushleft}
   This phase will produce the second half of the paper and New \acrshort{eden}.
   \end{flushleft}
   \subsection*{Schedule}
   % Schedule of work
   % Much thanks to: 
   % https://www.tablesgenerator.com/
   % https://en.wikibooks.org/wiki/LaTeX/Tables#Text_wrapping_in_tables
   % https://stackoverflow.com/questions/790932/how-to-wrap-text-in-latex-tables
   \begin{center}
   \begin{tabular}{|l|l|p{0.1\linewidth}|p{0.1\linewidth}|p{0.6\linewidth}|}
   \hline
   Phase & Quarter & Activity& Time Allocation & Goal                         \\ \hline
   One   & Fall    & Reading & 80\%            & 800 pages read with notes    \\ \cline{3-5} 
         &         & Writing & 10\%            & Notes and planning documents \\ \cline{3-5} 
         &         & Coding  & 10\%            & Notes on \acrshort{eden} program structure   \\ \hline 
   Two   & Winter  & Reading & 40\%            & Final selection of bibliography with additions based on notes and readings.   \\ \cline{3-5} 
         &         & Writing & 30\%            & Outline for final paper with ~20\% of content finished \\ \cline{3-5} 
         &         & Coding  & 30\%            & Rough modifications completed, but in need of polish  \\ \hline
   Three & Spring  & Reading & 10\%            & Unexpected additions to literature and gathering specific quotes from previously completed elements.   \\ \cline{3-5} 
         &         & Writing & 40\%            & Final paper \\ \cline{3-5}
         &         & Coding  & 40\%            & Full branch of \acrshort{eden} project with documentation and demonstration server. \\ \hline
   \end{tabular}
   \end{center}
   % \newpage

   \subsection*{Fieldsite Details}
   % left over
   \acrshort{eden} was initially developed and deployed by an alliance of companies in the Sri Lankan \acrfull{ict} sector in response to the Indian Ocean Earthquake \& Tsunami. \acrshort{eden} has been used in response to a number of disasters in the developing world as well as ongoing supply management programs in the developed world \citep{Sahana_Foundation_undated-hl}. It is written in the \acrshort{foss} language \Gls{python} and uses the \acrshort{foss} \gls{web framework} web2py to build its components. Its functionality is broken up into many different modules which group common functionality within a layer of access conventions. \acrshort{eden} also uses \acrshort{foss} tools wherever possible: its databases of choice (\Gls{MySQL} and \Gls{PostgreSQL}) are \acrshort{foss} databases.

   \begin{flushleft}
   The software's homepage can be found here: \url{https://sahanafoundation.org/}. \linebreak 

   \acrshort{eden}'s source code can be found here: \url{https://github.com/sahana/eden} \linebreak 

   \acrshort{eden} does not have a formal release system so this project will be done on the version represented at git commit 653f76aa574cb2dddb9053eef0f8d815156cb168.
   \end{flushleft}


   % Description of methods
   \subsection*{Preparation of Researcher}
   As a Science, Technology \& Society graduate student I've been engaging with analytical questions around the impact of tools and their power to structure space. A previous project of mine used a energy pricing tool as a jumping off point for a critique of approaching the question of energy supply financially and suggested an alternative or additive perspective and gave me experience engaging with the social implication of a technical system.

   Previous to being a graduate student I obtained an undergraduate degree in computer science and worked professionally as a programmer for about eight years. I did most of my work writing and designing web services and generally used python for this purpose (like \acrshort{eden} in both cases). I have experience using most of the tools and environments that \acrshort{eden} utilizes and expect that this background will be an asset in implementing changes in phase three.

   I enter this project with a practiced skepticism about the promises of technoscience from my STS education, which professional software experience supports. However, I believe that software is not magic and that, like \citet{Latour1983-bs}'s expanding laboratory, there is hope for us to reveal the man behind the technoscientific software curtain.

   \subsection*{Goals}

   % How it will help us understand X (and Y)
   This project will try to be an example of \citet{Sismondo2008-xz}'s engaged program - connecting analysis to real world issues and expending our understanding of both. It also hopes to be an example of and a coherent argument for further interventions using STS theories in the real world. I hope to respond to \citet[p.55]{Haraway2016-nc}'s statement that it "matters with which ways of living and dying we cast our lot" by pushing the discipline to become more materially engaged with the world we seek to understand.

   \printnoidxglossaries
   \setlength{\parindent}{4em}
   \bibliography{full_proposal}

\end{document}