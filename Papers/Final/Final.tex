% !TEX options=--shell-escape
% https://github.com/SublimeText/LaTeXTools/issues/1082
%  _____ _             _ 
% |  ___(_)_ __   __ _| |
% | |_  | | '_ \ / _` | |
% |  _| | | | | | (_| | |
% |_|   |_|_| |_|\__,_|_|
%                        
                   
% -------------------------------------------------------------------------
% https://www.kammerl.de/ascii/AsciiSignature.php - standard
% I've moved beyond justifying latex and am now just living in it.


% Use APA module
% Based on template from https://www.overleaf.com/latex/templates/your-apa6-style-manuscript/kngbbqpypjcq
\documentclass[a4paper,man,natbib,floatsintext]{apa6}
% Imports natbib
% Ref Sheet: https://gking.harvard.edu/files/natnotes2.pdf

% Packages
% Unicode in latex is a new, previously undiscovered circle of hell
\usepackage[english]{babel}
\usepackage[T1]{fontenc}
\usepackage[utf8x]{inputenc}
\usepackage{textcomp}
% /unicode
\usepackage{amsmath}
\usepackage{graphicx}
% allow single spacing
\usepackage{setspace}
% glossary
\usepackage[acronym]{glossaries}
% fancy quotes
\usepackage{epigraph, varwidth}
% For formatting of potential bibliography
\usepackage{enumitem}
% html links
\usepackage{hyperref}
%\usepackage[colorinlistoftodos]{todonotes}
% Allow seperate footnotes
\usepackage{sepfootnotes}
% adding spaces around paragraphs
% https://tex.stackexchange.com/questions/74170/have-new-line-between-paragraphs-no-indentation
%\usepackage{parskip}
% Diagrams!
\usepackage{svg}
\usepackage{graphicx}
% vertical space around list items
\usepackage{enumitem}
% small font in verbatum
% https://tex.stackexchange.com/questions/171803/change-font-size-of-the-verbatim-environment
\usepackage{fancyvrb}

% image location
\graphicspath{ {images/} }
% less space in bulleted lists
\setlist{nosep}
% set single spacing

% never ever split foot notes for any reason
% https://tex.stackexchange.com/questions/32208/footnote-runs-onto-second-page
\interfootnotelinepenalty=10000

% Trying to include graphics has been a JOURNEY
% Links used:
% aborted attempt to use svgs
% https://tex.stackexchange.com/questions/542766/inkscape-1-0-not-able-to-export-files-needed-for-svg-package
% https://tex.stackexchange.com/questions/2099/how-to-include-svg-diagrams-in-latex
% (updated comand line is: inkscape -D --export-latex  "EDEN_diagram.svg" --export-type="pdf" --export-filename="EDEN_diagram_svg-tex.pdf")
% https://www.overleaf.com/learn/latex/Inserting_Images
% convert EDEN_diagram.png -resize 50% EDEN_diagram.png


% This section was found in a stack overflow comment about making the epigraph length different and I include it here as an incantation against bad formatting
\renewcommand{\epigraphsize}{\small}
\setlength{\epigraphwidth}{0.6\textwidth}
\renewcommand{\textflush}{flushright}
\renewcommand{\sourceflush}{flushright}
% A useful addition
\newcommand{\epitextfont}{\itshape}
\newcommand{\episourcefont}{\scshape}

\makeatletter
\newsavebox{\epi@textbox}
\newsavebox{\epi@sourcebox}
\newlength\epi@finalwidth
\renewcommand{\epigraph}[2]{%
  \vspace{\beforeepigraphskip}
  {\epigraphsize\begin{\epigraphflush}
   \epi@finalwidth=\z@
   \sbox\epi@textbox{%
     \varwidth{\epigraphwidth}
     \begin{\textflush}\epitextfont#1\end{\textflush}
     \endvarwidth
   }%
   \epi@finalwidth=\wd\epi@textbox
   \sbox\epi@sourcebox{%
     \varwidth{\epigraphwidth}
     \begin{\sourceflush}\episourcefont#2\end{\sourceflush}%
     \endvarwidth
   }%
   \ifdim\wd\epi@sourcebox>\epi@finalwidth 
     \epi@finalwidth=\wd\epi@sourcebox
   \fi
   \leavevmode\vbox{
     \hb@xt@\epi@finalwidth{\hfil\box\epi@textbox}
     \vskip1.75ex
     \hrule height \epigraphrule
     \vskip.75ex
     \hb@xt@\epi@finalwidth{\hfil\box\epi@sourcebox}
   }%
   \end{\epigraphflush}
   \vspace{\afterepigraphskip}}}
\makeatother
% End epigraph modifications

\makenoidxglossaries
% These used to be in a seperate file, but I want them in here because it will let me cite things from the bibliography in them

%   _____ _                                
%  / ____| |                               
% | |  __| | ___  ___ ___  __ _ _ __ _   _ 
% | | |_ | |/ _ \/ __/ __|/ _` | '__| | | |
% | |__| | | (_) \__ \__ \ (_| | |  | |_| |
%  \_____|_|\___/|___/___/\__,_|_|   \__, |
%                                     __/ |
%                                    |___/ 


\newacronym{foss}{FOSS}{Free open source software}
% \newacronym{sts}{STS}{Science, Technology \& Society}
\newacronym{ict}{ICT}{Information and Communications Technology}
\newacronym{eden}{Eden}{Emergency Development ENvironment}
\newacronym{ict4d}{ICT4D}{Information and Communication Technologies for Development}


\newglossaryentry{ICT}
{
    name=Information and communications technology,
    description={Information and communications technology: A term for the mix of physical devices and software that mix analog and digital techniques to allow communication, recall and storage of data.}
}
 
\newglossaryentry{FOSS}
{
    name=Free Open Source Software,
    description={Free Open Source Software: A method of developing software that does not charge for use of the software. The software's source code is also freely distributed, allowing users to modify their copy of the software or suggest changes to the organization developing it.}
}

\newglossaryentry{pyt}
{
    name=python,
    description={Python is an open source, interpreted, high-level, general-purpose programming language. It was initially released in the early 1990s, but has been continuously updated since then. Python 2 was the standard from 2000 to 2008, but in 2008 Python 3 was released with backwards-incompatible changes. This split the Python community, leading to Python 2 and Python 3 having slightly different versions of tools available. Python 2 continued to receive support and updates until 2020, when official support for Python ended (Wikipedia contributors, 2020). The open source nature of Python means that it is likely that Python 2 will continue to be updated by users after official support ends. }
}
% http://eden.sahanafoundation.org/wiki/WikiStart#WhatisSahanaEden
\newglossaryentry{EDEN}
{
   name=Eden,
   description={The Emergency Development ENvironment, developed by the Sahana Foundation, is an Open Source Humanitarian Platform which can be used to provide solutions for Disaster Management, Development, and Environmental Management sectors}
}

\newglossaryentry{ICT4D}
{
   name=Information and Communication Technologies for Development,
   description={The movement to use digital technologies to serve the needs of the underprivileged and those living in the developing world. Criticized for being directed more at the developed worlds' love of technology than the actual needs of those it purports to serve. Also called digital humanitarianism}
}

\newglossaryentry{sp}
{
    name=software project,
    description={Drawing from \citet{Mackenzie2006-hb}, this is the total social, material and organizational structure associated with the production, distribution and use of a particular piece of software. This includes the people who work directly on the source code of any \gls{sa} associated with the project as well as anyone who is part of the organization that manages, funds and directs the development of the software. It may also involve people who are not employed by the formal owner of the software project (this is especially true for open source developers). Any \glspl{sa} assembled by members of the project who are acting as members (and not on their own) are also part of the project. \glspl{sa} assembled using source code from the project are not part of the project per-se, but the software project bears some measure of responsibility for how those \glspl{sa} function and act in the world}
}

\newglossaryentry{sa}
{
    name=software artifact,
    description={An entity of uncertain composition that behaves, for common users (as opposed to technical staff) as a single entity. A software-artifact could be a single binary with all its functionality provided by machine code included in that single location in memory. It could also be a binary that makes calls to various libraries. In the case of EDEN, it's an expensive set of Python files - some of which are human readable and some of which have been modified to make them faster to read. In this case a separate program, the Python binary itself, reads and executes the EDEN program. Even though EDEN is many separate files and contains no machine code, it is properly considered a software-artifact because users interact with it as a single entity.  Software-artifacts are produced periodically by software projects, but are not themselves entirely "software." The idea of software must be large enough to allow multiple software-artifacts to exist within the same project: the current version of your phone OS and the next version, the Facebook app on your phone and the web server that tells it what has happened. Nevertheless, when we are using "software" we are always directly interacting with one or more software-artifact(s). They are the material reality of software projects}
}

\newglossaryentry{ap}
{
    name=assembly process,
    plural=assembly processes,
    description={All software must be assembled. In the simplest case this means the placing of machine code instructions, one after the other, in computer memory. In more complicated processes, this can include the translation from one or more programming language to the appropriate machine code, linking libraries together, retrieving source code from a remote location, and so on. This is rarely a manual processes, and is often done through one or more software artifacts. How a software artifact is assembled has an impact on the behavior of the artifact. Components can be placed in such a way that they do not function as expected, that their contribution to the artifact subverts expectations (an artifact might include a component designed for machine learning but only use its error formatting functionality) or in a way that makes functioning impossible. Assembly processes are often managed by a software project with the aim of producing their software artifact(s), but the process should not be considered a component of the artifact. It impacts the artifacts' structure, but isn't carried around in the artifact}
}

\newglossaryentry{cmp}
{
  name=composite,
  description={Composites are fixed assemblies of more than one component which inherits the material qualities of its components, mediated through the structure which those components are assembled into. Once assembled, a composite can operate without human intervention. Composites are typified by extremely high complexity and extremely complex behavior, but once assembled their possible actions are set. They cannot be changed, only replaced. Each component of a composite may be an instruction or a datum, but they are generally composites themselves. Generally, each component makes a proportional contribution to the material qualities of the composite it is integrated into, but layers of assembly can place components in unexpected relationships with unforeseen outcomes}
}

\newglossaryentry{webframework}
{
  name=Web Framework,
  description={A library (or set of libraries) that handles translation from and to the language of HTTP and other web technologies. The library of functionality they provide can range from wrapping basic text in a format a web browser will understand to providing functionality that allows users to log in, receive real-time messages and interact with the same web site from multiple clients simultaneously \citep{Statz2010-hv}. Each Web framework also has its own philosophy about what information the developer is expected to provide and how much the developer needs to set. Some frameworks attempt to provide everything that is required to build almost any website, while others focus on providing a core of functionality and may not include support for connecting to a database or managing logins. Generally, any web framework may be used to make any website, but framework selection will dramatically change the experience of the software engineers developing that web site }
}

\newglossaryentry{vm}
{
  name=Virtual Machine,
  description={A family of packaging techniques that allow packaging and distribution of an entire computing environment (operating system, libraries, programs) in a single digital object. This allows one physical computer to run several independent "machines" simultaneously. When a virtual machine is started, it the hardware it sees is virtual hardware simulated by the program "hosting" the virtual machine. The hosting program then translates requests to that virtual hardware into requests to the real hardware and relays the responses. Running a program in a virtual machine is slightly less efficient than running it on real hardware, but this cost is often less than the cost of manually arranging non-virtual installs on real machines}
}

\newglossaryentry{sf}
{
  name=Sahana Foundation,
  description={The nonprofit that manages the EDEN project. Based in Los Angeles with an international scope and focuses on supporting communities in disaster preparedness and response through information technology.}
}

\newglossaryentry{w2p}
{
  name=web2py,
  description={An open source web framework for Python 2.7 and 3+. Provides elements that handle the common tasks of providing a web service: producing HTML, handling headers, communicating with web servers, responding appropriately to various HTTP codes. A primary component of Eden}
}

%      _______ _                                
%     / / ____| |                               
%    / / |  __| | ___  ___ ___  __ _ _ __ _   _ 
%   / /| | |_ | |/ _ \/ __/ __|/ _` | '__| | | |
%  / / | |__| | | (_) \__ \__ \ (_| | |  | |_| |
% /_/   \_____|_|\___/|___/___/\__,_|_|   \__, |
%                                          __/ |
%                                         |___/ 

% \loadglsentries{glossary}

\renewcommand{\bibsection}{\section}

%  ______          _               _            
% |  ____|        | |             | |           
% | |__ ___   ___ | |_ _ __   ___ | |_ ___  ___ 
% |  __/ _ \ / _ \| __| '_ \ / _ \| __/ _ \/ __|
% | | | (_) | (_) | |_| | | | (_) | ||  __/\__ \
% |_|  \___/ \___/ \__|_| |_|\___/ \__\___||___/
                                            

\sepfootnotecontent{softwarereleases}{Software is an ongoing process not a fixed object \citep{Mackenzie2006-hb}. \Gls{sp} are always making decisions about how they should allocate their time and energy. This can involve changing how they structure their \glspl{sa}, but it also generally involves maintaining some number of existing \gls{sa}. Because \gls{sa} are complicated, it's common for them to contain bugs the \gls{sp} did not initially detect. When those bugs are found, the \gls{sp} generally reacts by going back and updating their old \gls{sa}. However, time and resources are finite, so \glspl{sp} generally place limits on how long they update older releases.}

\sepfootnotecontent{componenttrajectory}{\Glspl{sp} are aware that that future will come. They know that it takes time and attention to keep a projects' relationship with all of its components current. Some users of a \gls{sp} want to have access to \glspl{sa} with the latest changes, no matter how unreliable they are. Some users of \glspl{sp} are using their \glspl{sa} for tried-and-true core features. In response \glspl{sp} follow a somewhat standard method for releasing changes. Stable versions are released rarely. They often have many changes from the last stable version, but those changes have been tested and examined for a relatively long period. Often \glspl{sp} will release potential stable releases as "release candidates" as a final check to validate their reliability. Unstable versions, on the other hand, are released often. They have relatively little testing, a high chance of containing bugs and often contain large changes.}

\sepfootnotecontent{dslnote}{'General purpose' programming languages are all Turing complete, which means they can fully describe the behavior of a 'Turing machine' - a theoretical automaton described by Alan Turing in the early 20th century. This means that all general purpose programming languages can be reduced to one another (a series of instructions generated by one language can be generated by another).}

\sepfootnotecontent{socialcharacter}{I see a straight line between the historical social oddities of computer workers and the formation of the modern \acrshort{foss} movement. Having a space where computer people already understood themselves as being set apart (even in an pejorative way) helped them imagine different understandings of intellectual property. The early history of Linux was fueled by the unusual legal position of AT\&T's Bell Labs (they were banned from making money off their work) which led to them taking few steps to protect their copyright and helping seed the open source movement \citep{Kelty2008-jm}.}

\sepfootnotecontent{csblackbox}{Computer science draws on the general blackbox concept that Latour drew inspiration from to talk about acting on formal function definitions without knowing about how the function is implemented \citep[p. 33]{Abelson1996-fk}.}

\sepfootnotecontent{longquote}{Emphasis added by me. This quote mentions "regression-testing," which is worth briefly explaining. Regressions are when old, previously fixed problems reappear in new versions of the \gls{sa}.}

\sepfootnotecontent{ethnographylimits}{It's here I really miss an ethnographic component. A lot of my logic here is necessarily consequentialist based on my understanding of how software development works, but the logic the team is following could easily be deeper and more complicated than I imagine. Nevertheless I hope to draw conclusions that will be based on the material outcomes of their decisions and don't depend on the \gls{EDEN} project possessing a particular internal understanding.}

\sepfootnotecontent{working}{To be blunt, I find the common ideas of 'working' or 'broken' to not be that useful and suspect that a deeper literature review would have helped me to speak clearly on this question.}

\sepfootnotecontent{runtimeimports}{It's worth expanding a little how this works. The \gls{pyt} \gls{sa} executes the python in the initial \gls{pyt} file provided. This could involve executing statements immediately or it could involve defining programmatic objects that may be invoked later. When a \gls{pyt} \gls{sa} encounters an \textit{import} statement, it executes the associated file. This means that many \gls{pyt} programs have an initial \textit{import} chain of stand-alone \textit{import} statements. However, \textit{import} statements can also be located in the programmatic objects that are defined for later use - and those objects will not be executed unless they are invoked elsewhere. So assembly is a somewhat fractured process with \Gls{pyt} and other interpreted languages.}

\sepfootnotecontent{pythonexeq}{When \Gls{pyt} import statements are causing new files to be read and executed by the interpreter, we are in a kind of mixed execution / assembly phase. The \Gls{pyt} is doing a mix of adding symbols to the symbol table (which other languages do during compilation) and executing program statements. If the \Gls{pyt} \gls{sa} were to fail to find an import or encounter a syntax error, that is an assembly error. If executing a program statement causes an error, it was not in assembly. This means that Python may translate some python statements to machine code, execute them, then fail when it reaches another assembly stage. I do not believe this is problematic for my work.}

\sepfootnotecontent{askingforhumaninput}{A \gls{sa} can be designed to seek input at any point, but this does not change the fact that humans cannot \textit{intercede and change} the actions the \gls{sa} will take. The list of actions the \gls{sa} takes may involve seeking human input, but the possible range of actions is already fixed. As \citet[p. 181-182]{Mackenzie2006-hb} points out, once agency is fixed in software, the software will reflect it in its execution. Any opportunity to "give feedback" to an assembled \gls{sa} will, at best, select another path from the possible paths fixed by assembly.}

\sepfootnotecontent{libversions}{The exact versions of the components may not be the ones stored in the canonical git repository. Each repository was at a particular git hash for this project. Eden was at 0718c0681bb58576a613e0edc4d4070ac214be21. web2py was at 93ef108c0b4b1c100622bf0002ec0972dec8be46. pydal was at cecd77127c122404c1aee7f6377c6a0150d86d84. yatl was at 5deb403a9e45f617588f02cd8f7682b3f98571b4. Some of these only exist in my fork of the respective repositories. }

\sepfootnotecontent{symbol}{In computer programming languages, a symbol a generic term for a saved value. The name comes from the practice in many languages of using a symbol table, which is an index of variable name to current value.} %todo: find source?

\sepfootnotecontent{moduledetails}{In order for a directly to be a well formed \Gls{pyt} module, it must contain a \Gls{pyt} file named \_\_init\_\_.py. This file may be empty, but may also execute \Gls{pyt} code. \_\_init\_\_ files are commonly used to make the elements of the library that are intended to be used by outsiders more easily accessible. For example, an \_\_init\_\_.py file in the \textit{example} module could contain the statement 'from .functions import run.' This would allow module users to type 'from example import run' instead of 'from example.functions import run.' This approach would allow \textit{example} users to know less about the internal details of \textit{example}.}

\sepfootnotecontent{softwaretoo}{For software, situations where prediction is impossible are quite common \citep{Kaplan_undated-xy}. There is nothing that requires composites to be software, per-se, however, and any sufficiently complex mechanistic assembly would meet this requirement.}

%      ________          _               _            
%     / /  ____|        | |             | |           
%    / /| |__ ___   ___ | |_ _ __   ___ | |_ ___  ___ 
%   / / |  __/ _ \ / _ \| __| '_ \ / _ \| __/ _ \/ __|
%  / /  | | | (_) | (_) | |_| | | | (_) | ||  __/\__ \
% /_/   |_|  \___/ \___/ \__|_| |_|\___/ \__\___||___/

% \title{Software: Gradually and Then Suddenly}
\title{Software as Composites: A Maze of Twisty Passages}
\shorttitle{Composites}
\author{Daniel "Drex" Drexler}
\affiliation{Center for Science, Technology and Society at Drexel University}
\date{April 2020}

\abstract{Software presents new problems for understanding how creator agency is expressed through the material character of tools. Existing work emphasizes both the relationships at the center of software creation and the tendency for personal bias to drive the behavior of tools. I investigate how these forces are mediated by the material qualities of software, especially the software creation process, and how they appear in the behavior of assembled pieces of software in the world. I find that existing theoretical models that might be used to understand how software impacts the world are insufficiently reflective of the degree to which those who work on software are unable to predict its behavior. In response to my investigation and my own work in software, I suggest a new way of thinking about software, composites, whose qualities encompass both the tendency of software to faithfully transmit the intent of its creators and the potential for it to subvert that intent. }
\authornote{\singlespacing Thank you to my project advisor Dr. Vincent Duclos and the rest of my committee: Dr. Gwen Ottinger, and Dr. Alison Kenner. Particular thanks to Dr. Ottinger who spent an enormous amount of her time helping me develop my ideas. This project would not exist without her help. Thanks to the entire Center for Science, Technology \& Society at Drexel university. I owe everything I learned to the many kind, patient and brilliant faculty with whom I had the privilege to work.}

\begin{document}
   \maketitle
   % single space toc
   \singlespacing
   \tableofcontents
   % back to double spacing
   \doublespacing
   % This is a progressive work that's trying to explain my situatedness
   \section{It's Software's World, We Just Live in it}
   \singlespacing
   \epigraph{[It] matters with which ways of living and dying we cast our lot}{\textit{\citet[p. 55]{Haraway2016-nc}}}
   \doublespacing

   Software is now well represented in every nook and cranny of the world. Though this project directly engages software, software was also used at every stage of its production: conception, composition, distribution and (in the age of covid-19) discussion. In the same moment that our reliance on software reaches ever loftier heights, we are surrounded by stories of important software projects being biased against vulnerable groups. Standpoint Theory suggests that such biases can be explained by the hidden or unacknowledged perspectives of the people making software \citep{Harding1992-od,Haraway1988-nh}. In this sense, the problems caused by bias in software are not new problems. But, as \citet{Mackenzie2006-hb} and \citet{Kitchin2011-af} have noted, the management of a software project is filled with the work of managing relationships. The relationship between past and present \glspl{sa} . Relationship between humans working on the project. Relationship to the imagined future. Software also exists within ecosystems: code written for a particular software project is surrounded and supported by code which is used by many \glspl{sp}. How a particular \gls{sa} impacts the world reflects the intent of both the organization that released this \gls{sa} and the diffuse intent of the surrounding software/hardware ecosystem. Software interventions into existing projects can highlight the paths of agency within \glspl{sa}. What can we learn about effecting change in software by studying how agency is mediated by the assembly process?

   \Glspl{sa} are the result of an assembly process that brings together the source code from many different software projects. Each \gls{sa} appears to be what Latour called a "blackbox" - a container that hides complexity. But we will see that that idea of a blackbox obscures important qualities of software. In Latour's theory, each blackbox contains other blackboxes with clean separations between those layers \citep[p. 183]{Latour1999-ui}. \Gls{sa} components, in contrast, often overlap and interleave. This level of complexity lead \gls{sa} to be predictably unpredictable, a tendency which developers have responded to with surveillance practices. I think understanding \gls{sa} as \glspl{cmp} - fixed heterogeneous arrangements of components. A \glspl{cmp}' behavior emerges out of the behavior of individual components, mediated by the particular arrangement of those components. \Gls{sa} are not changed post-assembly, so while users always encounter them assembled, developers are always thinking about and changing with pre-assembly components. Developers leave imprints of their views on each component, and ensure that those views survive assembly with surveillance-based testing practices. However, due the limits of developer attention, unforeseen qualities of \glspl{cmp} constantly threaten to emerge unexpectedly when users interact with them. Because \glspl{cmp} often have other \glspl{cmp} as components, these unforeseen qualities can emanate from any element. Their unpredictable nature is what sets \glspl{cmp} apart from other tools.

% ===================================================
% =  _     _ _     ____            _                =
% = | |   (_) |_  |  _ \ _____   _(_) _____      __ =
% = | |   | | __| | |_) / _ \ \ / / |/ _ \ \ /\ / / =
% = | |___| | |_  |  _ <  __/\ V /| |  __/\ V  V /  =
% = |_____|_|\__| |_| \_\___| \_/ |_|\___| \_/\_/   =
% ===================================================

   % \section{How I Think About Software}

   % Modern Software Production
  % Sources in this section:
  % Mackenzie, A. Cutting Code: Software and Sociality. Edited by Steve Jones. Peter Lang Publishing, 2006.
  %   - relationality of software
  %   - the process of creating software is a process of managing relationships: with the future, with the past, with other software
  
  % Bivens, Rena. “The Gender Binary Will Not Be Deprogrammed: Ten Years of Coding Gender on Facebook.” New Media & Society 19, no. 6 (June 1, 2017): 880–98.
  % Elish, Madeleine Clare. “Moral Crumple Zones: Cautionary Tales in Human-Robot Interaction.” Engaging Science, Technology, and Society 5, no. 0 (March 23, 2019): 40–60.
  
  % ...?
  \section{Software: a Thing and a Process}
  \singlespacing
  \epigraph{I have passed through a membrane where the real world and its uses no longer matter. I am a software engineer[.]"}{\textit{\citet[p. 3]{Ullman2012-fq}}}
  \doublespacing
  
  It is difficult to talk with precision about what software 'is.' Individual programs that have been written in (or translated into) machine code are software (I will be calling these '\glspl{sa}'). Networks of \glspl{sa} which only function when connected (such as the combination of one or more Facebook apps and the Facebook servers) are also clearly software. There are also software projects like the IETF Task Force that produce no \glspl{sa} at all, but create and manage standards that are essential for the interoperability of software. The field is diverse and engaging with it demands some definitional boundaries.
   
  \citet{Mackenzie2006-hb} focuses on understanding how 'software projects' function. The \glspl{sa} produced by (or associated with) a given project, the people who work for the organization that owns the project, and imaginaries  about the future or the purpose of the project are all part of Mackenzie's definition. A software project is made of individual artifacts (both software and others) and relationships. This means including both the particular qualities of each \gls{sa} (how does the current version of a program work and how the next one \textit{will} work) as well as the relationships between each \gls{sa} (how the current version will give way [or not] to the next version) in software. A particular \gls{sa} cannot be usefully separated from the ongoing social, economic, cultural and organizational context in which that \gls{sa}. Software is designed by the people who make it to fit within their view of the world and, simultaneously, to perform the material role that software might usefully fill \citep{Mackenzie2006-hb}. Software is both material and socially defined. 

  In order for a future \gls{sa} to have certain properties, the people making that \gls{sa} it must first imagine those properties in a shared social context. Like \citet{Fleck2012-qr}'s thought collective, people need to imagine futures together. The material work of preparing software to deal with a world is always predicated on needing to imagine the world that software will find itself in. This both leads to collective planning between \glspl{sp} and also means that the material qualities of \glspl{sa} reflect the qualities that their creators believed would be useful in the future. Examining the details of what software does now will tell us about what its creators imagined the would might be.

  % Social footprint of software and digital cultures
  % Sources in this section:
  % Boellstorff, Tom. “The Opportunity to Contribute: Disability and the Digital Entrepreneur.” Information, Communication and Society 22, no. 4 (March 21, 2019): 474–90.
  %   - pushing back against reducing the social to the commercial -or- a higlighting of the social character of commercial work
  % Humphreys, Lee. The Qualified Self: Social Media and the Accounting of Everyday Life. MIT Press, 2018.
  %   - use of digital medium to demonstrate qualificiations
  %   - NOT SURE I'LL USE
  % Jurgenson, Nathan. The Social Photo: On Photography and Social Media. Verso Books, 2019.
  %   - use of media to transmit social qualities
  %   - structural nature of the world as mediated by software giving rise to particular kinds of expression and communication
  
  % todo?:
  % Tiidenberg, Katrin, ed. Selfies: Why We Love (and Hate) Them. Emerald Publishing Limited, 2018.
  %   - more on using media to communicate meaning
  %   - new forms of media facing hostility
  \subsection{The Material Character of Software}
  The material and social characteristics of software are co-constitutive its impact on the world. As Facebook attempted to respond the changing social situation around gender, they experienced a back-and-forth between what was technically possible and socially desirable. They eventually settled on a solution that allowed lacuna to exist in the database in return for a more acceptable list of gender options \citep{Bivens2017-tc}. Softwares flexibility means that nearly any situation can be represented and recorded, but each new piece of software is built on top of an ecosystem. That ecosystem is at the end of a chain of decisions to build one capability over another - this makes certain things easy and other things hard. The database practices Facebook relied on in their early days represented gender as a binary value - a choice which later made it impossible to both offer non-binary genders and have a value in the gender field of the database. Social systems demand material changes in software so that the software can properly reflect the social reality.

  Sociological work on new media supports the idea that there is something unique about how software impacts the world. Many of the ways we socialize through software have pre-digital antecedents, but the unique power of software to create new experiences is clearly visible. You do not have to be making software to feel the power \textit{of} software to structure and channel your life. We are both using media in was that are only possible through digital systems and using digital media to account for ourselves in new ways and to new audiences \citep{Humphreys2018-ge,Jurgenson2019-tl}. Though we have always been different things to different people, our individual social horizons have been so dramatically expanded by networked communication that many more people experience a much greater diversity of social perception than ever before \citep{Dean2010-lk}. The important thing to understand is that these new social realities reflect the new material force of software being brought to bear on society. These new bottles contain old wine, but the bottle is how we get to the wine. It matters. It isn't necessary for software to be attempting to transform the structure of society, its material force drives social transformation even when that transformation isn't imagined by its creators. 


  % Social organizations around software
  % Sources in this section:   
  %   - legitimacy of doing anthropoligical work in purely digital spaces, actual v.s. virtual - all real
  % Cox, Geoff, and Christopher Alex McLean. Speaking Code: Coding as Aesthetic and Political Expression. MIT Press, 2013.
  %   - Background on software studies
  %   - Code as speech but *not* software as speech
  %   - Involves software, but as seen through the act of "coding"
  % Ensmenger, Nathan L. The Computer Boys Take Over: Computers, Programmers, and the Politics of Technical Expertise. MIT Press, 2012.
  %   - Social history of programming and selecting programmers
  %   - More about the particular social and historical shape of programming and programmers
  %   - background for ethnographic work
  % Gabriella Coleman, E. Coding Freedom: The Ethics and Aesthetics of Hacking. Princeton University Press, 2012.
  %   - social idea of hacking and of freedom
  %   - ways of experiencing freedom within software and software groups
  % Kelty, Christopher M. Two Bits: The Cultural Significance of Free Software. Duke University Press, 2008.
  %   - rules of FOSS give rise to certain consistent social rules
  \subsection{Software Cultures are Cultures of Materiality}

  Ethnographic work on communities that are deeply involved with software support the idea that the qualities of software materially impact those who work on it. When \citet{Kelty2008-jm} argues that the open culture of digital spaces follows from the commitment to sharing, he is also suggesting that the material character of technology and the social goals of its builders are necessarily connected. It would be possible to build software \textit{wrong} and undermine the \gls{foss} movements' goals \citep{Kelty2008-jm}. His work underlines the importance of understanding the material qualities of software through describing the impact of those qualities on the social world.

  The history of computers and computer work speaks to the material demands of software and the way that those material characteristics drive social factors. Computer workers gained their reputation for keeping odd hours and existing outside or alongside the traditional power hierarchy because they needed to do work on mainframe computers when those computers weren't doing business calculations \citep{Ensmenger2012-kz}. The particular materiality of computers at that time (that they were singular and could only run one task at a time) gave rise to social characteristics have have somewhat endured to today\sepfootnote{socialcharacter}. Code does not have agency, but its execution (like the performance of a speech) has an efficacy that exceeds its written form. Reading a program is one thing, experiencing its execution is another \citep{Cox2013-zo}.

  There are even pre-computer arguments for seeing computer software as having unusual and deeply important material realities. Hacking, a cultural movement that began before software, is focused on joyful play that elides and subverts objects' intended purpose \citep{Drexler2019-ja,Gabriella_Coleman2012-lq}. It's no accident that hacking has become so associated with digital technology. There is no other medium better suited to both playfulness and immediate access to material impact on the world.

  % Situated knowledges and translation review
   % Said, Edward W. Orientalism. Vintage Books, 1979.
   % pre-
   % Latour, Bruno. “Give Me a Laboratory and I Will Raise the World. En Science Observed: Perspectives on the Social Study of Science (pp. 141-170).” Beverly Hills: Sage Publications, 1983.
   % skip maybe?
   % Haraway, Donna. “Situated Knowledges: The Science Question in Feminism and the Privilege of Partial Perspective.” Feminist Studies: FS 14, no. 3 (1988): 575–99.
   % Harding, Sandra. “Rethinking Standpoint Epistemology: What Is‘ Strong Objectivity?’” The Centennial Review 36, no. 3 (1992): 437–70.
   % Haraway, Donna. “Situated Knowledges: The Science Question in Feminism and the Privilege of Partial Perspective.” Feminist Studies: FS 14, no. 3 (1988): 575–99.
   % Harraway, Donna. “Modest_Witness@Second_Millennium.FemaleMan©_Meets_OncoMouseTM.” Routledge New York, 1997.
   % Fleck, Ludwik. Genesis and Development of a Scientific Fact. University of Chicago Press, 2012.
   % Subramaniam, Banu. Ghost Stories for Darwin: The Science of Variation and the Politics of Diversity. University of Illinois Press, 2014.
   % Traweek, Sharon. Beamtimes and Lifetimes. Harvard University Press, 2009.
  %   - cultural charaters of teams show up in the tools that they build to do their work
   % Eubanks, Virginia. Automating Inequality: How High-Tech Tools Profile, Police, and Punish the Poor. St. Martin’s Press, 2018.
  %   - materialization of human perspectives in computer algorithms
  %   - questioning the projection of agency onto computer code v.s. understanding computer code as "doing the work" of humans
  %   - more examples of tools containing material standpoints (specifically the eugenical quality of population statistics)
   \subsection{Materialized Perspectives}
   \singlespacing
   \epigraph{A model is worked, and it does work}{\textit{\citet[p. 63]{Haraway2016-nc}}}
   \doublespacing

   My account of the pluripotency of software and its particular material impacts begs a question about software: how do we account for the difference between what its makers intend and what software does? Software is epistemic - it functions through selecting and acting on representations of the world. Understanding bias and developing language around bias is a well-studied area in its own right. Bias can persist in knowledge and practice nearly accidentally (syphilis retains a moral dimension to this day) or it can be central to how a field of thought operates (like the defunct field of orientalism) \citep{Fleck2012-qr,Said1979-jw}. 

   The theoretical framework I'm using is \citet{Harding1992-od} and \citet{Haraway1988-nh}'s Standpoint Theory. Standpoint Theory connects the background of people making epistemic decisions to the epistemic representations they select. Without those connections, ideas become what \citet{Haraway1988-nh} calls a god trick: the appearance of knowledge simply \textit{being}, as if it came from nowhere, creating the appearance of objectivity. Truths are found first by people most able to see them. Connecting the circumstances that led them to that discovery strengthens our understanding of that truth. God tricks can also be found in tools. Choosing one way over another without situating (explaining) that choice creates the appearance there is no other possible choice. This allows problematic understandings, like a eugenic understanding of population, to remain within tools without critique \citep{Subramaniam2014-wg}. God tricks also aid interested parties in portraying limited views of complex systems as the ultimate goal. The focus on genes in genetic research was enormously profitable for some, but was not enough to effectuate revolutionary change in the enormously complex system of biological life. \citet{Harraway1997-va,Reardon2017-bo}. Problematic characters of systems are allowed to stay hidden because the problematic qualities are not problematic for those who own the systems. 

   It is not necessary for a system to become problematic before we can learn by investigating standpoints encoded into it. Software, like any other tool, exists to help save human effort. In relieving humans from action, tools must trigger outcomes that might otherwise involve more human choice. The number and nature of choices that can be elided within tools have only grown more diverse with time. The real world work of finding and addressing standpoints in problematic software can be assisted by finding and addressing standpoints that are not themselves problematic. When \citet{Traweek2009-uu} found that philosophical differences between teams of high energy physicists led them to make dramatically different particle detectors, her findings did not need to find a problem to be of use. This project is interested in finding choices that software makes but does not acknowledge. God tricks escape from documentation or acknowledgment. They are useful markers for further investigation and may be useful in explaining how software came to be as we experience it.

  
  % Cheney-Lippold, John. We Are Data: Algorithms and the Making of Our Digital Selves. NYU Press, 2018.
  % todo: 
  % Boellstorff, Tom. Coming of Age in Second Life: An Anthropologist Explores the Virtually Human. Princeton University Press, 2015.
  \subsection{Software Without Ethnography}
  \singlespacing
  \epigraph{[T]he "writing of technology" is by by no means universal; the opaque and stubborn places do not lie simply beneath technology, but are wrapped around and in it}{\textit{\citet[p. 181]{Mackenzie2006-hb}}}
  \doublespacing

  The bulk of sociological work around software has been work that with the material effect of software through its impact on social worlds. This project does not do that. It is conducted with the assumption that omitting social engagement will limit, rather than eliminating, the usefulness of the project. This belief is based on the studies of social outcomes that suggest that software possess a strong material character outside of social construction \citep{Dean2010-lk,Eubanks2018-hc,Humphreys2018-ge,Kelty2008-jm}. It is also based on work that show that once social information has passed into digital systems, that information will be manipulated and used in ways that can be driven as much by the quirks of digital systems as by social goals \citep{Cheney-Lippold2018-lw}.

  Better understanding the nature of software isolated from social situations will enable more complex work that studies both. I'm interested in trying to engage directly with the world \citet[p. 3]{Ullman2012-fq} says you find when you enter the world of computers and "pass through a membrane where the real world [no longer matters."

  % Kitchin, Rob, and Martin Dodge. Code/space: Software and Everyday Life. MIT Press, 2011.
  %   - power of code / software in social space
  %   - creation of spaces through software
  % Cramer, Florian, and Matthew Fuller. “INTERFACE.” In Software Studies: A Lexicon, edited by Matthew Fuller, 149–53. MIT Press, 2008.
  % Yuill, Simon. “INTERRUPT.” In Software Studies: A Lexicon, edited by Matthew Fuller, 161–68. MIT Press, 2008.
  % Cramer, Florian. “LANGUAGE.” In Software Studies: A Lexicon, edited by Matthew Fuller, 168–74. MIT Press, 2008.
  % Crutzen, Cecile, and Erna Kotkamp. “OBJECT ORIENTATION.” In Software Studies: A Lexicon, edited by Matthew Fuller, 200–207. MIT Press, 2008.  
  \subsection{Yes, but what IS Software?}

  Work on the powerful impact of software on our social world (and the power impact of our social world on software) somewhat elides a central question: what \textit{are} \glspl{sa}? How do understand these things? How do they exist in the world? \citet{Cramer2008-cw} focused on the Janus-like qualities of computer languages - forced to fulfill the needs of humans and machines. While \citet{Cramer2008-cw} notes that the only actual computer language are the machine languages spoken by computer chips, he also points out that the primary concern of general purpose 'computer' languages\sepfootnote{dslnote} (including \Gls{pyt}) is attending to different human priorities. This is especially true because all general purpose languages can be reduced to one another, so any difference in functionality come only from the difference in how well the language suits the preferences of the human writing it. \citet{Yuill2008-by} talks about the character of the ecosystem of programs and the centrality of "interrupts" to modern system design. Most software written today is written with the expectation that it may be interrupted. \Gls{sa} live in the same world as we do, even if their way of existing is very different. Understanding how they impact us demands we understand how they exist.

  I find \citet{Latour1999-ui}'s blackboxes a good basis\sepfootnote{csblackbox} for thinking about \glspl{sa} and how they functions in the world. Blackboxes are enclosed units that accomplish a task without exposing outsiders to the interior details of how it works. Like \glspl{sa}, blackboxes can contain almost anything with any level of complexity, but to those outside the box it appears singular and cohesive. It's only in failure that blackboxes are differentiated from other objects. Once a blackbox failed, the fact that it has components becomes socially available and socially unavoidable. What is inside a blackbox varies, but they often contain other blackboxes. Like blackboxes, \glspl{sa} are complicated objects that move around the world with socially invisible internals.

  % Intervention
  % Zuiderent-Jerak, Teun. Situated Intervention: Sociological Experiments in Health Care. MIT Press, 2015.
  % Sismondo, Sergio. “Science and Technology Studies and an Engaged Program.” The Handbook of Science and Technology Studies 3 (2008): 13–32.
  
  
  \subsection{Taking Action}
  \singlespacing
  \epigraph{We live in capitalism, its power seems inescapable – but then, so did the divine right of kings. Any human power can be resisted and changed by human beings.}{\textit{\citet{Le_Guin2014-ue}}}
  \doublespacing
  
  So in one hand I hold Latour's blackboxes and in the other \citet{Harding1992-od} and \citet{Haraway1988-nh}'s Standpoint Theory. As \citet{Mackenzie2006-hb} recounts, the process that produces software is complex, relational, and heterogeneous. It defies simple analysis and it, itself, a deeply social thing that will also produce \glspl{sa} that can be mechanistically executed \citep{Cox2013-zo}. Yet we are beset on all sides with software whose bias is plain for all the world to see \citep{Bivens2017-tc,Dean2010-lk,Elish2019-ls,Eubanks2018-hc,Schull2012-nc}. How can our academic theories be used to address these practical concerns?

  The first step to to try doing real work on real software. Social scientists can and must take sides on the issues they study and intervene to gain firsthand knowledge on the subjects we claim be experts on. Medical professionals use interventions (and their outcomes) as tools to generate knowledge and test assumptions \citep{Zuiderent-Jerak2015-go}. My own pre-academic experience as a software engineer supports the knowledge generating power of careful and measured intervention. I will be looking for materialized perspectives to investigate within my site and, though engaging with the source code and the \glspl{sa} of that site, learn something about how we might address materialized bias in situations where the stakes are higher.

  \newpage
% =======================================
% =  __  __      _   _               _  =
% = |  \/  | ___| |_| |__   ___   __| | =
% = | |\/| |/ _ \ __| '_ \ / _ \ / _` | =
% = | |  | |  __/ |_| | | | (_) | (_| | =
% = |_|  |_|\___|\__|_| |_|\___/ \__,_| =
% =======================================

  
  \section{The object of study}
  \subsection{Software in General}
  Before diving in, I should explain some ideas I am going to be using frequently. Some of these are rooted in the literature I just went over and some are simply particular understandings of common ideas.

  A \textbf{\gls{sa}} is any arrangement of machine code, scripts, configuration files, network connections, or other constitutive resources that can be executed by a computer. A collection of such resources become a software artifact as soon as the computer can be passed the bundle of elements and carry out the instructions encoded within the artifact ('execute' it). Each component is equally a part of the artifact, but some components are given power over others in the assembly process, leading some components to have more influence than others. Software is largely distributed in the form of artifacts, though development generally takes place on individual components. 

  A \textbf{\gls{sp}} is, as \citet{Mackenzie2006-hb} says, the extended social, material and organizational structure that produces "software." These projects are not unified by the production of \glspl{sa} per-se. They can also be focused on creating standards that allow \glspl{sa} to interact, or managing the way that software warps space \citep{Kitchin2011-af}. Though software projects may produce \glspl{sa}, they do not have to. Technical organizations that create standards that other software projects might follow are also software projects. The same is true for professional organizations that aim to shape the culture of software production. 

  Throughout this paper I will speak about the 'intent' or 'actions' of a \gls{sp}. This is not an attempt to say that the \gls{sp} totally controls the actions of individuals involved with it, but a recognition that the people who are part of a \gls{sp} are acting both as themselves and also as agents of the larger undertaking. When people are acting as agents of a larger project they will certainly have their own thoughts an opinions about any particular action, but the body of work they contributing to is more the result of the social context inside the project than it is the particular person who happens to be acting.

  Finally, an \textbf{\gls{ap}} is the arranging of elements of a \gls{sa} in such a way that the computer can execute them. Such a process always happens (even if it is just placing machine language instructions sequentially in memory) before the computer can execute the \gls{sa}. This process is often carried out by one or more \gls{sa}(s). Sometimes these \glspl{sa} are part of the software project making the \gls{sa}, sometimes the are tools shared between many software projects (generally with configuration supplied by this particular project).

  \subsection{Method}
  This project was designed to take advantage of our ability to learn about things by changing them. It's a practice I'm familiar with from my personal background in software engineering, but it's a practice that many other disciplines use \citep{Zuiderent-Jerak2015-go}. 

  My plan was to engage with my site (\gls{EDEN}) as both a programmer and a user. I would learn its background, learn how to set it up, learn how it works, learn how it might fail. This happened on multiple levels. First on the level where I interact with \glspl{sp} and \glspl{sa} as blackboxes, without looking inside to understand how they work. Second, going back and investigating what was happening on the inside (and disrupting the idea that \gls{sa} are blackboxes). Third, writing my own small \gls{sp} that produces its own \gls{sa} in reaction to what I found in my investigations. The mix of connecting theory to existing work and writing my own response to those theoretical connections will, I hope, yield novel results.

  Because of my limited time and resources, this my engagement is not ethnographic\footnotemark. I hope the experiment of applying social theory to software outside of the social context that software was created in is useful. I hope that whatever I learn can be applied to software in its full social context, but I could be wrong.

  \footnotetext{As I explain later, I had some contact with the \gls{EDEN} team that was influential on my thinking. I hope I draw conclusions from my own understanding of their words and I avoid characterizing the culture inside \gls{EDEN}.} 
% =====================================================
% =  ____  _ _          _______ ____  _____ _   ___   =
% = / ___|(_) |_ ___   / / ____|  _ \| ____| \ | \ \  =
% = \___ \| | __/ _ \ | ||  _| | | | |  _| |  \| || | =
% =  ___) | | ||  __/ | || |___| |_| | |___| |\  || | =
% = |____/|_|\__\___| | ||_____|____/|_____|_| \_|| | =
% =                    \_\                       /_/  =
% =====================================================

  \subsection{What is EDEN?}
  \singlespacing
  \epigraph{That virtual worlds are places means they can be fieldsites;}{\textit{\citet[p. 107]{Boellstorff2015-al}}}
  \doublespacing

   The particular piece of software I engaged with is \acrfull{eden}. \gls{EDEN} was originally created by a coalition of Sri Lankan \acrfull{ict} companies in response to the 2004 Indian Ocean Earthquake and Tsunami. The project is now managed by the nonprofit \gls{sf}. It has been used in numerous disaster responses since 2004 and is also used by a number of organizations for managing resources outside of any specific disaster \citep{Sahana_Foundation_undated-hl}. \gls{EDEN}'s functionality includes tracking and managing resources. Its power comes from the enormous range and detail of the information it knows how to track. \gls{EDEN} has indexes for organizations, people, projects, events, facilities, supplies, documents, possible scenarios, and events. It also provides on-platform messaging and special tools for the management of emergency shelters \citep{Sahana_Foundation2011-od}. Each resource is meant to be connected to other resources: people work for organizations, projects are run by organizations and are associated with people, facilities are linked with projects and organizations and other resources like supplies and verticals. The flexibility of the software means that it can be used to manage a single organization, multiple organization, or used as a hub for many different groups coordinating around shared goals but without a single central authority. The system tries to be as comprehensive as possible and to use the same entity for resource and system management. The entries for people can act as a simple rolodex and will also double, if desired, as that person's account within \gls{EDEN}. The interface for managing goods is the same interface for managing access to the management system. \gls{EDEN} is structured to allow the same people who do the day-to-day work of the organization to administer the system that manages the organizations' resources. It is also designed to be an effective tool for people at all levels: administrators can track where people and supplies are allocated and volunteers can access their assignments, documents and information about how to use resources. The range and breadth of its functions are impressive and seem to reflect \gls{EDEN}'s repeated use as a response to disasters.

   \subsubsection{Why EDEN}
   I selected \gls{EDEN} for this project for two reasons: it is not obviously problematic and it is a \acrfull{foss} software project in a language I am familiar with. The idea of improving software is difficult and dependent on the question "for whom"? I lack the experience and expertise to say that I think \gls{EDEN} is \textit{good}. Assessing if software is "good" or "bad" is not straightforward. Simply examining \glspl{sa} in isolation tell us very little. It's only through engaging with one or more \gls{sa}(s) as they exist in the lived world and contextualizing that with detailed ethnographic work that it's possible to start making value judgments \citep{Eubanks2018-hc,Schull2012-nc}. Those judgments wouldn't be universal, of course, but would be about a particular population. So this paper is done from the perspective that \gls{EDEN} seems fine to me. The project is interested in the technical details of how the \gls{EDEN} \gls{sa} emerges out of its \gls{sp} and how agency is modified by that process. 

   I chose these limits, in part, because the the final project for a masters degree is relatively short. I performed this work over during nine graduate units of class work. I didn't feel like I had time to find \textit{and} fix flaws. For similar reasons, I selected a project that I would not have to apply to gain access to its source code, and also one written in a language I was already familiar with\footnotemark.

   \footnotetext{Even so, I managed to make several serious errors in the software I wrote for this project.}

   Finally, though project does not investigate the social world of \gls{EDEN}, it also does not imagine that technical qualities exist independently from social ones. \citet{Gabriella_Coleman2012-lq} and \citet{Kelty2008-jm} have both compellingly shown that the social and technical co-produce each other. However, we can still usefully speak about qualities that technical systems have and how those qualities impact our lived experience. That this project has obvious extensions in the social and ethnographic realm is a strength and declining to investigate them should be understood as a concession to time.

   \subsubsection{Out of Many, One}
   Wherever possible, \gls{EDEN} uses \acrshort{foss} technologies. The language it is written in, the libraries it relies on to provide functionality, the tools it uses to support its functionality, and its recommended operating system are all both open source and available at no cost. \gls{EDEN} will generally operate on top of non-\acrshort{foss} systems like Microsoft Windows, but the team doesn't prioritize systems outside of the \acrshort{foss} systems they develop and test on \citep{Sahana_Foundation2015-zs}. \acrshort{foss} is both a philosophy and a system of development that has material impacts \citep{Kelty2008-jm}. One of the effects of \gls{EDEN} committing to use the \acrshort{foss} ecosystem is it makes my form of engagement possible. Though it is possible to examine compiled machine code and draw some conclusions about the intent and process that assembled it, such a project would be far outside my capabilities. Instead, the source code of \gls{EDEN}, web2py (the \gls{webframework} \gls{EDEN} uses), Python (the language \gls{EDEN} and web2py are written in) and all of the libraries used by the project are open source. Their preferred databases (MySql or PostgreSQL) are also fully open source projects. 

   Open source projects often don't just publish their latest source code. They also maintain full histories of who changed what, when it was changed. Often they also record why \textit{that particular} change was made over any other possible change \citep[p. 13-16]{Chacon2014-im}. Though there are many systems for this, the most popular tool is Git, and it is the tool the \gls{EDEN} uses. 

   \subsubsection{Enumerating The Specifics of my Engagement}
   
   \gls{EDEN} has a large number of individual components. Executing a fully assembled \gls{EDEN} \gls{sa} would involve a number of \Gls{pyt} libraries, a number of other \glspl{sa} (a compatible database, various compiled libraries) and a large amount of non-python code (such as the javascript contained in the \gls{EDEN} repository). Engaging all of this code would certainly possible if given enough time, but it did not seem possible given the constraints of my project. So, I have chosen to focus on a limited number of \Gls{pyt} libraries used by \gls{EDEN}\sepfootnote{libversions}. 

  % much thanks to https://www.tablesgenerator.com/#
  \begin{table}[ht]
  \caption{Purpose of libraries}
  \label{tab:table-1}
  % https://stackoverflow.com/questions/2895780/how-to-code-tables-with-multi-line-cells
  \begin{tabular}{|l|p{.65\textwidth}|}
  \hline
  Component Name & Brief Description of Functionality                                                 \\ \hline
  Eden           & Implements the logic for managing the various resources tracked by Eden            \\ \hline
  web2py         & Provides interfaces to easily provide web services over HTTP and related protocols \\ \hline
  pydal          & Library for interacting with a database using Python objects instead of raw text   \\ \hline
  yatl           & Library for generating HTML content through templates                              \\ \hline
  \end{tabular}
% This MUST be at the start of the line without spaces before it or it will cause an error
% https://tex.stackexchange.com/questions/62278/problems-with-endfloat-package
\end{table}
  
  The \gls{EDEN} component I chose because it contains all the \Gls{pyt} code written by the \gls{EDEN} project. It is also placed in the most powerful position inside an \gls{EDEN} \gls{sa}. It is the starting point for nearly every operation within the \gls{EDEN} \gls{sa} and so including it is essential.

  I do not know that the other three components are the most influential components. That judgment would require a good deal more study of \gls{EDEN} than I have done. I would also need to develop a method to compare functionality across programming language boundaries. For the same reason, my analysis omits all non-\Gls{pyt} components (databases, etc). This is another omission caused by time limits as opposed to a considered opinion that those components contribute less to the behavior of \gls{EDEN}. 

  I focus on the components in Table \ref{tab:table-1} because they were the ones I interacted with most during my investigation of the \gls{EDEN} project. There are also formal links between the projects: the \gls{EDEN} project identifies web2py as its \gls{webframework} and web2py identifies both pdal (as DAL) and yatl components of itself \citep{Di_Pierro2020-wn,Sahana_Foundation_undated-ww}. That each component includes prominent references in their documentation to the other components suggests this is a suitable set to focus on.

  It's foreseeable, but unlikely, that my conclusions might not extend to the entire \gls{sa} / project. While it is likely that the \gls{EDEN} component is the most important part of the \gls{EDEN}, small changes elsewhere can have large impacts. I will show how much of the code that impacts elements within one Component resides in another component. It may be useful to refer to the size (in lines of code) of the various libraries to get a sense of their relative size. 

  \begin{table}[ht]
  \caption{Size of Libraries}
  \label{tab:table-2}
  \begin{tabular}{|l|c|c|}
  \hline
  Component Name & Lines of Python Code & Lines of Non-python Code \\ \hline
  Eden           & 469,117              & 793,661                  \\ \hline
  web2py         & 103,593              & 41,793                   \\ \hline
  pydal          & 24,518               & 416                      \\ \hline
  yatl           & 1.131                & 69                       \\ \hline
  \end{tabular}
\end{table}

  In general, the number of lines of code in a piece of software will correspond to that programs capabilities. Lines of code can also be used to roughly compare the relative contributions of components, especially when all components are written in the same language. It's possible to write lines of code that have no meaningful impact and it's easy to imagine scenarios where single lines of code make thousands meaningless (by turning off a feature, for instance), but I imagine such exceptional circumstances would be evenly distributed. Similarly -  it is much easier to imagine accomplishing the work done in fifteen lines of code with ten different lines, but it is much harder to imagine doing the work of fifteen hundred lines of code in a thousand. Efficient code is efficient, but there is a limit.
  \newpage

% ====================================
% =  ____                 _ _        =
% = |  _ \ ___  ___ _   _| | |_ ___  =
% = | |_) / _ \/ __| | | | | __/ __| =
% = |  _ <  __/\__ \ |_| | | |_\__ \ =
% = |_| \_\___||___/\__,_|_|\__|___/ =
% ====================================

  \section{Chronicling my Exploration of EDEN}
  Now I want to talk about some of what I saw when I was investigating the \gls{EDEN} \gls{sp}. I'll go over my experience of trying to assemble a working \gls{EDEN} \gls{sa}, a brief and revealing conversation with developers, and finally talk about the software I wrote to try and respond to what I saw.

  \subsection{What is a Validator?}
  Though I looked at a lot of the capabilities of \gls{EDEN} \glspl{sa}, both of my detailed anecdotes involved a single element of \gls{EDEN}: Validators. Validators are \Gls{pyt} objects that perform a small but essential role that helps humans reliably interact with machines. Many operations within \glspl{sa} will only work on certain kinds of data. If you ask a human to mathematically multiply one word by another word they will just be confused. If you ask a \gls{sa} to do the same thing, it will generally cease functioning. Ensuring that data inside a \gls{sa} is of the expected format is important. Along with the core purpose of checking the format of data, Validators do a great deal of housekeeping: informing the user of problems, passing correct data to the next part of the \gls{sa}, etc.

  Each Validator is a \Gls{pyt} object that ensures that any data it is given fits a particular format. I will focus most on one that checks the format of phone numbers, but some others checks for latitude, or if the data is a UTC date \& time. Not all of the \Gls{pyt} code that defines the behavior of these objects exists in the \gls{EDEN} source code. The original Validator class exists in the \gls{w2p} source code. The code in \gls{w2p} defines how all Validators behave in common situations. A coding practice called inheritance allows the creating new \Gls{pyt} objects which inherit all the qualities of another \Gls{pyt} object, with the ability to override any aspect of that behavior. The \gls{w2p} Validator object is designed to be inherited from. The section of the \gls{w2p} Validator that approves or rejects a particular piece of data is isolated, so other \gls{sp} can override the core function and keep all the housekeeping. The design approach of transmitting behavior without forcing the developer to remake it is very Latour's blackboxes\footnotemark.

  \footnotetext{Within computer science, abstraction is a key concept. Components present an interface that will produce certain outputs for certain inputs. Developers are encouraged to treat this as a blackbox as, unless they need to understand the internal details of the component they are using, understanding it is seen as unlikely to be useful.}

  The \gls{EDEN} Validators inherit from the \gls{w2p} Validator. This means that the \gls{EDEN} source code only needs to write source code that approves or rejects data. This is \textit{both} a social and material quality. There is a social division where certain aspects are provided by the \gls{w2p} project and certain aspects are provided by the \gls{EDEN} project. There is also a material reality where, when \Gls{pyt} assembles an \gls{EDEN} \gls{sa}, the \gls{EDEN} source code and \gls{w2p} are combined into a single \Gls{pyt} object.

% ==============================================================================
% =     _                           _     _         _____                      =   
% =    / \   ___ ___  ___ _ __ ___ | |__ | |_   _  | ____|_ __ _ __ ___  _ __  =
% =   / _ \ / __/ __|/ _ \ '_ ` _ \| '_ \| | | | | |  _| | '__| '__/ _ \| '__| =
% =  / ___ \\__ \__ \  __/ | | | | | |_) | | |_| | | |___| |  | | | (_) | |    =
% = /_/   \_\___/___/\___|_| |_| |_|_.__/|_|\__, | |_____|_|  |_|  \___/|_|    =
% =                                         |___/                              =
% ==============================================================================

  \subsection{Assembling a Software Artifact}
  \singlespacing
  \epigraph{If you wish to make an apple pie from scratch, you must first invent the universe.}{\textit{\citet{Sagan2010-th}}}
  \doublespacing
  
  My work began, as all software work does, with collecting the components needed to assemble an \gls{EDEN} \gls{sa}. The necessary components are heterogeneous. They include \Gls{pyt} libraries, a Database \gls{sa}, a web-server \gls{sa}, a \gls{sa} that manages communication between the \Gls{pyt} \gls{sa} and the web-server. It is a large and complex ecosystem. Some of the components are \Gls{pyt} source code, some require their own \gls{ap} to create the needed \gls{sa}, some are distributed as \glspl{sa}. One can learn a lot by just reading source code and design documents and descriptions of how \gls{EDEN} is expected to work, but there's no replacement for the process of seeing how the various elements fit together. 

  The \gls{EDEN} project has a few different resources that help developers get a copy of an \gls{EDEN} \gls{sa}. They provide a \gls{vm} which will install a \gls{EDEN} in its own environment. They also publish a number of scripts that, when executed, create a \gls{EDEN} \gls{sa}. Each script is differentiated through the primary components it uses and / or the digital environment it is designed to operate in \citep{Sahana_Foundation2018-eq}. 

  All of the scripts and guides published by the \gls{EDEN} project recommend components from a particular time period, but also recommend using the latest version of the \gls{EDEN} source code. In particular, \gls{EDEN} recommends using components from around 2014 \citep{Canonical2020-ru,Sahana_Foundation2018-eq}. Many of the \glspl{sp} that make those components no longer support the versions from 2014 and therefore say no one should use them\sepfootnote{softwarereleases}. Some users have written their own scripts with newer components and shared them on \gls{EDEN} the mailing list, but I could not find any official scripts or guides that offered updated components.

  I thought this focus on an older set of components was interesting. Officially, \gls{EDEN} supports a range of components released between 2010 and 2020. They have also mentioned that they are ending support for older versions of \Gls{pyt} \citep{Konig2019-ya}. Still, none of their official guides describe how to use any components other than the set from 2014. Even after announcing a move from \Gls{pyt} 2.7 to 3.5+, their official assembly instructions all use \Gls{pyt} 2.7. In addition, they never comment on their choice to focus on this particular set of components. I thought this was especially odd given that so many of the components makers recommended they not be used. This was the first unacknowledged materialized perspective I found and one of the first places I decided to explore more deeply. 

  \subsubsection{Investigating Assembly}
  I decided that the way I wanted to approach investigating the selection of components was to try and use a different, more recent set. My general goal was to select component versions that were still supported by their \glspl{sp}. However, as soon as I embraced that goal, I encountered an unexpected problem: the \gls{EDEN} component list is not comprehensive.

  When the \gls{EDEN} project lists what component versions they support, they do not list every single component. To do that, they would need to release a full script or guide that lists every individual library. This put me in a strange position once I selected \Gls{pyt} 3.6.9 (released 2019) as my first component. The only official \gls{EDEN} resource I could find are the installation scripts with component lists for \Gls{pyt} 2.7 \citep{Sahana_Foundation2018-eq}. This meant that I needed to go, component by component, checking to see if that particular component was compatible with \Gls{pyt} 3.6.9.

  It turns out, unsurprisingly, that 6 years is a long time in the world of \glspl{sp}. Though some \gls{EDEN} components had not needed an update since the versions released in 2014, many did. In addition, some components were only compatible with \Gls{pyt} 2.7.X and I needed to find replacements to work with 3.6.X. This problem is not new! \Gls{pyt} has been engaged in a slow, uneven journey away from 2.7 and towards the version 3 \glspl{sa} for over a decade. There are plentiful resources for people searching for newer versions of components that only support 2.7.

  Still, the right choice isn't always clear. The assembly of \gls{EDEN} is a heterogeneous process: try to install some \Gls{pyt} components, investigate problems, install newly discovered requirements or replace components that won't install. It's a mix of running scripts that are part of the \gls{EDEN} assembly instructions, running scripts that assemble the components \gls{EDEN} uses, and manual work of searching for new options when a dead end is reached.

  That being said, I generally selected the most recent stable version\sepfootnote{componenttrajectory} of a given component, treating them like blackboxes that should continue to behave similarly \citep{Latour1999-ui}. In most cases this worked easily, but it had one interesting and revealing issue. When it came to \gls{EDEN}'s most prominent non-\gls{EDEN} component, \gls{w2p}, I selected their latest stable release (2.18.5). This was a large change from the version supported in some \gls{EDEN} installation scripts, but other \gls{EDEN} installation scripts used it\footnotemark. I did not expect problems.

  \footnotetext{As we will see, they made some small modifications to it that I missed.}

  \subsubsection{Relationship Trouble}
  By all appearances I had a working \gls{EDEN} \gls{sa}. I could navigate around, look at all the categories of resource \gls{EDEN} tracks. I could log into the accounts that \gls{EDEN} inserts directly into the database. The problem came when I tried to add anything new to the system: \gls{EDEN} would crash. The method used to provide \Gls{pyt} web services like \gls{EDEN} mean that, even if it crashes when you try do to one thing, another copy of the \gls{EDEN} \gls{sa} will be created to keep trying to answer requests. However, my \gls{sa} wasn't much use if I couldn't enter any information. \gls{EDEN}, as I said, is a system for managing resources and I wasn't able to add any resources.

  So what was going wrong? It turned out I was having my first encounter with the Validators that I described in [\textbf{What is a Validator?}]. Specifically, I was seeing what a relationship failure between different \glspl{sp} looked like \citep{Mackenzie2006-hb}. It turns out that one of the changes that the \gls{w2p} project made was changing how they invoked part of the Validator that accepts or rejects a datum. They attempted to continue to support the old way of doing it, but there was a mistake in that attempt. This broken relationship meant that the \gls{sa} I had successfully assembled had extremely limited capabilities. All of the systems for adding data through the \gls{sa} used the Validators, so any attempt to add data would fail\footnotemark.

  \footnotetext{This probably isn't totally true. \gls{EDEN} implements some Validators of their own, but they also use many of the Validators sub-types that \gls{w2p} provides. There may be some places where, because there are no \gls{EDEN} Validators, data can be entered - but I did not find any.}

  The problem, it turns out, is a single line of code in \gls{w2p}. The \gls{w2p} team had thought they had repaired the problem before release, and a developer from the \gls{EDEN} project submitted a fix\footnotemark for the problem to the \gls{w2p} project a day after the release of 2.18.5. This seems like a clear example of a mistake and one that was quickly repaired. 

  \footnotetext{The main script I was following contained a fix for this problem in a slightly more compact form. I'm embarrassed to say that I missed it, but I think it led me down an interesting path.}

  It's also a very interesting mistake. I accidentally created an object that crystallizes the moment that two projects fell out of relationship with one another. All \gls{w2p} releases prior to and after 2.18.5 work. It's also an error that can only be seen after assembly. Both \gls{w2p} and \gls{EDEN} work perfectly well on their own. The new method \gls{w2p} created for invoking the core behavior of Validators looks good. The Validator code in \gls{EDEN} likewise does exactly what it's intended to do. Like a puzzle, both pieces appear well shaped and have no obvious defect until you try to combine them.

  \subsubsection{Reflecting on A Relationship Break}
  This left me pondering several observations that I want to raise before moving on with results.

  First, whose agency is being expressed here? There is extensive sociological work on how tools affect the world in ways that agree with the creators intent \citep{Mackenzie2006-hb,Subramaniam2014-wg}. But that doesn't seem to be what is happening here. Neither the \gls{EDEN} project nor the \gls{w2p} project intended to break the relationship. I wasn't trying to either. Of course - mistakes happen! This is clearly a mistake, but a mistake that had enormous material consequences on the behavior of the \gls{EDEN} \gls{sa}. This problem, which can be fixed by changing a single line of code, subverted the intent of both \gls{EDEN} and \gls{w2p} teams. I suspect that there is a more complicated relationships between agency and \gls{sa} than in other tools.

  Second, this doesn't quite fit with the idea I started with that \glspl{sa} are blackboxes. In one sense, the \gls{EDEN} \gls{sa} looks just like a blackbox: a uniform exterior with a complex interior. However, it also subverts the idea of blackboxes containing other blackboxes. Where is this problem? In the \gls{EDEN} component blackbox? If any other version of \gls{w2p} would work, are the contents socially invisible? This seems less like an object whose complex contents can remain socially invisible than something whose identity derives from the exact combination of its components.

  Finally, my experience of needing to select particular components makes me reflect on the choice by the \gls{EDEN} project to recommend older components. I suspect the recommend the older components because it's more predictable, less likely to have unexpected interactions like this. Of course, this whole experience is about the unexpected. The \gls{w2p} team did not notice a problem that \gls{EDEN} noticed immediately. The problem was equally discoverable by both teams, anyone can build an \gls{EDEN} \gls{sa}. What is the right way to think about the extremely large number of possible components? What does it say about software?

  For now, let's move on to when I have a more-or-less a fully functional \gls{sa} and my next encounter with Validators.
  \newpage

% =======================================================================
% =  __  __                                  _____                      =
% = |  \/  | ___  ___ ___  __ _  __ _  ___  | ____|_ __ _ __ ___  _ __  =
% = | |\/| |/ _ \/ __/ __|/ _` |/ _` |/ _ \ |  _| | '__| '__/ _ \| '__| =
% = | |  | |  __/\__ \__ \ (_| | (_| |  __/ | |___| |  | | | (_) | |    =
% = |_|  |_|\___||___/___/\__,_|\__, |\___| |_____|_|  |_|  \___/|_|    =
% =                             |___/                                   =
% =======================================================================

   \subsection{Reckoning with Failures}
   After assembling a fully functional \gls{sa} I began to explore how my \gls{EDEN} \gls{sa} functions on its own terms. I did this by entering the high level details of this project into \gls{EDEN}'s project tracking feature. This included creating an entry for Drexel University's Center for Science, Technology and Society, an entry for my project within that organization, and an entry for myself as the manager of that project. During this process I encountered both a bug in the source code of \gls{EDEN} and another unacknowledged materialized perspective. 

   The form that \gls{EDEN} uses to record the details of an individual has a field for a phone number. There's no indication on the form that this number number should be formatted in a particular way. I didn't think anything the lack of instruction until \gls{EDEN} informed me that I needed to "Enter a valid phone number" \citep{Konig2020-yx}. I didn't understand because my phone number is valid - it's a phone number. I had followed the common conventions for phone number formats ('xxx.xxx.xxxx' as well as 'xxx-xxx-xxxx' and 'xxxxxxxxxx'). The \gls{sa} wasn't making sense, I needed to look at the source code. When I investigated the the \gls{EDEN} part of the source code I discovered that I was back in the Validators. This phone number Validator had a setting that I was not aware of to require the phone number in international format (beginning with a '+' and a country code). Also, the Validator seemed to have a bug in its message code - it was saying the wrong thing (complaining that the input was not a phone number instead of complaining it was not an international phone number). Now knowing what to do, I entered the phone number in the format the Validator expected and created my profile.

   
   This seemed like an opportunity to investigate. It appears, if one only has access to the \gls{sa}, that there are two layers of unacknowledged materialized perspectives: requiring international format numbers without mentioning it and treating all non-international numbers as invalid. Because I have access to the source code I can tell that the \gls{EDEN} project intends to require international numbers and the confusing error message is an error. However, I think this problem highlights the power of perspective and I want to expand on it a little.


   \subsubsection{A Matter Of Perspective}
   First, imagine if the \gls{sa} returned the right error message. It told users their number needed to be international (instead of valid). When the user enters a number without a country code, the \gls{EDEN} \gls{sa} will reject their number and tell them it must be international. The user is confused about what the \gls{sa} needs to allow them to advance - do they respond to the error message or the field asking for any phone number?

   Second, we can imagine an \gls{EDEN} \gls{sa} that asks a user to use the international format. But, if the user leaves the code off, they are told their number is invalid. Again the user would be confused about what the \gls{sa} is seeking: does their number need to be put in the international format, or is there something that makes the number they entered "invalid"?

   I can tell, by looking at the source code of the \gls{EDEN} Validator, the difference between returning an incorrect error message and omitting a message to the user about a requirement. But I can only tell because I can see the source code. For a user who only has access to the assembled \gls{sa}, both situations come down to the \gls{sa} sending different messages and the user needing to decide which message to 'trust' in order to advance. We can start to see here the difference in how users and developers, those who can see inside the \gls{sa} and are stuck outside, interact differently with these objects. 

   \subsubsection{Seeing Eden Through Another's Eyes}
   So, after discovering it, I decided to see what I could learn by contributing a bug fix to the \gls{EDEN} project and fixing this (extremely minor) bug. I wrote and submitted a small change to the \gls{EDEN} components source code that intended to fix the problem. Unfortunately, much to my embarrassment, my fix unintentionally changed other behavior - it incorrectly eliminated another error message that could, in certain circumstances, be displayed to the user. As a result my fix was rejected, though once the \gls{EDEN} project was aware of the bug they immediately wrote their own fix \citep{Konig2020-ey}.

   In the discussion about why my fix wasn't accepted, one of the \gls{EDEN} project developers said several interesting things about software.

   \begin{quote}
   "And no, there's no test case as to what error message exactly is returned.

   Whilst we could unit-test it routinely, \textbf{regression-testing too much detail can add excessive rigor thus make development harder}. \textbf{A wrong error message is an annoyance, but neither does it break the intended functionality}, nor does it jeopardize the data integrity - so it is low priority and should probably not be CI-tested." \citep{Konig2020-yx}\sepfootnote{longquote}
   \end{quote}

   There are a couple of interesting ideas in this quote that, I think, can help us understand how software (or at least the \gls{EDEN} project) functions. First, there's the idea that testing too much could be bad. This might come from the extra work of writing tests, the ongoing work of keeping tests up to date as messages change, or standards involving all tests passing at all times\sepfootnote{ethnographylimits}. In any case, the decision to limit testing is also a decision to, in practice, pay attention to a particular subset of all \gls{sa} behavior. It's also a decision to choose not to watch some behavior of \gls{EDEN} \glspl{sa} that the developers know users will see. Development would demand more attention if the full scope of possible functionality was checked. So, it's better to be ignorant about some less important behavior.

   The second idea is that accepting or rejecting the number based on its use of an international country code (which it does successfully) and informing the user of what they need to do should be considered separately. That the Validator works if it accepts numbers in the desired format and rejects others. The message the Validator sends matters less. This seems to be a world view that, if any component is expected to have multiple outputs for a given input, each output should be considered working or non-working separately and that there is a hierarchy to which functions are more or less important. 
   \newpage
% =============================================================================
% =  __  __         ___       _                            _   _              =
% = |  \/  |_   _  |_ _|_ __ | |_ ___ _ ____   _____ _ __ | |_(_) ___  _ __   =
% = | |\/| | | | |  | || '_ \| __/ _ \ '__\ \ / / _ \ '_ \| __| |/ _ \| '_ \  =
% = | |  | | |_| |  | || | | | ||  __/ |   \ V /  __/ | | | |_| | (_) | | | | =
% = |_|  |_|\__, | |___|_| |_|\__\___|_|    \_/ \___|_| |_|\__|_|\___/|_| |_| =
% =         |___/                                                             =
% =============================================================================

  \section{Taking Action}
  Now that I've covered a couple of highlights from my explorations of \gls{EDEN}, I want to talk about the software I wrote in response to what I found.  

  A primary motivation for this project was the sense that Standpoint Theory felt like an unfinished tool for the practice it was critiquing. Our ability to see the truth is impacted by our life circumstances. Individuals cannot identify falsities that appear true because the falsity and the person exist within a power structure that supports the view that the falsity is true \citep{Haraway1988-nh}. Standpoint Theory explains the mechanism through which we inappropriately attribute objectivity to things that are actually compelling because of personal background, but it is of little use in understanding to affect changes to the things. 

  If \glspl{sp} believe false things because of the collective backgrounds of their members, convincing them that those views are false does not help them address it. The same cultural limitations that hem in action would also hamper any effort to escape from those cultural limitations.

  So I hoped to find ways to better address the standpoints of development teams. I thought that starting from software elements that had unacknowledged materialized perspectives would help me imagine a response in software. I hoped to give teams tools to explicate what is happening in some way. I took a wide view of what "address" might mean - it could be a direct improvement, it could be a way of highlighting alternatives, it could be tools that help teams see the boundaries of their own perspective. As I worked, however, it became clear that without an ethnographic engagement this goal faced real difficulties.

  \subsection{Improved for Who}
  No sociological work is needed to learn how to improve the \gls{EDEN} \gls{sa} from the point of view of the \gls{EDEN} project. If I want to improve \gls{EDEN} for them they have instructions. The \gls{EDEN} project Github has a list of things they would like to change and an open invitation for anyone to help them. The project also communicates about their long term plans. They are open and encouraging to outsiders and users alike who might want to comment or ask for changes in future \gls{EDEN} \glspl{sa}. These conversations often require a certain level of technical expertise to fully participate in, but the project still has clear ideas about what it means to improve the project. The problem of software expertise shrinking who has access to software is common to \acrshort{foss} \glspl{sp}, but those projects can still communicate to the population who can understand them \citep{Drexler2019-tk,Kelty2008-jm}. 

  My decision not to engage with populations that use \gls{EDEN} \gls{sa} means that I have no reason to believe I can understand the interests or needs of people who might use the software. Even taking \citet{Kitchin2011-af} into account and proceeding from the point of view that the software might impact space in particular ways outside of particular individuals is not much help. I am not experiencing a disaster\footnotemark and even if I was it's unlikely that I have a sense of how \gls{EDEN}, among all the factors, was impacting me. That kind of work benefits from longitudinal work that I am not planning or prepared to do.

  \footnotetext{Part way through this project I, and most everyone else in the world, experienced the Covid-19 pandemic. As of this writing, the pandemic is ongoing. However, I have no way of appreciating how \gls{EDEN} would impact me were it mediating my access to supplies or services. I could not replicate that power relationship and it would be inappropriate to act as if I had.}

  Thankfully, a perspective I can hope to make improvements from is the perspective that I hold: that of a software engineer and a science, technology \& society scholar. I was aware that assembly processes were involved in the creation of \glspl{sa}, but it was only as I worked on this project that I began to focus on them. These processes are involved, but not particularly meaningful, when software includes only the work of one author\footnotemark. However, as \glspl{sa} grow to include artifacts from many different \glspl{sp}, it seemed like there were real questions of agency involved. 

  \footnotetext{More often software contains the source code of one author and an assembled \gls{sa} from a \gls{sp} that handles the translation from source code to machine language.}

  If a \gls{sp} I am involved with is used in ways I don't agree with, giving developers more tools to assess their involvement would be useful. If I create a tool and it's used to do harm, I should consider my role in enabling that harm. The idea that developers' responsibility stops once they've released their work to the public is wrong. Software developers have worked very hard to give control to developers, both in software and in legal systems, after developers release source code or \glspl{sa} to the public \citep{Gabriella_Coleman2012-lq,Kelty2008-jm}. There are few easy answers to be had, but software developers need tools to help them assess how much their work has contributed to other work. 

  \subsection{The Goal of My Intervention}
  I decided to make a tool that would connect Git repositories and \Gls{pyt} files using the \Gls{pyt} import system. The tool traces a path from file to file, using import statements to connect them. It will try to record if an entire files is imported (as in 'import X') or if smaller portions are imported instead (as in 'from X import a, b'). After locating the relevant files, the program collects all of modifications associated with them recorded in the Git repository. Then it matches the identities attached to each modification to each file. Finally the tool assembles a list authors and a list of lines of code in specific files that are contributing to the function of a particular file.

  My hope in building this tool and not another was to highlight the material contributions of the work of various authors to a particular source file. This doesn't reflect anything close to all of the work that has contributed this particular file existing over all other possible files (there are many other factors: the cultural context of each particular files' creation, the immediate and long term goals of the functionality the file attempts to implement, etc). However, for every element of a \gls{sa}, there is a calculable set of particular components that affect that elements' function. Those components have a concrete set of authors. I hope this is a useful starting point.
  \\
  \begin{flushleft}
  I named my \Gls{pyt} module containing my tool \textbf{materiality}.
  \end{flushleft}

  \subsection{Finding the Pieces}
  First, I should briefly explain the relationship \Gls{pyt} modules and \Gls{pyt} source files. The idea of a \Gls{pyt} module is used to organize functionality within the \Gls{pyt} programming language. A \Gls{pyt} module contains \Gls{pyt} objects (including, possibly, other \Gls{pyt} modules). Each \Gls{pyt} module contains at least one \Gls{pyt} source code file. A \Gls{pyt} module that does not contain other \Gls{pyt} modules contains exactly one \Gls{pyt} source file. A great deal of the work \textbf{materiality} does is connecting modules mentioned in \Gls{pyt} source files to particular files in particular places in computer storage. As a result, I will frequently be talking about 'modules' or 'files' nearly interchangeably: the module being the logical representation inside the \Gls{pyt} \gls{sa}, the file being its material instructions stored on the computer.

  I started by trying to use existing \gls{pyt} libraries to scan \Gls{pyt} code, but found that they weren't well suited to the task. The first library I looked in to, py2deps, was based on a command line tool for making graphical representations of the libraries used by a \Gls{pyt} \gls{sa}. This seemed very similar to my goal, but it turned out that py2deps did not differentiate between using some or all of a python library. It was designed to detect if a library was involved in any way. It did not differentiate between using some and all of a library. I tried some other libraries, but none of them seemed to match my desired feature set: finding import statements and connecting an import statement to one or more lines in the referenced file.

  After being unable to find a \Gls{pyt} library to do what I wanted, I decided to use the abstract syntax tree (AST) functionality in the \Gls{pyt} standard library to build my own. A \Gls{pyt} AST is generated by reading a file of \Gls{pyt} source code. The resulting AST will contain all the \Gls{pyt} statements in the file it read, but expressed as \Gls{pyt} objects. This is possible because programming languages are designed to be mechanistically readable and makes it easier to interact with the statements in that language. The AST functionality was originally written to allow the processing and execution of \Gls{pyt} source code by the \Gls{pyt} \gls{sp}, but it can also be used to help \Gls{pyt} programs process \Gls{pyt} source code for other purposes.

  I used the AST module to parse each \Gls{pyt} file I wanted to investigate and catalog three things within each file: the import statements, anything that ads a symbol to the symbol table (as they may be imported), and the line number of each statement in the file. Once I had code that could extract this information from any properly formatted \Gls{pyt} file, I was ready to begin the real work of collating contributions from various authors.

  \subsection{The Algorithm}
  \singlespace % it looks bad if I don't single space the brief intro sentances
  Briefly, here is the high-level algorithm \textbf{materiality} implements. The first \Gls{pyt} file is provided to the tool and will be referred to by the name \textlangle first\_file\textrangle.
  \begin{enumerate}
    \item Create an empty index of python files named \textlangle imported\_files\textrangle.
    \item Create an empty list named \textlangle read\_files\textrangle to store file locations I have seen.
    \item Create a list named \textlangle files\_to\_scan\textrangle with \textlangle first\_file\textrangle in it. 
    \item While there are any file locations in \textlangle files\_to\_scan\textrangle, do:
    \begin{enumerate}
      \item Remove a file location from \textlangle files\_to\_scan\textrangle.
      \item Build an AST (referred to as \textlangle AST\textrangle) from the \Gls{pyt} file at that location.
      \item Add this file location to \textlangle read\_files\textrangle.
      \begin{enumerate}
        \item Create a new tracking \textit{object} named \textlangle file\_info\textrangle.
        \item In \textlangle file\_info\textrangle, record:
        \begin{itemize}
          \item Any \textit{import statements} found in \textlangle AST\textrangle.
          \item The \textit{symbol}s\sepfootnote{symbol} generated by any \textlangle AST\textrangle nodes, as well as what those symbols would contain.
          \item The lines associated with each \textit{symbol} found in the \textlangle AST\textrangle.
        \end{itemize}
        \item For each \textit{import} in \textlangle file\_info\textrangle try to find a file location that it refers to:
        \begin{enumerate}
          \item Try to locate a file using several techniques, not all of which apply to all \textit{imports}:
          \begin{itemize}
            \item If the \textit{import} is relative, navigate the file structure searching for the file it refers to.
            \item Ask the \Gls{pyt} \gls{sa} to find the file location for the \textit{import}.
            \item Ignore \textit{import}s that fit these descriptors:
            \begin{itemize}
              \item \textit{Import}s that come from the python standard library (I did not know how to account for elements of this code that exist in within the \Gls{pyt} \gls{sa}).
              \item \textit{Import}s which, when asking the \Gls{pyt} \gls{sa} to find them, returned an error.
            \end{itemize}
          \end{itemize}
          \item If a Git repository has a copy of the \Gls{pyt} file I just located, exchange the location I found for one in the Git repository (so its history can later be read).
        \end{enumerate}
        \item Add each file locations to \textlangle files\_to\_scan\textrangle as long as they meet these requirements:
        \begin{itemize}
          \item They were not ignored as described above.
          \item A file location could be found.
          \item The location is not already stored in \textlangle read\_files\textrangle.
        \end{itemize}
        \item If the file location was added to \textlangle files\_to\_scan\textrangle do the following:
        \begin{itemize}
          \item Retrieve the list of changes and associated authors made to this file from the appropriate Git repository and add them to \textlangle file\_info\textrangle.
          \item Add \textlangle file\_info\textrangle to \textlangle imported\_files\textrangle.
        \end{itemize}
      \end{enumerate}      
    \end{enumerate}
    \item Empty the \textlangle read\_files\textrangle list. 
    \item Create am empty tally of authors and change numbers named \textlangle changes\textrangle.
    \item Load the \textlangle file\_info\textrangle associated with \textlangle first\_file\textrangle from \textlangle imported\_files\textrangle.
    \item Repeat the following for each \textlangle file\_info\textrangle selected:
    \begin{enumerate}
      \item Add this \textlangle file\_info\textrangle to the list of \textlangle read\_files\textrangle.
      \item Add the stats associated with this file to \textlangle changes\textrangle.
      \item Examine each \textit{import} in this \textlangle file\_info\textrangle:
      \begin{itemize}
        \item If the \textit{import} is not in \textlangle read\_files\textrangle yet, select it and repeat this process.
      \end{itemize}
    \end{enumerate}
    \item \textlangle changes\textrangle now contains the list of authors and files which have an impact of the contents of \textlangle first\_file\textrangle.
  \end{enumerate}

  This produces a list of as many authors as possible and generally characterizes their particular contributions to \textlangle first\_file\textrangle. As of right now, there are a number of flaws in the results and I would not recommend anyone use \textbf{materiality} without it being further developed\footnotemark. 
  \doublespace

  \footnotetext{Also, \textbf{materiality} is currently in the \gls{EDEN} repository for no particular reason other than I was using \gls{EDEN} as an object of analysis. The module name is "materiality."}

  \subsection{Methodological Choices}
  There is not one obvious way to approach this and my tool reflects my choice to approach things one way over other ways. I want to briefly acknowledge and situate those choices.

  First is that this tool collects changes over time and not simply the authors who are responsible for the last changes to each particular line in a file. This is an attempt to reflect the way that files evolve over time and are touched by many hands. Git will record the last person to change each line, but those changes are almost always based on a lineage of work that stretches back in the history of the file. On the other hand, this approach risks elevating substantial work from the distant past over recent and more impactful work. Concerns about how to properly compare compare different instances of work are pervasive throughout the tool.

  Second is that the tool ignores \Gls{pyt} system libraries, though \Gls{pyt} libraries are used throughout the \gls{EDEN} code. \Gls{pyt} system libraries are libraries whose functionality comes both from \Gls{pyt} source code and functionality within the machine code of the \Gls{pyt} \gls{sa}. I avoided counting them because I was unsure about how to account for that mixed responsibility. Including system code would mean fully grappling with cross-language material connections in a way I'm not prepared to do. In addition this tool continues to ignore all non-\Gls{pyt} code for similar reasons.

  \subsection{Flaws in Current Functionality}
  \textbf{materiality} is a flawed creation. I was learning about the details of how \Gls{pyt} handles import statements, the \Gls{pyt} AST and a number of other elements of the \Gls{pyt} ecosystem. This project was also built in concert with the academic work to bring this software into conversation with science, technology and society literature over a relatively short period of time. The result is imperfect. I will briefly go over known problems.

  \subsubsection{A Brief Overview of Symbols}
  The interaction of \Gls{pyt}'s system for referring to modules and my management of the \Gls{pyt} symbol table has a number of problems. To explain I need to go into some detail about these two systems and how they interact.

  Symbol tables in \Gls{pyt} generally follow patterns in other languages. \Gls{pyt} maintains a list of all \Gls{pyt} libraries, with each \Gls{pyt} library taking on either the name of the file the library is defined in or, if there is more than one file, the directory the library is contained\sepfootnote{moduledetails} in. Each \Gls{pyt} file that the \Gls{pyt} \gls{sa} has seen is either recorded as a library, or (if that file is a component of a larger python library) as a sub-section of a \Gls{pyt} library. This means that, for instance, my library is named \textit{materiality} because that is the name of its directory. The \verb|ast_crawler.py| file is stored under the \textit{materiality.ast\_crawler} label and so on.

  This system is used by \Gls{pyt} to avoid redundant loading and processing of \Gls{pyt} source code. The first time \textit{materiality.ast\_crawler} is encountered the \Gls{pyt} \gls{sa} loads and processes that file. On subsequent encounters \Gls{pyt} refers back to its previous work. 

  A similar system is used to track objects within \Gls{pyt} modules. Various \Gls{pyt} statements (including import) add entries to a table of names and their associated \Gls{pyt} objects. The behavior of this system can become complex, but at a high level it functions like the \Gls{pyt} module system - symbol tables can be navigated using names separated by dots. For example, returning to \textit{materiality.ast\_crawler}, \textit{ast\_crawler} contains the \Gls{pyt} object \textit{ImportReference}, which in turn contains a value named \textit{SYSTEM\_LIBRARY}\footnotemark. To bring this value into your python module, you could write \verb|import materiality.ast_crawler.ImportReference.SYSTEM_LIBRARY|. This import statement causes the \Gls{pyt} \gls{sa} to navigate into a directory, select a particular file, then make two selections from two symbol tables. 

  \footnotetext{This constant is used to mark import statements which refer to \Gls{pyt} system libraries.}

  \subsubsection{Symbol Trouble}
  The current version of \textbf{materiality} frequently has trouble with symbol tables. I will do my best to explain, but if I fully understood what was going wrong I would have simply fixed the problems!

  Some \Gls{pyt} statements that add a symbol are simple to process. Function definition statements and object definition statements each include a single name for the symbol. However, there are instances where symbol creating statements (such as assignment operations using the '=' sign) can have complicated structures. For instance, one might write: \verb|(a, (b, (c, d))) = function_call()|. There's an implicit structure to this statement that I can write out\footnotemark, but I was unable to write \Gls{pyt} code that handled situations like this one. I was both hampered by the structure of the code I had already written and layers of state management I would need to handle. Handling this properly would require reference objects that stand in for the results of future \Gls{pyt} \gls{sa} operations, which I did not have time to write.

  \footnotetext{In this case, if we refer to the value of function\_call() as X, the symbol table would look like this: $a\Leftarrow X[0]$, $b\Leftarrow X[1][1]$, $c\Leftarrow X[1][2][1]$, $d\Leftarrow X[1][2][2]$} 

  Another problem is that, simply, there are instances where my programming errors have made it difficulty to navigate the shared file \& symbol table structure I outlined above. There are many errors in the logs which indicate that my tool is unable to match an import statement against my version of the appropriate symbol table. I suspect many of these problems are caused by the fact that I don't (as \Gls{pyt} expects), add things that are imported to the symbol table of the module they are imported into. For instance, the long import statement I used as an example previously (\verb|import materiality.ast_crawler.ImportReference.SYSTEM_LIBRARY|) would add a \verb|SYSTEM_LIBRARY| symbol to the symbol table of the module. The current version of my tool does not do this and many \Gls{pyt} modules make use of this technique.

  I believe that these errors minimally impact the results. \textbf{materiality}'s current results avoid duplicate references to the same file and I believe that few of my known errors would hamper the basic author count. 

  However, the flaws in my work on symbol management and others which I will go over in a moment, indicate that my methodological approach when writing this program is flawed and the results should be treated as highly suspect. I believe they are still interesting and informative even in their less-definitive state. 

  \subsubsection{Counting Errors}
  The initial "final" version of \textbf{materiality} was double (and more) counting files connected to the target of its analysis. I had written a system that was intended to, for a given symbol, return information only on \textit{the section of the module containing that symbol}. In fact, much of my development time was spent trying build a system that would achieve that granularity. Unfortunately, it does not seem like I succeeded (or I failed to properly collate the results). In any case, modules that were accessed through specific symbols inside the module (as opposed to importing the entire module / file) had the change counts for the entire module

  \subsubsection{Other flaws}
  There are other signs that I have made mistakes. Though the structure generated by this algorithm should be a non-cyclical tree graph if it is constructed properly (import statements must form a tree), trying to navigate the entire structure results in infinite loops. I do not understand why. 

  Were I to do this again, I would maintain a single internal representation that smooths the difference between the computers' file structure and a python objects' symbol table, but I did not understand I needed to do this early enough and the code is a mess as a result. 

  The procedures for associating a particular \Gls{pyt} file with a particular \Gls{pyt} likely could fail less often than they do now. However, without adding records of when and how they fail that are reported at the end of the process, estimating the impact of an improvement in this area is difficult. I also do not evaluate if import statements are used. Its possible for a file to import a \Gls{pyt} module that it does not use (I have unused imports in the \textbf{materiality} source code). These can be detected, but I do not do it.

  Finally, there were cases where overlapping errors that, individually, would have broken the \gls{sa} effectively canceled each other out. In one place I designed the system for processing \textit{import} statements to mark an import as invalid if the \Gls{pyt} \gls{sa} could not locate a file for it. However, in another place, I ignored that marker and assigned it a file based on the module name the \textit{import} referred to. Each system ignoring the other was an unintentional omission, but in combination they produced a functional system (albeit one that frequently falsely complains it can't find a module).

  \section{Material Connections of Validators}
  In some sense, the tool speaks for itself. It is, as are all \glspl{sa}, materialized intent. There is a level of understanding of \glspl{sp} and artifacts that can only be acquired through interacting with them. However, it's no fun to make something and not show it off, so here are my results for one particular file in the \gls{EDEN} component source code.

  \subsection{Validators by the Numbers}
  Below are the results of running \textbf{materiality} on the \Gls{pyt} file\footnotemark containing the source code for the Validators that I repeatedly encountered. This is the file that contains the \gls{EDEN} part of the Validator functionality that is partially defined in \gls{EDEN} project source code and partially defined in \gls{w2p} source code.

  \footnotetext{The path inside \gls{EDEN}'s repository is /modules/s3/s3validators.py}

  \begin{table}[ht]
  \caption{Result Summary}
  \label{tab:result-summary}
  \begin{tabular}{|l|l|}
  \hline
  Authors Involved               & 119                                   \\ \hline
  Files Involved                 & 86                                    \\ \hline
  Lines of Code                  & 41,226                                \\ \hline
  Total Change Count to Files    & 170,806                               \\ \hline
  Longest File Active Lifetime   & 8 years                               \\ \hline
  Average File Active Lifetime   & Between 4 and 5 years                 \\ \hline
  \end{tabular}
\end{table}
  
  \begin{itemize}
    \item The author count includes myself, because I (defiantly) merged my fix to the bug I found in the phone number validator (after changing it to work correctly).
    \item The count of files show here involved includes the \verb|s3validators.py| file \textbf{materiality} was given as input. 
    \item File active lifetime should be understood as the period of active work on that file. For example, if a file was created 6 years ago and last changed 4 years ago it would have an active lifetime of 2 years.
    \item These data are summaries of a larger dataset. An author-by-author breakdown can be seen in the [\textbf{Table}] section at the end of this paper and even more detail is accessible through a Google sheet whose link can be found in the same section.
  \end{itemize}
  
  These results highlight the collaborative nature of software development, especially \acrshort{foss} software development. There are many authors and files that support the various Validator classes \gls{EDEN} has defined. In particular, it's interesting that these files contain just over forty thousand lines of code, but over four times that number of lines have been added or removed from those files. This shows how code changes and evolves over time and highlights how software is not any particular version of the source code or \gls{sa}, but a process and practice that occurs over time \citep{Mackenzie2006-hb}.  

  Diving a little deeper into the data, it's possible to characterize the weight of the impact from the different projects on the set of validators defined in \verb|s3validators.py|:

  \begin{table}[ht]
  \caption{Percent of Involved Changes By Project}
  \label{tab:change-locations}
  \begin{tabular}{|l|c|c|}
\hline
\textbf{Project} & \textbf{Location of Connected Changes} & \textbf{Connected Changes As \% of Project} \\ \hline
Eden                  & 16.1\%                                       & 0.44\%                           \\ \hline
web2py                & 62.2\%                                       & 3.30\%                           \\ \hline
pydal                 & 20.7\%                                       & 9.23\%                           \\ \hline
yatl                  & 1.1\%                                        & 77.75\%                          \\ \hline
\end{tabular}
\end{table}

  I think the most interesting thing here is how the work on the shared Validator class from web2py is reflected in this breakdown. The \gls{EDEN} team is able to do less work by relying on the work put into the object(s) they are extending (and the other objects inside web2py that support that foundational work). It would also clearly help if I had some method that accounts for project size - \gls{EDEN} is over four hundred and sixty times larger (\ref{tab:table-2}) than yatl, meaning that 0.44\% of \gls{EDEN} is likely\footnotemark more work than 77.75\% of yatl.

  \footnotetext{The reason I say likely is that I'm not sure how to compare the total number of changes in the entire repository (which include the non-\Gls{pyt} code) to the language-specific line counts in \ref{tab:table-2}.}

  \subsubsection{What can't be Seen}
  I'll reflect more on this in my discussion on possible improvements, but I should note here the overwhelming flatness of my results. There is a myopic quality to looking at programming in this way. The act of displaying all the contributors together, without differentiation, is itself a \citet{Haraway1988-nh}-esq god trick. It took a good amount of work to get to this point - more than I'd like to admit and more than a skilled programmer would need. I believe there is worth in these results as a reflection of the exercise and the exploration, but they are a snapshot from a journey that may never be completed - not a triumphant demonstration of the ideas I've been exploring.
  \newpage
% ==================================================
% =  ____  _                        _              =
% = |  _ \(_)___  ___ _   _ ___ ___(_) ___  _ __   =
% = | | | | / __|/ __| | | / __/ __| |/ _ \| '_ \  =
% = | |_| | \__ \ (__| |_| \__ \__ \ | (_) | | | | =
% = |____/|_|___/\___|\__,_|___/___/_|\___/|_| |_| =
% ==================================================

   \section{Reflecting on What I've Seen}
   Now I want to turn to drawing out the themes in the things I've seen and the work that I've done. In some ways 

% =================================================================================================
% =     _                           _     _         ____  _                        _              =
% =    / \   ___ ___  ___ _ __ ___ | |__ | |_   _  |  _ \(_)___  ___ _   _ ___ ___(_) ___  _ __   =
% =   / _ \ / __/ __|/ _ \ '_ ` _ \| '_ \| | | | | | | | | / __|/ __| | | / __/ __| |/ _ \| '_ \  =
% =  / ___ \\__ \__ \  __/ | | | | | |_) | | |_| | | |_| | \__ \ (__| |_| \__ \__ \ | (_) | | | | =
% = /_/   \_\___/___/\___|_| |_| |_|_.__/|_|\__, | |____/|_|___/\___|\__,_|___/___/_|\___/|_| |_| =
% =                                         |___/                                                 =
% =================================================================================================
  \subsection{Who wants a Relationship to Fail?}
  The initial failure I encountered where I discovered a break in the relationship between \gls{w2p} and \gls{EDEN} has got me thinking about \glspl{sa} and agency. On one hand, small mistakes are universal and this was certainly a small mistake. When the \gls{w2p} project fixed this problem, it took only two lines of code \citep{Konig2019-fw}. But, because this is a failure to agree on how to transfer control from code written by the \gls{w2p} project to code written by the \gls{EDEN} project - any of the groups involved could have repaired it. \gls{EDEN} Could have chosen to change their source code, I could have added a new component to my \gls{EDEN} \gls{sa} that changed the relationship between \gls{EDEN} Validators and \gls{w2p} Validators. 

  It seems like expressing agency in \glspl{sa} is a narrow thing. There are many elements in each component and proper \gls{sa} behavior doesn't move through just one layer, but often through tens or hundreds of layers. Eighty six files were involved in supporting the behavior of the Validators objects. Each relationship that exists in those files is a relationship that exists on two levels: a material level as defined by the instructions in the code and a social level that exists in the humans writing that code. In order for people to imagine how to form new relationships in the material of software, they must first imagine what that relationship will be like \citep{Mackenzie2006-hb}. This work of projection and imagination exists at every level of the program. All of the potential fixes (in \gls{EDEN}, in \gls{w2p}, etc) involve envisioning the relationship in a particular way before engaging in the material work of reforming it.

  So what this problem shows is an escape of the material qualities of a \gls{sa} from the social imagining of how it should be. The escape, in this case, materially changed the structure the assembled \gls{sa} compared to the socially envisioned \gls{sa}. Throughout this project I've seen the process go from the material to the social and back to the material. The \gls{EDEN} project rejected my bug fix because it had a material failing - in certain circumstances it returned the wrong error. But that material failing was not demonstrated in a material fashion (assembly of a \gls{sa}), it was agreed upon socially. Any time we consider a potential problem, we cannot help but consider it in a social context because we are social creatures, but while a \gls{sa} is created within a social context, it is not a \gls{sa}. It is mechanical.

  Each line of source code has a material effect and a socially understood intent. Human thinking is fuzzy, but machine action is not. This can manifest in simple ways. A \gls{sa} defying developer intent by not telling the user what is happening. It can also manifest in complex ways. Systems can express bias that developers believe is not there. Systems can be contain instructions that will lead to decisions that developers believe are not possible. Once they are assembled and sent out in the world, \glspl{sa} will do whatever their code tells them to do, whatever their creators believe.

  \subsection{Software Artifacts as Blackboxes}
  Blackboxes were used by Latour to explain how tools become involved in our accomplishing of goals. It's too simple to say that we do something or the tool does something - the tool changes us and we change the tool \citep{Latour1999-ui}. On one hand this fits with how people interact with \glspl{sa}. Throughout this process I've used \glspl{sa} to assemble software, write this paper, distribute my work, etc. On the other hand, blackboxes are also a model to explain how complex things make up other complex things, and that does not fit what I have seen. \Gls{sa} is hard to see as a box containing other separate boxes. 

  The \gls{EDEN} source code does not just use the \gls{w2p}, it transforms it. When the \Gls{pyt} \gls{sa} executes the \gls{EDEN} (and \gls{w2p} and all other components) source code, what is placed in computer storage is a mishmash of all of the components. Once they are incorporated into a \gls{sa}, components mix with each other in a way that is particular to that set of components and that method of assembly. \Gls{sa} are blackboxes, but their internal structure is far more complicated than blackbox theory can capture. That complexity is key to understanding how agency is impacted and convoluted by \glspl{sa}

% ==============================================================================
% =  _____                       ____  _                        _              =
% = | ____|_ __ _ __ ___  _ __  |  _ \(_)___  ___ _   _ ___ ___(_) ___  _ __   =
% = |  _| | '__| '__/ _ \| '__| | | | | / __|/ __| | | / __/ __| |/ _ \| '_ \  =
% = | |___| |  | | | (_) | |    | |_| | \__ \ (__| |_| \__ \__ \ | (_) | | | | =
% = |_____|_|  |_|  \___/|_|    |____/|_|___/\___|\__,_|___/___/_|\___/|_| |_| =
% ==============================================================================

   % discussion
   \subsection{Implications of my Failed Bug Fix}
   Before going into the interesting aspects of the König quote, I want to talk about something I find interesting about the context around the discussion itself. This was a discussion between people who both had access to one (or more) \gls{EDEN} \glspl{sa}. I originally noticed the bug when I was interacting with my \gls{sa}. This how most software projects impact the world: through directly or indirect contact with \glspl{sa} the project produces or supports \citep{Kitchin2011-af}. When we say that software impacts our lives in direct and indirect ways, that impact flows from particular people interacting with particular \glspl{sa}. Though software has made it possible to remove humans at many of the intersections of power, it is still experienced through a multitude of relational contacts between a representative of power (a person or a \gls{sa}) and a particular human \citet{Cheney-Lippold2018-lw,Deluze1995-eo}. 

   \subsubsection{Do Software Artifacts Matter}
   Given all of that, I think it's interesting that the entire discussion about this bug took place without referencing the behavior of any particular \gls{sa}. There are other situations where the \gls{EDEN} project ask about component selection to try and understand the source of problematic behavior in a \gls{sa} \citep{Trendspotter2020-ps}. The idea that \gls{sa} may behave in ways that are deceptive to the user was also an aspect of the discussion around the bug. Why isn't it useful to talk about any \gls{sa} in particular?

   When a \gls{sa} is under discussion, that discussion often starts by trying to find a mistake or a mismatched relationship in the assembly process. My bug fix, in contrast, was to a particular pre-assembly component. In order to realize my change, a new \gls{sa} would need to be assembled. Nothing in my attempt to fix the bug or in the eventual official bug fix can repair existing \glspl{sa} \citep{Konig2020-yx,Konig2020-ey}. This explains the lack of interest in any particular \gls{sa} - nothing about any \gls{sa} that might or might not express this bug is of interest. Instead, when the bug is fixed in a component of \gls{EDEN}, any potentially problematic \gls{sa} will be replaced with a new one assembled from updated components.

   % https://stackoverflow.com/questions/1945075/how-do-i-create-binary-patches
   To underline this point again: \glspl{sa} are never updated. The practice of patching, or distributing updates to \glspl{sa}, has the goal of changing an existing \gls{sa} to be identical a newly assembled \gls{sa}. A patch program is a different \gls{sa} that will effectuate a different assembly process given its target \gls{sa} and itself. Patches are generated by comparing the old \gls{sa} to the new one and storing a list of changes to make to transform the old to the new. It has the goal of producing a \gls{sa} that is the same as the one generated by a full re-assembly with all of that \gls{sa}'s components \citep{Endsley_undated-iy}. Whether a \gls{sp} replaced entire artifacts when they find problems or if they create special purpose \glspl{sa} to transform existing \glspl{sa}, the desired \gls{sa} is always the \gls{sa} that is generated by a full assembly of the latest components.

   The bug was fixed by changing a component with the intent of the changed component being assembled into a new \gls{sa}. It would be possible to, for each update, write source code with the intent of creating a \gls{sa} to repair that particular problem. This would be similar to how physical objects are repaired - building a new piece of software to carry out each reparative step. But such a practice would be too labor intensive. The closest thing is the practice of binary patching I just described, which depends on fully generating a fresh \gls{sa}. So \glspl{sa} are always replaced, because digital technology makes it easy to do so and the practical consequences are the same either way: a particular number of bytes in a particular order on digital storage. 

   This means that scholars must be careful when applying concepts whose identities were formed based on our collective experiences with physical goods. It's true that software is maintained, updated, repaired, cleaned, polished, and so on. Those operations often serve the same purpose as similar operations on physical tools but the material character of how they are carried out will be dramatically different. This changes the worlds of the people doing the work and the particulars of how these operations effect software, and the proper weighting of concerns in assessing the impact of these operations. Because the material that composes the physical character of \glspl{sa} is cheap and plentiful (electrical charge), many of the previous systems that evolved in response to more complicated systems of material construction have limited applications to software projects and \glspl{sa}.

   \subsection{The Limits of Developer Attention}
   König expressed the idea that testing in too much detail can harm software development. I think this idea, in combination with some discussion of assembly in the following section, will be helpful for thinking about agency in software. An uncharitable understanding of the idea that too much testing harms development is that developers might not care about some dysfunction, but I see no evidence of this. The \gls{EDEN} project immediately implemented a fix for the problem I found. 

   Instead, I think this approach reflects a \citet{Citton2017-xq}-esq approach to the problem posed by the enormous complexity of \glspl{sa}. In just the four libraries I focused on for this project there are over five hundred and fifty thousand lines of code (See Table \ref{tab:table-2}). The version of \gls{EDEN} I have been working on has seventy two authors (excluding myself). Even if each developer was contributing equally (which they are not), this would require each developer to test every single outcome of around seven thousand lines of code. That is more code than many contributors have added to the project. Without drawing conclusions about the degree to which developers have a responsibility to monitor behavior of their software artifacts, it seems very likely that it would be impossible for them to monitor all possible behaviors.

   The implication of this is not to let developers off the hook because monitoring all behavior is impossible. Instead I want to this about the social context that a \gls{sp} exists within. It seems like the \gls{EDEN} project is focused on how their components interact with data rather than how they interact with people. That is to say, they would rather create a \gls{sa} that gives the user incorrect information about its actions, but records the information that the developer expects. This runs the risk of becoming disconnected from user experience of \gls{EDEN}\footnotemark.

   \footnotetext{I have no reason to believe this has occurred.}

   \subsubsection{What About when Developers Aren't Looking?}
   This focuses on aspects of \gls{sa} behavior that are inaccessible to users. It's true, to a certain degree, that users can be confused by particular elements of how a \gls{sa} is functioning and still have an overall impression of what 'it is doing.' However, the general impression of what \gls{EDEN} (or any \gls{sp}) 'does' isn't transmitted directly from the understanding of the \gls{sp}. Instead, those impressions are formed from users' many experiences with particular \gls{sa} in particular situations \citep{Boellstorff2015-al,Eubanks2018-hc,Schull2012-nc}. Often, users do not even have to interact with a particular \gls{sa} to form an impression of its impact \citep{Kitchin2011-af}. The experiential nature of how impressions of software are formed means that users are unlikely to change their impressions of the software because one or more message is wrong, but it seems certain that at some point users will feel misinformed to such a degree that their impression changes.

   There's also the risk that material qualities of the software and the messages it sends to its users begin to diverse in unsustainable ways. Developers can choose to have their \glspl{sa} send messages to users that don't reflect the internal state of the software, but this will likely make continuing to operate the hidden aspects of that \gls{sa} (and project) difficult \citep{Bivens2017-tc}. Producing a \gls{sa} that has external representations that aren't supported by all of its internal state means severing internal relationships and creating parallel internal functionality. 

   Developers choosing to watch certain parts of their \glspl{sa} and choosing not to watch others creates a separation between the social and material worlds around a particular \glspl{sa}. The material behavior that developers choose to surveil is socially visible. The source code or testing practices mean that the the developers have a relationship to the idea that their \gls{sa} should behave a particular way \citep{Mackenzie2006-hb}. However, based on these experiences, it seems clear that developers do not pay equal attention to every material element of their \glspl{sa}. There are aspects of \gls{sa} behavior that they can't afford to pay attention to or that they believe are not important enough to pay attention to. The choice to ignore certain behavior based on the idea that it's less important means that we should be asking "less important to whom?" \citep{Harding1992-od,Haraway1988-nh}. 

   It's possible (even likely) that members of a \gls{sp} are the individuals who understand that \gls{sp}, but the impacts of a given \gls{sp} can be felt far outside the social world of that particular project \citep{Kitchin2011-af}. \Gls{sp} run the risk of being blind to serious issues when they accept an understanding of what is inside and outside the bounds of reasonable concern, because what is concerning to us is always driven by our individual and group identities.

   Beyond the implications for those outside a \gls{sp}, the nature of software means this has important material implications for \glspl{sp}. Each component of a \gls{sa} will have an understanding of that components' material impact that is socially constructed by the components' \gls{sp}. There will be material effects of that component that are invisible to social context within its project. The project can't warn people about the material qualities of their own work that they do not see. This means that each component in a \gls{sa} will behave in ways that the components' makers are not aware of and cannot speak about. Because each \gls{sa} component may itself be a \gls{sa} (and so on and so forth) there can be hundreds of thousands of layers of socially invisible material action.

   \subsubsection{Watching the Machines}
   I think that the work that \glspl{sp} do to observe the \glspl{sa} they create and ensure they behave as expected is almost certainly best understood as surveillance. Surveillance studies is a large field and one in which I have little experience. Applying theories in this field to understanding how \glspl{sp} seek to prevent their creations from escaping their authority would be interesting.

   \subsection{What does Working Mean Here?}
   I want to return to talking about the various experiences of functionality and non-functionality I've encountered while trying to fix the bug\sepfootnote{working}. First I'm going to try and explore the ideas of assembly and then how to think about the functionality of assembled artifacts.

   Assembly is a process whose limits are challenging to clearly mark. There are instances where assembly is clearly successful, such as when a \gls{sa} works correctly in all ways or in instances where the failure of an assembled \gls{sa} clearly comes directly from a single flaw in a single component (such as the international number bug). There are also limits where assembly is clearly unsuccessful: when a component cannot be found, when a language that must be translated into machine code (or any language) is found to contain a syntax error \citep{Cramer2008-cw}. However, when I created a software artifact that failed to work because of a broken relationship between its components, I began to wonder if my definition of assembly was flawed.

   Unlike compiled languages, which fully parse their constituent source files before reaching a state where an executable \gls{sa} exists, Python \gls{sa} performs assembly immediately before executing a python program\sepfootnote{runtimeimports}. In fact, the \gls{pyt} assembly is piecemeal - each file is only read when needed, so files that are part of the \gls{sa} could go unused for any period of time before being used in the \gls{sa}. It would be possible to, for instance, begin running an \gls{EDEN} \gls{sa} and replace the line of web2py code that, if it were executed, would cause a crash. If the code was replaced before the \gls{EDEN} \gls{sa} called the code with the bug, no crash would occur. So do I need one approach for assembly with interpreted languages and another approach with compiled languages?

   No, because the apparent differences are illusions. Compiled programs can also have their contents changed mid-executions\footnotemark as well. Changing the contents of a program that has already been translated to machine code is simply much less convenient than changing the text in the source file of an interpreted language. The moment of failure that I encountered when web2py tried to invoke \gls{EDEN} code it no longer knew how to find was a moment where the computer tried to follow a chain of relationships. It would not be possible to know if any given instance of an \gls{EDEN} validator that inherits from web2py code would work correctly without running the program. Knowing the results of a program before it executes is something that computer science says is impossible, at least with structure used by our current computers \citep{Kaplan_undated-xy}. 

   \footnotetext{Very old programs will do this intentionally: re-writing their code in memory as they execute. In the modern context, it generally happens when a program is being broken into by a hacker. The hacker will exploit insufficiently paranoid programming practices to get the \gls{sa} to replace its current machine code with machine code supplied by the hacker.}

   For the purposes of this this project, what I call assembly is every aspect of preparing a program to execute that can be done without needing to actually execute the program (i.e. the limits of pre-execution analysis). These processes (compilation, the locating of python files) are often themselves executed by \glspl{sa}. As long as the work being done could be done without executing source code, it is considered part of assembly\sepfootnote{pythonexeq}. 

   Why is this important? Does it matter for sociologists when assembly stops and execution begins? The end of assembly is the point where human intervention into a \gls{sa} stops being possible. After assembly is complete, the balance of which components will impact the overall behavior of the \gls{sa} is fixed. The only human intervention that is possible is the decision to start or stop the \gls{sa}\sepfootnote{askingforhumaninput}. This is doubly important because our ability to predict what a \gls{sa} will do when it is executed is hampered by both practical and theoretical limits \citep{Kaplan_undated-xy}. 

   This reality is reflected the practice of testing in the development of \glspl{sa}. The \gls{sa} assembled during \gls{EDEN}'s testing process is not exactly like the final \gls{sa} (it largely differs in the database entries that compose it). For instance, the assembly error I encountered with web2py was easily detected by tests. This adds to my sense that testing is a reaction to the fixing of agency in a \gls{sa} and an attempt to ensure that the agency reflected in the final \gls{sa} is sufficiently similar to the developers' intent. Tests can take the form of source code that is incorporated into one or more testing \glspl{sa} or in structured human interactions with the final \gls{sa} before releasing it to the public. In either case the intent is likely to surveil and control the behavior of the \gls{sa}. 

% =================================================================================
% =  ___       _                            _   _               ____  _           =
% = |_ _|_ __ | |_ ___ _ ____   _____ _ __ | |_(_) ___  _ __   |  _ \(_)___  ___  =
% =  | || '_ \| __/ _ \ '__\ \ / / _ \ '_ \| __| |/ _ \| '_ \  | | | | / __|/ __| =
% =  | || | | | ||  __/ |   \ V /  __/ | | | |_| | (_) | | | | | |_| | \__ \ (__  =
% = |___|_| |_|\__\___|_|    \_/ \___|_| |_|\__|_|\___/|_| |_| |____/|_|___/\___| =
% =================================================================================  
  \section{Reflecting on my Intervention}
  Now I'm going to briefly reflect on what I found by writing software of my own. I have some ideas about where people might take this in the future and how some of the shortcomings might be addressed.
  
  \subsection{Foreseeable Improvements}
  I will briefly catalog things I think could be improved about this work. This will both sooth my frustrations and mark what directions the \textbf{materiality} \gls{sp}, as it exists today, could be taken in.

  \subsubsection{Deeper Engagement in Git}
  The current version of \textbf{materiality} engages with modification history recorded in git in a very shallow manner. It counts changes to files without reference to content. It does not detect when a file has been moved. It can't tell the difference between a truly new file and a new file that is created by removing some of the content of an existing file. Any time source code is copied or moved its history is lost. In short, the statistics that the current \gls{sp} produce suffer from a number of foreseeable problems that bias the results in uncertain ways. Not all of these scenarios are clearly detectable - text matching can be difficult and uncertain - but an effort could be made.

  \subsubsection{Change Distance}
  The greatest weakness in the current form of \textbf{materiality} is the flatness with which it presents its results. The most obvious improvement would be for \textbf{materiality} to characterize the distance of the various impacts in some way. At the moment, if any file can be reached through the tree of imports leading out from the one \textbf{materiality} starts from, its authors and change counts are added to the list without adjusting for the distance. This means tallying work directly on the file of interest and work on, say, a file used by a related library to print error messages would be displayed in an undifferentiated way. The current \gls{sp} tosses all change counts and authors into a single list, but this defies the nature of software. Much of this paper and this project has focused on the power of assembly and how, in this case, placing the \gls{EDEN} component in the most important spot within the \gls{EDEN} \gls{sa} is why the \gls{EDEN} component is the most influential. \textbf{materiality} does not reflect that and it should.

  Bringing a sense of distance, or varying amounts of weight, would also be an opportunity to apply the sociological theories that form the theoretical background for my thinking in this project. Both linking together various \Gls{pyt} files and attempting to weigh the relative impact of those links recalls \citet{Mackenzie2006-hb}'s work on how relationality structures \glspl{sp}. It reminds me of \citet{Haraway2016-nc}'s notion of "tentacularity" - the relentlessly interconnected nature of the world and how that wealth of connections to enliven our sense of the possible. It's also clear that the way \textbf{materiality} works now - making no effort to characterize how any particular change is related to the file being analyzed, is the kind of 'god trick' that \citet{Harding1992-od} and \citet{Haraway1988-nh} set out to critique. I have created another view from nowhere.

  I do not know now what the best way to characterize the disparate contributions from changes at different points in the tree of related files. Usage analysis could be a large improvement. Import statements add objects to to the module symbol table and \textbf{materiality} could analyze the usage of those objects. Carrying this practice across the entire import tree would highlight links of action (as opposed to an unused link). However, such an analysis might be deceptive. There are real limits to how much we can say about how a \gls{sa} will behave without executing it \citet{Kaplan_undated-xy}. As much as the current program is a view from nowhere, I want to avoid rendering a different kind of false view.

  \subsubsection{Ways of Seeing}
  The current output of the tool is, to put it bluntly, opaque. Quoted below is a sample output from the \textit{adapters} module in the \textit{pydal} module. This is as close as the \gls{sa} can come to displaying the web of connections. 

  \singlespace
  % https://www.overleaf.com/learn/latex/code_listing
  % https://tex.stackexchange.com/questions/171803/change-font-size-of-the-verbatim-environment
  \begin{Verbatim}[fontsize=\small]
    </Users/ddrexler/src/python/web2py/gluon/packages/dal/pydal/adapters/ __init__.py>
    SM[pydal.adapters]-> Module[]
      Imports:
        [1]Import <!S>re
        [2]From <+R>pydal._gae Import gae
        [3]From <+R>pydal.helpers._internals Import Dispatcher
        [71]From <+R>pydal.adapters.base Import SQLAdapter,NoSQLAdapter
        [72]From <+R>pydal.adapters.sqlite Import SQLite
        [73]From <+R>pydal.adapters.postgres Import Postgre,PostgrePsyco,PostgrePG8000
        [74]From <+R>pydal.adapters.mysql Import MySQL
        [75]From <+R>pydal.adapters.mssql Import MSSQL
        [76]From <+R>pydal.adapters.mongo Import Mongo
        [77]From <+R>pydal.adapters.db2 Import DB2
        [78]From <+R>pydal.adapters.firebird Import FireBird
        [79]From <+R>pydal.adapters.informix Import Informix
        [80]From <+R>pydal.adapters.ingres Import Ingres
        [81]From <+R>pydal.adapters.oracle Import Oracle
        [82]From <+R>pydal.adapters.sap Import SAPDB
        [83]From <+R>pydal.adapters.teradata Import Teradata
        [84]From <+R>pydal.adapters.couchdb Import CouchDB
  \end{Verbatim}
  \doublespace

  This form was designed to help me understand what the program was and was not finding as it executed. It helped my surveil my own \gls{sa}. The numbers in \verb|[]| are line numbers in the \Gls{pyt} file. The symbol and letter inside the \verb|<>| symbols communicates the status of the \textit{import}. \verb|<!S>| \textit{import}s are system libraries and will be ignored. \textit{Import} statements with a \verb|<+R>| before them are relative imports whose referent has been successfully found in my computers' file system\footnotemark. 

  \footnotetext{For example, the import starting with [2] was originally written "from ..\_gae import gae", the program has checked that this file can be found and changed from a relative path to an absolute path within the \textit{pydal} module.}

  Many different graphical depictions would be possible. The file itself could be laid out and sections of text connected to the tree of python files that influence that texts' behavior. \textbf{materiality} could generate graphical representations of the networks of \Gls{pyt} modules involved. Instead of focusing on the files themselves, the collaborations between authors could be highlighted. Each file could seen as moving between contributors over time and \textbf{materiality} could render a graph of how the feature being analyzed had been built by humans in specific ways and at specific times in the past.
  \newpage

  \section{Returning at last to Software}
  I opened this paper and this project by asking how assembly mediates agency in software. The literature is clear, and I agree, that there is a strong connection between \gls{sp} intent and \gls{sa} behavior. The \gls{EDEN} project describes their mission as "sav[ing] lives by providing information management solutions that enable organizations and communities to better prepare for and respond to disasters" and everything I found in the \gls{EDEN} source code or the \gls{EDEN} \gls{sa} suggests that the impact of \gls{EDEN} is in line with their goal \citep{Sahana_Foundation_undated-hl}. The complexities of software development do not change the previously established connection between tool impact and creator intent \citet{Subramaniam2014-wg}. 

  I also found, at every turn, material qualities of software that threaten to escape the ability of creators to detect and tools to prevent. One consequence of this is that software developers are expected to trigger and manage the assembly processes for any \gls{sa} they wish to work on. The experience developers derive from this is key to learning about the components used by the \gls{sp} in the \gls{sa}, but it redoubles the limited view already embodied through \gls{sp} surveillance of \gls{sa} ([\textbf{The Limits of Developer Attention}]). Any developer who engages with the components (source code, \gls{sa} libraries) and the \glspl{sa} of a \gls{sp} is constantly experiencing suggestions about the nature of the world. Within König's idea that "testing too much detail can add excessive rigor," I see a world view that the correct way to imagine and survival software is attending to a particular set of behavior over all others \citep{Konig2020-yx}.

  The \gls{EDEN} project and the \gls{EDEN} \gls{sa} is not equipped to detect deviations in behavior they are not watching. Once a \gls{sa} is sent out into the world, its possible behavior is set. Problems discovered within it will need to be worked around or the problematic aspect ignored until it is replaced. \citet{Latour1999-ui}'s concept of blackboxes focused on their ability to allow complex objects to be socially invisible, but that's not what I see with \glspl{sa}. In both [\textbf{What does Working Mean Here?}] and [\textbf{What About when Developers Aren't Looking?}] I encountered an object whose behavior defied shared social understandings. It was not that my \gls{EDEN} \gls{sa} 'broke,' but that it broke in a particular and limited way. In the case of the phone number message, the failure fell outside the sight of the \gls{sp}. The other failure was cause by my choices in assembly. Both failures could only be understood by looking inside the assembly process but, critically, not inside the particular \glspl{sa}. Unlike a blackbox, whose contents become socially visible\footnotemark when they fail, \glspl{sa} keep their secrets. I think blackboxes aren't helping me understand the social qualities particular to software. 

  \footnotetext{To elaborate on this theme, Latourian blackboxes do not actually prevent people from seeing what is inside them. He is instead trying to speak to why people \textit{can} operate \textit{as if} the internals of a blackbox are invisible. The fact that blackboxes contain other blackboxes goes to show that everything within a blackbox has a social identity that people can and will recognize and interact with if they are shown it. \glspl{sa} on the other hand, have their internals transformed from an object that some can understand (source code) to something nearly totally opaque to everyone (machine code).}

  \subsection{Composites}
  Software artifacts are not blackboxes, they are \textbf{\glspl{cmp}}. \Glspl{cmp} are fixed assemblies of more than one component which inherits the material qualities of its components, mediated through the structure into which those components are assembled. Once assembled, a composite can be mechanistically 'read' by a machine to tell that machine which action to take. This process does not require human intervention. Composites are typified by extremely high internal complexity and complex behavior. Once assembled their possible actions are set. Those actions cannot be predicted by examining a \gls{cmp}. \Gls{cmp} cannot be changed, only replaced. Each component of a \gls{cmp} may be an instruction or a datum, but they are generally composites themselves. Each component makes a contribution proportional to its size to the material qualities of the \gls{cmp} it is integrated into. However, that contribution is mediated by layers of assembly and can place components in relationships that the component creators did not foresee and which result in the component behaving in new ways.

  The identity of a \gls{cmp} derives from its behavior, not its label. The different \glspl{sa} of \gls{EDEN} are not similar because they share the name "eden," they are similar because they are constructed from the same components and because the \gls{EDEN} source code is in the position of primary importance. As \gls{EDEN} evolves and changes over time, its social identity will "stretch" to reflect a diversity of behavior, but that same stretching betrays that multiple objects with different qualities are sharing a single label. 

  \Glspl{cmp} are immensely useful objects. They can encode complex methods for weighing alternatives that could allow nearly immediate decisions in time-sensitive situations. If the required material resource are made available to the \gls{cmp}, it will continue until it exhausts its resources or is stopped. The humans that start them running do not need any knowledge of their internals. This also makes \glspl{cmp} very dangerous. They are powerful conduits for systems of control and surveillance. 

  Due to their complexity, \glspl{cmp}' behavior is unpredictable. How changes made to a component will appear in the behavior the assembled \gls{cmp} is difficult, sometimes impossible\sepfootnote{softwaretoo}, to predict. Multiple components or layers of assembly can interact in unforeseeable ways\footnotemark. \Gls{cmp} makers respond to this quality by implementing systems of surveillance to observe their own creations, but this system has obvious flaws. In order to imagine what behavior is worth testing, \gls{cmp} makers must imagine what their composites might do. Systems of surveillance must be told what to surveil and \gls{cmp} makers cannot forsee all ways their \gls{cmp} might be used. In addition, these systems of surveillance often involve new \glspl{cmp} purpose-build to observe expected behavior of the final \gls{cmp}. Like all \glspl{cmp}, these surveillance \glspl{cmp} can be expected to behave in unpredictable ways from time to time. 

  \footnotetext{See \textbf{Other flaws}.}

  \subsection{Composites in Society}
  Scholars have noted specific situations where \glspl{cmp} have, intentionally or unintentionally, acted in problematic and damaging ways \citep{Cheney-Lippold2018-lw,Eubanks2018-hc,Schull2012-nc}. Previously, the focus has been on the various projects (software and otherwise) that are creating and placing \glspl{cmp} in positions of authority. This is important work, but it elides the social qualities that \glspl{cmp} do not share with previous systems of control: their opacity. The qualities of a \gls{cmp} can only be discovered by interacting with and observing it. 

  Standpoint Theory was motivated by the harm caused when we separate an idea from its creator. It highlighted that whether or not ideas are true \textit{in essence}, they first must \textit{seem} true to the people who 'discover' them \citep{Harding1992-od,Haraway1988-nh}. The same is true of \glspl{cmp}: the behavior that creators understand as desirable must agree with their world view. How the makers of a \gls{cmp} choose to decide to release a composite into the world is a decision mediated and constructed by the culture of the people making that \gls{cmp}. Even though specific tests of that \gls{cmp} might be defined in terms of the material inputs and outputs, there will always be a layer of social consensus around how to understand the results of those tests. Situating a \gls{cmp} within the culture of the people who created it will make its influences socially visible, but the \gls{cmp} will still \textit{behave} the same way. Its material qualities, when executed by a machine, will be the same, even if the social layer around the \gls{cmp} make them inaccessible to us.

  So any social engagement with \glspl{cmp} must reckon with two truths: our experience of them is mediated by our social experiences and \glspl{cmp} are machines. A \gls{cmp} may be re-contextualized, re-interpreted, re-situated or (most commonly) replaced. It cannot be changed. Each \gls{cmp} is created in a particular social context and that people from that social context will reliably approve of a portion of the behavior of that \gls{cmp} \citep{Mackenzie2006-hb}. But that \gls{cmp} is not trapped within that social context. If the outputs the \gls{cmp} produces for a given input seems useful to other groups, it may be adopted as a component. But this adoption cannot not take just one part of the \gls{cmp}. It must take the entire thing. Including any behavior its original creators missed, or behavior its new adopters have misunderstood. \Glspl{cmp} are mechanical things masquerading as cultural.

  This means we have a tricky path to navigate with \glspl{cmp}. As I noted in \textbf{Investigating Assembly}, small  changes can have large impacts on \gls{cmp} behavior. The behavior of a \gls{cmp} cannot be reduced to the behavior of a particular component. Relationships between components do not have to be enabling, they can be inhibitory. So changing any element of a component of a \gls{cmp} can, upon re-assembly, be shown to have set off a chain of other effects in components and, ultimately, \gls{cmp} behavior. Tracking down the exact source of misbehavior may be possible, but requires a great deal of work. The \gls{EDEN} \gls{cmp} is entirely open source. I could investigate any element of any component. That is not typical. Many of the \glspl{cmp} that are used in other \glspl{cmp} are closed source. They are protected from modification or investigation by copyright law \citep{Kelty2008-jm}. This means that flawed \glspl{cmp} are rarely, if ever, fixed directly. Instead, a new element would be added to the assembly process to inhibit or sideline the misbehaving component. Every \gls{cmp} is a patchwork of immediate solutions to problems of enormous complexity\footnotemark.

  \footnotetext{A quality they share with many socially important objects}

  \subsection{Living With Composites}
  All of us are already living with \glspl{cmp} every day. Even on days when we don't see an obvious \gls{cmp}, they are now deeply involved in the management of our world. They impact the space we live in even if they are not present \citep{Kitchin2011-af}. One might imagine that the deceptive and unpredictable nature of \glspl{cmp} would make them overwhelming and confusing, and reports of being overwhelmed and confused by software are common enough. At the same time, the idea of software as objective (like the idea of facts as objective) persists. 

  I suspect that is because we were prepared for the complexity, opacity and unmarked agendas of \glspl{cmp} by another common feature of our world: humans. I think \glspl{cmp} are the closest we have come to realizing the dream of creating a new kind of creature. As speculative fiction authors have pointed out the results are often nightmarish, but they can also be deeply compelling. \Glspl{cmp} have enabled people to accomplish many tasks that once seemed impossible. The relative value of Wikipedia or Google or Facebook is open to debate, but there is an awesome material force to the way humans have used \glspl{cmp} to transform the way many people live in interact with each other.

  I fear humanity is sleepwalking into a complicated future. Complex times demand complicated tools to meet them and I do not think we should (or could) abandon \glspl{cmp}. That does not mean we can afford to ignore their specific qualities and how those qualities will lead to \glspl{cmp} suddenly, unexpectedly diverging from social expectations. We must speak honestly about accepting limits on our own ability to understand the things we've created and ensuring that \glspl{cmp} are placed in situations where their agency is limited and mediated. 
  \newpage

% ============================================
% =   ____ _                                 =
% =  / ___| | ___  ___ ___  __ _ _ __ _   _  =
% = | |  _| |/ _ \/ __/ __|/ _` | '__| | | | =
% = | |_| | | (_) \__ \__ \ (_| | |  | |_| | =
% =  \____|_|\___/|___/___/\__,_|_|   \__, | =
% =                                   |___/  =
% ============================================
   \singlespacing
   \printnoidxglossaries

  \section{Additional Thanks}
  This project was produced in a digital soup and I have drawn on a nearly endless list of resources in order to complete it. These are not sources I cited, but without them you would not have the work before you.

  \subsection{Latex}
  \begin{itemize}
    \item \url{http://www.personal.ceu.hu/tex/breaking.htm}
    \item \url{https://github.com/SublimeText/LaTeXTools/issues/1082}
    \item \url{https://www.overleaf.com/latex/templates/your-apa6-style-manuscript/kngbbqpypjcq}
    \item \url{https://gking.harvard.edu/files/natnotes2.pdf}
    \item \url{https://tex.stackexchange.com/questions/74170/have-new-line-between-paragraphs-no-indentation}
    \item \url{https://tex.stackexchange.com/questions/171803/change-font-size-of-the-verbatim-environment}
    \item \url{https://tex.stackexchange.com/questions/32208/footnote-runs-onto-second-page}
    \item \url{https://stackoverflow.com/questions/2895780/how-to-code-tables-with-multi-line-cells}
    \item \url{https://tex.stackexchange.com/questions/62278/problems-with-endfloat-package}
    \item \url{https://www.overleaf.com/learn/latex/code_listing}
    \item \url{https://tex.stackexchange.com/questions/171803/change-font-size-of-the-verbatim-environment}
  \end{itemize}

  \subsection{Failed attempt to incorporate images}
  \begin{itemize}
    \item \url{https://tex.stackexchange.com/questions/542766/inkscape-1-0-not-able-to-export-files-needed-for-svg-package}
    \item \url{https://tex.stackexchange.com/questions/2099/how-to-include-svg-diagrams-in-latex}
    \item \url{https://www.overleaf.com/learn/latex/Inserting_Images}
  \end{itemize}

  \subsection{Other Resources}
  \begin{itemize}
    \item \url{https://stackoverflow.com/questions/1945075/how-do-i-create-binary-patches}
  \end{itemize}

  \subsection{Python}
  \begin{itemize}
    \item \url{https://stackoverflow.com/questions/17912307/u-ufeff-in-python-string}
    \item \url{https://gist.github.com/thatalextaylor/7408395}
  \end{itemize}


  \subsection{Tools}
  \begin{itemize}
    \item \url{https://www.kammerl.de/ascii/AsciiSignature.php}
    \item \url{https://www.tablesgenerator.com/}
  \end{itemize}

% =================================
% =  _____     _     _            =
% = |_   _|_ _| |__ | | ___  ___  =
% =   | |/ _` | '_ \| |/ _ \/ __| =
% =   | | (_| | |_) | |  __/\__ \ =
% =   |_|\__,_|_.__/|_|\___||___/ =
% =================================                             
  

   \section{Tables}
   These tables are missing some information that is available in this Google sheet: \url{https://bit.ly/Eden_Stats}. Additionally, the contributor who is identified as 'sherdim' used their real name, but I was unable to get \LaTeX to display their name properly, so I have used their github handle. Their profile can be found here: \url{https://github.com/sherdim}.

\begin{table}[ht]
\caption{Eden Contributors}
\label{tab:contribs-1}
\begin{tabular}{|l|r|r|}
\hline
Name            & Related Changes & Total Changes \\ \hline
Ashwyn          & 23              & 7,580         \\ \hline
Aviral Dasgupta & 88              & 6,147         \\ \hline
biplovbhandari  & 6               & 26,692        \\ \hline
Daniel Drexler  & 69              & 10,342        \\ \hline
Dominic König   & 14,070          & 1,940,890     \\ \hline
Fran Boon       & 12,314          & 3,816,361     \\ \hline
Graeme Foster   & 236             & 241,090       \\ \hline
hitesh96db      & 4               & 13,588        \\ \hline
James O'Neill   & 4               & 4,038         \\ \hline
Kunal Hari      & 11              & 8,920         \\ \hline
Michael Howden  & 313             & 143,771       \\ \hline
Pratyush Nigam  & 104             & 290           \\ \hline
raj454raj       & 1               & 10,917        \\ \hline
redsin          & 278             & 25,265        \\ \hline
tirgil          & 2               & 3,804         \\ \hline
VishrutMehta    & 2               & 1,153         \\ \hline
\end{tabular}
\end{table} 

\begin{table}[ht]
\caption{pydal Contributors 1}
\label{tab:contribs-2.1}
\begin{tabular}{|l|r|r|}
\hline
Name                 & Related Changes & Total Changes \\ \hline
abastardi            & 9               & 15            \\ \hline
alan                 & 2               & 11            \\ \hline
boa-py               & 2               & 2             \\ \hline
BuhtigithuB          & 670             & 5,524         \\ \hline
Cássio Botaro        & 2               & 2             \\ \hline
Christophe Varoqui   & 2               & 2             \\ \hline
Dan Feeney           & 11              & 11            \\ \hline
David Orme           & 30             & 30            \\ \hline
Dominic König        & 13              & 43            \\ \hline
dz0                  & 1               & 1             \\ \hline
Emmanuel Goh         & 4               & 4             \\ \hline
Fran Boon            & 3               & 21            \\ \hline
Francisco Tomé Costa & 4               & 4             \\ \hline
gi0baro              & 14,046          & 69,785        \\ \hline
Giovanni Barillari   & 6,112           & 8,240         \\ \hline
ilvalle              & 1,968           & 37,158        \\ \hline
Jack Kuan            & 7               & 7             \\ \hline
JusticeN             & 15              & 15            \\ \hline
jvanbraekel          & 40              & 40            \\ \hline
kvanzuijlen          & 2               & 2             \\ \hline
Leonel Câmara        & 191             & 896           \\ \hline
\end{tabular}
\end{table}

\begin{table}[ht]
\caption{pydal Contributors 2}
\label{tab:contribs-2.2}
\begin{tabular}{|l|r|r|}
\hline
Name                 & Related Changes & Total Changes \\ \hline
Martin Doucha        & 1,992           & 3,037         \\ \hline
maxcrystal           & 11              & 11            \\ \hline
Massimo DiPierro     & 6,320           & 903,691       \\ \hline
Michael Loster       & 2               & 2             \\ \hline
Michele Comitini     & 52              & 5,975         \\ \hline
nikakis              & 18              & 18            \\ \hline
niphlod              & 816             & 30,396        \\ \hline
preactive            & 2               & 2             \\ \hline
Remco Boerma         & 5               & 5             \\ \hline
Richard Boß          & 5               & 5             \\ \hline
rodwatkins           & 14              & 14            \\ \hline
Stephen Rauch        & 2,852           & 4,993         \\ \hline
Stephen Tanner       & 7               & 43            \\ \hline
Tim Nyborg           & 6               & 6             \\ \hline
Tom Stratton         & 9               & 9             \\ \hline
Victor Salgado       & 12              & 18            \\ \hline
Vinyl Darkscratch    & 2               & 49            \\ \hline
Wanderson Reis       & 5               & 5             \\ \hline
willimoa             & 7               & 64            \\ \hline
xuyangbocn           & 2               & 2             \\ \hline
\end{tabular}
\end{table}

\begin{table}[ht]
\caption{yatl Contributors}
\label{tab:contribs-4}
\begin{tabular}{|l|r|r|}
\hline
Name                           & Related Changes & Total Changes \\ \hline
Carlos Cesar Caballero Díaz    & 17              & 17            \\ \hline
Massimo DiPierro               & 1,689           & 2,089         \\ \hline
PhanterJR                      & 127             & 263           \\ \hline
\end{tabular}
\end{table}

\begin{table}[ht]
\caption{web2py Contributors 1}
\label{tab:contribs-3-1}
\begin{tabular}{|l|r|r|}
\hline
Name                           & Related Changes & Total Changes \\ \hline
abastardi                      & 27              & 153           \\ \hline
Adam Bryzak                    & 2               & 2             \\ \hline
Alexander Zayats               & 8               & 8             \\ \hline
alexdba                        & 17              & 17            \\ \hline
Alfonso de la Guarda Reyes     & 19              & 63            \\ \hline
Anssi Hannula                  & 10              & 12            \\ \hline
Batchu Venkat Vishal           & 2               & 2             \\ \hline
Carlos Cesar Caballero Díaz    & 34              & 3,136         \\ \hline
Carlos Costa                   & 13              & 236           \\ \hline
Cássio Botaro                  & 22              & 207           \\ \hline
cccaballero                    & 10              & 138           \\ \hline
Chen Rotem Levy                & 8               & 309           \\ \hline
Chris DeGroot                  & 4               & 4             \\ \hline
Chris Garcia                   & 12              & 34            \\ \hline
clach04                        & 45              & 45            \\ \hline
Daniel Libonati                & 2               & 9             \\ \hline
Denis Rykov                    & 4               & 4             \\ \hline
Dinis                          & 4               & 470           \\ \hline
Dominic König                  & 2               & 10            \\ \hline
Donald McClymont               & 7               & 7             \\ \hline
Erik Montes                    & 2               & 147           \\ \hline
Fran Boon                      & 11              & 28            \\ \hline
geomapdev                      & 153             & 223           \\ \hline
gi0baro                        & 12,022          & 38,807        \\ \hline
Giovanni Barillari             & 65              & 346           \\ \hline
hectord                        & 6               & 6             \\ \hline
ilvalle                        & 1,019           & 65,421        \\ \hline
Jack Kuan                      & 8               & 24            \\ \hline
Jan Beilicke                   & 13              & 13            \\ \hline
Jan M. Knaup                   & 12              & 12            \\ \hline
Jaripekkaf                     & 56              & 56            \\ \hline
Jeremie Dokime                 & 14              & 54            \\ \hline
Joel Rathgaber                 & 22              & 22            \\ \hline
Jonathan Bohren                & 57              & 59            \\ \hline
Jonathan Vasek                 & 2               & 2             \\ \hline
Jose C                         & 1               & 1             \\ \hline
jvanbraekel                    & 62              & 108           \\ \hline
kelson                         & 35              & 306           \\ \hline
Kiran Subbaraman               & 88              & 99            \\ \hline
Koen van Zuijlen               & 4               & 8             \\ \hline
Kristján Valur Jónsson         & 2               & 2             \\ \hline
\end{tabular}
\end{table}

\begin{table}[ht]
\caption{web2py Contributors 2}
\label{tab:contribs-3-2}
\begin{tabular}{|l|r|r|}
\hline
Name                           & Related Changes & Total Changes \\ \hline
Kurt Grutzmacher               & 352             & 2,378         \\ \hline
Leonel Câmara                  & 1,236           & 14,051        \\ \hline
Lisandro                       & 8               & 14            \\ \hline
Luca de Alfaro                 & 2               & 2             \\ \hline
Martin Doucha                  & 51              & 99            \\ \hline
Massimo DiPierro               & 76,420          & 825,014       \\ \hline
Mathieu Clabaut                & 24              & 88            \\ \hline
mcabo                          & 28              & 28            \\ \hline
Michele Comitini               & 1,284           & 6034          \\ \hline
mictee                         & 33              & 71            \\ \hline
Mirko Galimberti               & 11              & 52            \\ \hline
mpranjic                       & 6               & 36            \\ \hline
Nik Klever                     & 7               & 45            \\ \hline
niphlod                        & 6,304           & 45,996        \\ \hline
Oleg                           & 41              & 43            \\ \hline
Omar Trinidad Gutiérrez Méndez & 7               & 321           \\ \hline
omniavx                        & 2               & 2             \\ \hline
Oscar Fonts                    & 16              & 585           \\ \hline
Oscar Rodriguez                & 66              & 66            \\ \hline
Paolo Caruccio                 & 6               & 475           \\ \hline
Prasad Muley                   & 23              & 23            \\ \hline
Radu Ioan Fericean             & 30              & 66            \\ \hline
Ricardo Pedroso                & 108             & 263           \\ \hline
Richard Vézina                 & 5,135           & 17,379        \\ \hline
samuel bonilla                 & 4               & 93            \\ \hline
Scimonster                     & 4               & 6             \\ \hline
Seth Kinast                    & 10              & 18            \\ \hline
spametki                       & 473             & 3,404         \\ \hline
Stefan Pochmann                & 114             & 114           \\ \hline
tiago.bar                      & 2               & 4             \\ \hline
tim                            & 2               & 2             \\ \hline
Tim Nyborg                     & 43              & 109           \\ \hline
Tim Richardson                 & 222             & 337           \\ \hline
viniciusban                    & 18              & 261           \\ \hline
Vinyl Darkscratch              & 150             & 121,159       \\ \hline
winniehell                     & 5               & 5             \\ \hline
zvolsky                        & 11              & 1,543         \\ \hline
sherdim                        & 2               & 14            \\ \hline
\end{tabular}
\end{table}
   \setlength{\parindent}{4em}
   \bibliography{Final}

\end{document}