%  _____ _             _    ______             __ _ __  
% |  ___(_)_ __   __ _| |  / /  _ \ _ __ __ _ / _| |\ \ 
% | |_  | | '_ \ / _` | | | || | | | '__/ _` | |_| __| |
% |  _| | | | | | (_| | | | || |_| | | | (_| |  _| |_| |
% |_|   |_|_| |_|\__,_|_| | ||____/|_|  \__,_|_|  \__| |
%                          \_\                      /_/ 
                   
% -------------------------------------------------------------------------
% https://www.kammerl.de/ascii/AsciiSignature.php - standard
% I've moved beyond justifying latex and am not just living in it.


% Use APA module
% Based on template from https://www.overleaf.com/latex/templates/your-apa6-style-manuscript/kngbbqpypjcq
\documentclass[a4paper,man,natbib]{apa6}
% Imports natbib
% Ref Sheet: https://gking.harvard.edu/files/natnotes2.pdf

% Packages
\usepackage[english]{babel}
\usepackage[utf8x]{inputenc}
\usepackage{amsmath}
\usepackage{graphicx}
% set single spacing
\usepackage{setspace}
% glossary
\usepackage[acronym]{glossaries}
% fancy quotes
\usepackage{epigraph, varwidth}
% For formatting of potential bibliography
\usepackage{enumitem}
% html links
\usepackage{hyperref}
%\usepackage[colorinlistoftodos]{todonotes}

% This section was found in a stack overflow comment about making the epigraph length different and I include it here as an incantation against bad formatting
\renewcommand{\epigraphsize}{\small}
\setlength{\epigraphwidth}{0.6\textwidth}
\renewcommand{\textflush}{flushright}
\renewcommand{\sourceflush}{flushright}
% A useful addition
\newcommand{\epitextfont}{\itshape}
\newcommand{\episourcefont}{\scshape}

\makeatletter
\newsavebox{\epi@textbox}
\newsavebox{\epi@sourcebox}
\newlength\epi@finalwidth
\renewcommand{\epigraph}[2]{%
  \vspace{\beforeepigraphskip}
  {\epigraphsize\begin{\epigraphflush}
   \epi@finalwidth=\z@
   \sbox\epi@textbox{%
     \varwidth{\epigraphwidth}
     \begin{\textflush}\epitextfont#1\end{\textflush}
     \endvarwidth
   }%
   \epi@finalwidth=\wd\epi@textbox
   \sbox\epi@sourcebox{%
     \varwidth{\epigraphwidth}
     \begin{\sourceflush}\episourcefont#2\end{\sourceflush}%
     \endvarwidth
   }%
   \ifdim\wd\epi@sourcebox>\epi@finalwidth 
     \epi@finalwidth=\wd\epi@sourcebox
   \fi
   \leavevmode\vbox{
     \hb@xt@\epi@finalwidth{\hfil\box\epi@textbox}
     \vskip1.75ex
     \hrule height \epigraphrule
     \vskip.75ex
     \hb@xt@\epi@finalwidth{\hfil\box\epi@sourcebox}
   }%
   \end{\epigraphflush}
   \vspace{\afterepigraphskip}}}
\makeatother
% End epigraph modifications

\makenoidxglossaries

\singlespacing

\loadglsentries{glossary}

\renewcommand{\bibsection}{\section*}

% \title{Software: Gradually and Then Suddenly}
\title{Composites: A Maze of Twisty Passages, None Alike}
\shorttitle{Composites}
\author{Daniel "Drex" Drexler}
\affiliation{Center for Science, Technology and Society at Drexel University}
\date{April 2020}

\abstract{A study of software, the way it materializes perspectives, and the limits of the promulgations of those perspectives.}

\begin{document}
   \maketitle
   % This is a progressive work that's trying to explain my situatedness
   \section*{It's Software's World, We Just Live in it}
   Software worms its way into every nook and cranny of our world. Part of this project was working \textit{within} software, but in a much greater sense software mediates nearly all modern academic work. The systems that we use to typeset, distribute and display academic work are all software systems. Even the printing of physical books is now accomplished through software control systems. Software systems have become the common medium through which systems of maitinence and control are managed. This means that objects can be maintained remotely, but also that our experience of agentic power over the things we posses is changing \citep{Koebler2017-dt}. Very few of the goals people hope to accomplish through software are new. This project is not founded on the idea that software changes the nature of the world. However, software (and the \glspl{sa} that software projects produce) work in was that are particular to sofware. Having communication or control or any other abstract goal managed through software will have qualities that are particular to software and are often side effects \citep{Mackenzie2006-hb, Kitchin2011-af}. This work began by trying to understand how the ideas of situated knowledges and strong objectivity can inform the creation of software \citep{Harding1992-od,Haraway1988-nh}. Software, like all material things, encodes the perspectives and assumptions of its makers. But what use is it, really, to point out that the assumptions of Software makers remain in the things they produce? I hoped to find a path from finding and detailing perspectives encoded in software to a direct intervention at the software level \citep{Zuiderent-Jerak2015-go}. My approach assumed that details in the material qualities of software that would be useful in making a situated intervention into the perspectives encoded into it. Ultimately, it was the material aspects of how software exists differently for different stakeholders in its creation and use that would come to dominate that rest of the project.

   I believe the limits to applying situated knowledges to software lie within the blackbox of software itself \citep{Latour1999-ui}. Once the components that compose software are removed an examined, it becomes clear that software developers are always looking at the components inside the blackbox, while users always see it from the outside. The understand what must be changed involves working \textit{across} black box layer(s) in a way that is unique to software. Though Latour's blackbox framework of encapsulated functionality is a useful starting point, the reality of software is that every \gls{sa} relies on action across many levels of blackbox'ing \citep{Latour1999-ui}. It goes beyond Latour's framework and into a kind of object that I call a \textit{composite}, an object that itself is a black box and which contains black boxes, but where Latour claims that blackbox boundaries become socially visible only in failure, \textit{composites} are designed to use heterogeneous interactions across the boundaries of its constituent blackboxes. As with other blackboxes, the things inside a blackbox are normally invisible, and so the limits of situated knowledge are found in the difference between the situation of the user and the developer. The user, having no way to assess or understand how many different components are involved in each software component, can only communicate about the experience that emerges out of the final \textit{composite} of all components. The developer, however, is always making changes to a particular blackbox at a particular level and, though they also use the software as a user, the path from that user-experience to understanding how to fix problems (or how the developer's standpoint has impacted a given \gls{sa}) is unclear. This quality of software comes from how software is assembled into composites and shapes how the perspectives of those who make software come to be encoded in \glspl{sa}. The views of those who produce software are reliably represented, but those representations exist at different levels of the \textit{composite} how each materialized perspective comes to be seen by users of the artifact is highly individual.

   Returning to the question of what one can do to address perspectives in software, this project takes the \acrfull{foss} project \acrfull{eden} as its object of study and site of intervention. I chose \acrshort{eden} because it is, as far as I can tell, a well-made piece of software that fulfills its role in the world. Others will engage socially problematic software and try to fix it, but the goal of my work here is to engage with a piece of software which is not obviously harmed by the standpoints of its developers. This does not aim to be a critical engagement, but an engagement whose successes and failures may be informative for future critical projects. The result of the modifications to \acrshort{eden} can be seen at <url>. The changes to \acrshort{eden} highlight where \acrshort{eden} crosses blackbox boundaries, though my efficacy in making those crossings apparent is limited to particular \textit{kinds} of boundaries. As we will discuss, software is deeply heterogeneous and that heterogeneity impacts the work that is needed to engage with it. 

   \section*{What can we Say about Software?}
  % Social organizations around software
  % Sources in this section:   
  % Boellstorff, Tom. Coming of Age in Second Life: An Anthropologist Explores the Virtually Human. Princeton University Press, 2015.
  %   - legitimacy of doing anthropoligical work in purely digital spaces, actual v.s. virtual - all real
  % Cox, Geoff, and Christopher Alex McLean. Speaking Code: Coding as Aesthetic and Political Expression. MIT Press, 2013.
  %   - Background on software studies
  %   - Code as speech but *not* software as speech
  %   - Involves software, but as seen through the act of "coding"
  % Ensmenger, Nathan L. The Computer Boys Take Over: Computers, Programmers, and the Politics of Technical Expertise. MIT Press, 2012.
  %   - Social history of programming and selecting programmers
  %   - More about the particular social and historical shape of programming and programmers
  %   - background for ethnographic work
  % Gabriella Coleman, E. Coding Freedom: The Ethics and Aesthetics of Hacking. Princeton University Press, 2012.
  %   - social idea of hacking and of freedom
  %   - ways of experiencing freedom within software and software groups
  % Kelty, Christopher M. Two Bits: The Cultural Significance of Free Software. Duke University Press, 2008.
  %   - rules of FOSS give rise to certain consistent social rules
  \subsection*{The Social World of Software Makers}
   A great deal of excellent sociological work has been done on communities that produce various kinds of software. The communities that have been studied often have particular qualities of their social structures connected to their shared work of creating, maintaining and promoting software \citep{Kelty2008-jm}. Other work has focused less on the work of making software per-se and has instead highlighted how ideas of freedom, creativity and openness (of a particular kind) attract people to communities that make software \citep{Gabriella_Coleman2012-lq}. These ethnographic engagements with communities that produce software share a perspective that certain kinds of social structures are better suited for producing software (in general) and \textit{certain kinds of software in particular}. The \acrshort{foss} movement, Kelty claims, has a social structure that flows directly from the ethics of sharing and attribution that unite the software projects under the \acrshort{foss} banner \citep{Kelty2008-jm}. While there are important lacuna in \citet{Kelty2008-jm}'s somewhat utopian descriptions of \acrshort{foss} communities, especially for non-male and culturally heterogeneous people, it seems true that social structure and technical goals co-create and co-evolve with each other \citep{Penny2013-ic, Dean2010-lk}. Similar conclusions are reached by \citet{Boellstorff2015-al} when looking at online cultures and \citet{Ensmenger2012-kz} in his study of the history of the profession of computer programming. \citet{Boellstorff2015-al} vividly shows the validity of considering people spending in time in digital spaces \textit{isolated from} any actual world\footnotemark referents. He freely notes relationships between the accounts his interlocutors give of their actual world activities, but understands that it's still valid to engage with those representations directly without trying to prove any truth-claims about them. \citet{Ensmenger2012-kz}'s work is historical rather than ethnographic, but also supports the ways in which the material demands of making software created social structures that have endured until today. There are consistent links between the nature of the virtual and the social structure of the actual people who create digital products, but each of those qualities can be examined independently without needing to fully understand the nature of the links. \citet{Gabriella_Coleman2012-lq}'s focus, in particular, on "hacking" cultures that exist outside of any particular culture of "software" or "code" points to these social structures fitting particularly well with software, but not proceeding directly from software per-se \citep{Drexler2019-ja}. This is in line with \citet{Cox2013-zo}'s examination of the expressive qualities of "coding" (which may or may not be a synonym for "software") and how they relate to the long history of speech as well as how the particular material qualities of computers open up new avenues for expression. For this project, these works suggest that \acrshort{eden} does have a particular culture around it, especially because \acrshort{eden}'s makers (the Sahana Foundation) places the project within the \acrshort{foss} world of software development and the \acrfull{ict4d} world. Studying this particular social situation would doubtless be productive work and continue the conversation around the social character of collectives that create software and try to promote \acrshort{foss} ideas in particular historical, economic and legal regimes. However, this work also suggests that while the a particular \gls{so} and the culture around and within that \gls{so} co-create and drive each other's evolution, they also suggest that either can be considered in isolation at a particular moment in time. Indeed, \citet{Kelty2008-jm} and \citet{Gabriella_Coleman2012-lq}'s observations both depend on connecting independently observed qualities of \textit{culture} and \textit{particular \acrshort{foss} projects} in order to draw out the co-constitutive qualities. This project leaves whatever co-constitutive cultural links that exist between \acrshort{eden} and the people and cultures that produce it to others, though I am quite sure interesting work exists there.

    \footnotetext{\citet{Boellstorff2015-al} rejects the idea that events in digital spaces (like games or chat rooms) are any less "real" than things that exist in the physical world and thus rejects the common language of "real" and "virtual." He instead talks about "actual" (physical) spaces and "virtual" (within digital worlds) spaces. I will be describing the physical world as the "actual" world when appropriate and will talk about both virtual space (space that fully exist in computers such as Second Life and World of Warcraft) and digital space (spaces that are displayed though digital mediums, but are constituted of a mix of real and generated media [facebook, instagram]). It is all "real."}

  % Social footprint of software and digital cultures
  % Sources in this section:
  % Boellstorff, Tom. “The Opportunity to Contribute: Disability and the Digital Entrepreneur.” Information, Communication and Society 22, no. 4 (March 21, 2019): 474–90.
  %   - pushing back against reducing the social to the commercial -or- a higlighting of the social character of commercial work
  % Humphreys, Lee. The Qualified Self: Social Media and the Accounting of Everyday Life. MIT Press, 2018.
  %   - use of digital medium to demonstrate qualificiations
  %   - NOT SURE I'LL USE
  % Kitchin, Rob, and Martin Dodge. Code/space: Software and Everyday Life. MIT Press, 2011.
  %   - power of code / software in social space
  %   - creation of spaces through software
  % Ullman, Ellen. Close to the Machine: Technophilia and Its Discontents. Picador, 2012.
  %   - great quotes about the experience of writing software and how the creative expression in code comes to exist in the machine
  % Jurgenson, Nathan. The Social Photo: On Photography and Social Media. Verso Books, 2019.
  %   - use of media to transmit social qualities
  %   - structural nature of the world as mediated by software giving rise to particular kinds of expression and communication
  % Tiidenberg, Katrin, ed. Selfies: Why We Love (and Hate) Them. Emerald Publishing Limited, 2018.
  %   - more on using media to communicate meaning
  %   - new forms of media facing hostility

  % TODO: move age in second life into this section
  \subsection*{The Social World Software Helps Construct for Us}
   Even without a direct ethnographic engagement with the team working on \acrshort{eden} (or its users), previous work on software and its effect on the world has been useful for guiding the focus of this project. It goes too far to say that the nature of socializing has been transformed by digital networked communication, but society (and social scientists) are engaged in an ongoing effort to sort out truly new ways of using media and new forms for existing modes of socialization. Much of this work has focused on modern, media-focused ways of keeping socially connected with people \citep{Humphreys2018-ge,Jurgenson2019-tl,Tiidenberg2018-gh}. This work is not about making software, but speaks to the power \textit{of} software to structure and channel social activities. Like the social structures that exist around the act of making software, living with software pushes us to act in particular ways. \citet{Humphreys2018-ge} both highlights how modern practices of using global digital networks to keep in touch with each other proceed directly from more localized historical practices, and how the way we represent ourselves in digital spaces reflect possible social realities that only exist in those spaces. Software places is in new social configurations: a neighbor I have never met could see my vacation photos without asking me, I bring my assumptions developed over a career of software development for US software startups to my review of an internationally focused (but originally Indonesian) system for disaster management. Though each digital connection has a particular character and the field is difficult to tantalize accurately, many of the connections that have become more common bridge greater social distances than previously possible. \citet{Dean2010-lk} highlights this quality of digital spaces when she speaks about the collapse of predictability in how others will see us. The accessibility of digital spaces means that we may be famous "online" but anonymous in person, or visa-versa. 

  % Culture of making software / tools
  % Sources in this section:
  % Mackenzie, A. Cutting Code: Software and Sociality. Edited by Steve Jones. Peter Lang Publishing, 2006.
  %   - relationality of software
  %   - the process of creating software is a process of managing relationships: with the future, with the past, with other software
  % Traweek, Sharon. Beamtimes and Lifetimes. Harvard University Press, 2009.
  %   - cultural charaters of teams show up in the tools that they build to do their work
  % Eubanks, Virginia. Automating Inequality: How High-Tech Tools Profile, Police, and Punish the Poor. St. Martin’s Press, 2018.
  %   - materialization of human perspectives in computer algorithms
  %   - questioning the projection of agency onto computer code v.s. understanding computer code as "doing the work" of humans
  % Schüll, Natasha Dow. Addiction by Design: Machine Gambling in Las Vegas. Princeton University Press, 2012.
  %   - again, material encoding of human intent into technology
  %   - NOT SURE I'LL USE
  % Subramaniam, Banu. Ghost Stories for Darwin: The Science of Variation and the Politics of Diversity. University of Illinois Press, 2014.
  %   - more examples of tools containing material standpoints (specifically the eugenical quality of population statistics)

  % current events things
  % automating inequality
  % moral crumple zones
  % addiction by design
  % ...?
  \subsection*{Software Today}

  % latour black box

  \subsection*{How to think about software}  

  \section*{The object of study}

  \subsection*{EDEN}
   The particular piece of software this project studies is \acrlong{eden}. \acrshort{eden} was originally created by a coalition of Sri Lankan \acrlong{ict} companies in the wake of the 2004 Indian Ocean Earthquake and Tsunami. The project is now managed by the nonprofit Sahana Foundation. It has been used in numerous disaster responses since 2004 and is also used by a number of organizations for managing resources outside of any specific disaster \citep{Sahana_Foundation_undated-hl}. \acrshort{eden}'s functionality can fairly be described as tracking and managing resources. Its power comes from the enormous range and detail of the information it knows how to track. \acrshort{eden} has indexes for organizations, people, projects, events, facilities, supplies, documents, possible scenatios, and events. It also provides on-platform messaging and special tools for the management of emergency shelters \citep{Sahana_Foundation2011-od}. Each resource is meant to be connected to other resources: people work for organizations, projects are run by organizations and are associated with people, facilities are linked with projects and organizations and other resources like supplies and verticals. The flexibility of the software means that it can be used to manage a single organization, multiple organization, or used as a hub for many different groups coordinating around shared goals but without central authority. The system aims to be as comprehensive as possible and to use the same entity for resource and system management. The entries for people can act as a simple rolodex and will also double, if desired, as that person's account within \acrshort{eden}. The interface for managing goods is the same interface for managing access to the management system. To the degree possible, \acrshort{eden} is structured to allow the same people who do the day-to-day work of the organization to administer the system that manages the organizations' resources. It also aims to be an effective tool for people at all levels: administrators can track where people and supplies are allocated and volunteers can access their assignments, documents and information about how to use resources.

   \subsection*{Why EDEN}
   I selected \acrshort{eden} for this project for two primary reasons: it is a \acrlong{foss} software project in a language I am familiar with and it is not obviously problematic. I will detail its open source pedigree shortly, but first I want to explain my second selection criteria. "Improving" software (whatever improving means) can usefully be separated into at least two questions: what quality should change and how do you bring that change about? This project attempts only to speak about the second (likely easier) question. Partially this is because the project was designed to fit within the confines of a final project for a masters degree. This project represents, roughly, three graduate level classes worth of work. I didn't feel like I had time to find \textit{and} fix flaws. So this project should be understood as neutral on the question of, "is \acrshort{eden} good?" Nothing about \acrshort{eden} seems bad to me. They have received numerous awards, glowing testimonials, and are used by many large organizations that could use other products if \acrshort{eden} were lacking. 

   I lack the experience and expertise to say that I think \acrshort{eden} is \textit{good}. Assessing if software is "good" or "bad" is not straightforward. Simply examining \glspl{sa} in isolation tell us very little. It's only through engaging with one or more \gls{sa}(s) as they exist in the lived world and contextualizing that with detailed ethnographic work that it's possible to start making value judgments \citep{Eubanks2018-hc,Schull2012-nc}. Those judgments wouldn't be universal, of course, but would be about a particular population. So this paper is done from the perspective that \acrshort{eden} seems fine to me. The project is interested in the technical details of how the \acrshort{eden} \gls{sa} emerges out of its \gls{so} and how agency is modified by that process. 

   Finally, and briefly, this project does not take the position that the technical qualities it investigates are free from the impact of social structures. \citet{Gabriella_Coleman2012-lq} and \citet{Kelty2008-jm} have both compellingly shown that the social and technical co-produce each other. However, we can still usefully speak about qualities that technical systems have and how those qualities impact our lived experience. That this project has obvious extensions in the social and ethnographic realm is a strength and declining to investigate them should be understood as a concession to time.

   \subsection*{Out of Many, One}
   Wherever possible, \acrshort{eden} uses \acrshort{foss} technologies. The language it is written in, the libraries it relies on to provide functionality, the tools it uses to support its functionality, and its main operating system are all both open source and available at no cost. \acrshort{eden} will generally operate on top of non-\acrshort{foss} systems like Microsoft Windows, but the team doesn't prioritize systems outside of the \acrshort{foss} systems they develop and test on \citep{Sahana_Foundation2015-zs}. \citet{Kelty2008-jm} talks about how \acrshort{foss} is both a philosophy and a system of development that has practical impacts. One of the side effects of \acrshort{eden} committing to use the \acrshort{foss} ecosystem is it makes my form of engagement possible. Though it is possible to examine compiled machine code and draw some conclusions about the intent and process that assembled it, such a project would be far outside my capabilities. Instead, the source code of \acrshort{eden}, web2py (the \gls{web framework} \acrshort{eden} uses), Python (the language \acrshort{eden} and web2py are written in) and all of the libraries used by the project are open source. Their preferred databases (MySql or PostgreSQL) are also fully open source projects. Open source projects often don't just publish the current source code, but offer full histories of what change, when it changed, who changed it and how it was changed. These changes often include notes about why \textit{that particular} change was made over any other possible change \citep[p. 13-16]{Chacon2014-im}. Open source projects also commonly have systems for tracking lists of unfixed flaws as well as planned future improvements (both types of \citet{Mackenzie2006-hb} relationships), but those systems operate above the layer of source code so this project does not engage with them.

  \subsection*{The Inner Workings of EDEN}
  


   
  
   \printnoidxglossaries
   \setlength{\parindent}{4em}
   \bibliography{Final_Draft}

\end{document}