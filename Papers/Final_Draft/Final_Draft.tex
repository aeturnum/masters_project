%  _____ _             _    ______             __ _ __  
% |  ___(_)_ __   __ _| |  / /  _ \ _ __ __ _ / _| |\ \ 
% | |_  | | '_ \ / _` | | | || | | | '__/ _` | |_| __| |
% |  _| | | | | | (_| | | | || |_| | | | (_| |  _| |_| |
% |_|   |_|_| |_|\__,_|_| | ||____/|_|  \__,_|_|  \__| |
%                          \_\                      /_/ 
                   
% -------------------------------------------------------------------------
% https://www.kammerl.de/ascii/AsciiSignature.php - standard
% I've moved beyond justifying latex and am not just living in it.


% Use APA module
% Based on template from https://www.overleaf.com/latex/templates/your-apa6-style-manuscript/kngbbqpypjcq
\documentclass[a4paper,man,natbib]{apa6}
% Imports natbib
% Ref Sheet: https://gking.harvard.edu/files/natnotes2.pdf

% Packages
\usepackage[english]{babel}
\usepackage[utf8x]{inputenc}
\usepackage{amsmath}
\usepackage{graphicx}
% set single spacing
\usepackage{setspace}
% glossary
\usepackage[acronym]{glossaries}
% fancy quotes
\usepackage{epigraph, varwidth}
% For formatting of potential bibliography
\usepackage{enumitem}
% html links
\usepackage{hyperref}
%\usepackage[colorinlistoftodos]{todonotes}

% This section was found in a stack overflow comment about making the epigraph length different and I include it here as an incantation against bad formatting
\renewcommand{\epigraphsize}{\small}
\setlength{\epigraphwidth}{0.6\textwidth}
\renewcommand{\textflush}{flushright}
\renewcommand{\sourceflush}{flushright}
% A useful addition
\newcommand{\epitextfont}{\itshape}
\newcommand{\episourcefont}{\scshape}

\makeatletter
\newsavebox{\epi@textbox}
\newsavebox{\epi@sourcebox}
\newlength\epi@finalwidth
\renewcommand{\epigraph}[2]{%
  \vspace{\beforeepigraphskip}
  {\epigraphsize\begin{\epigraphflush}
   \epi@finalwidth=\z@
   \sbox\epi@textbox{%
     \varwidth{\epigraphwidth}
     \begin{\textflush}\epitextfont#1\end{\textflush}
     \endvarwidth
   }%
   \epi@finalwidth=\wd\epi@textbox
   \sbox\epi@sourcebox{%
     \varwidth{\epigraphwidth}
     \begin{\sourceflush}\episourcefont#2\end{\sourceflush}%
     \endvarwidth
   }%
   \ifdim\wd\epi@sourcebox>\epi@finalwidth 
     \epi@finalwidth=\wd\epi@sourcebox
   \fi
   \leavevmode\vbox{
     \hb@xt@\epi@finalwidth{\hfil\box\epi@textbox}
     \vskip1.75ex
     \hrule height \epigraphrule
     \vskip.75ex
     \hb@xt@\epi@finalwidth{\hfil\box\epi@sourcebox}
   }%
   \end{\epigraphflush}
   \vspace{\afterepigraphskip}}}
\makeatother
% End epigraph modifications

\makenoidxglossaries

\singlespacing

\loadglsentries{glossary}

\renewcommand{\bibsection}{\section*}

\title{Software: Gradually and Then Suddenly}
\shorttitle{Software}
\author{Daniel "Drex" Drexler}
\affiliation{Center for Science, Technology and Society at Drexel University}
\date{April 2020}

\abstract{A study of software, the way it materializes perspectives, and the limits of the promulgations of those perspectives.}

\begin{document}
   \maketitle
   % This is a progressive work that's trying to explain my situatedness
   \section*{It's Software's World, We Just Live in it}
   Software worms its way into every nook and cranny of our world. Part of this paper is based on work I did \textit{in} software, but the paper itself has also moved through software. Systems that were once realized physically have, for reasons beyond the scope of my work, moved into software. This means that academics use software to compose and share their ideas, while farmers fight companies for access to the software that runs their tractors \citep{Koebler2017-dt}. Many of the roles that software has stepped into are not new. This project and paper are not founded on the idea that software allows anything that is unique per-se. I do believe, however, that software (and the artifacts that emerge out of the software process) have particular material qualities that change how they are in the world and the way they impact the world \citep{Mackenzie2006-hb, Kitchin2011-af}. This work began by trying to understand how the ideas of situated knowledges and strong objectivity can inform the creation of software \citep{Harding1992-od,Haraway1988-nh}. Software, like all material things, encodes the perspectives and assumptions of its makers. But what use is it, really, to point out that the assumptions of Software makers remain in the things they produce? I hoped to find a path from finding and detailing perspectives encoded in software to a direct intervention at the software level \citep{Zuiderent-Jerak2015-go}. My approach assumed that details in the material qualities of software that would be useful in making a situated intervention into the perspectives encoded into it. Ultimately, it was the material aspects of how software exists differently for different stakeholders in its creation and use that would come to dominate that rest of the project.

   I believe the limits to applying situated knowledges to software lie within the blackbox of software itself \citep{Latour1999-ui}. Once the components that compose software are removed an examined, it becomes clear that software developers are always looking at the components inside the blackbox, while users always see it from the outside. The understand what must be changed involves working \textit{across} black box layer(s) in a way that is unique to software. Though Latour's blackbox framework of encapsulated functionality is a useful starting point, the reality of software is that every artifact of the software process relies on action across many levels of blackbox'ing \citep{Latour1999-ui}. It goes beyond Latour's framework and into a kind of object that I call a \textit{composite}, an object that itself is a black box and which contains black boxes, but where Latour claims that blackbox boundaries become socially visible only in failure, \textit{composites} are designed to use heterogeneous interactions across the boundaries of its constituent blackboxes. As with other blackboxes, the things inside a blackbox are normally invisible, and so the limits of situated knowledge are found in the difference between the situation of the user and the developer. The user, having no way to assess or understand how many different components are involved in each software component, can only communicate about the experience that emerges out of the final \textit{composite} of all components. The developer, however, is always making changes to a particular blackbox at a particular level and, though they also use the software as a user, the path from that user-experience to understanding how to fix problems (or how the developer's standpoint has impacted a given software-artifact) is unclear. This quality of software comes from how software is assembled into composites and shapes how the perspectives of those who make software come to be encoded in software artifacts. The views of those who produce software are reliably represented, but those representations exist at different levels of the \textit{composite} how each materialized perspective comes to be seen by users of the artifact is highly individual.

   Returning to the question of what one can do to address perspectives in software, this project takes the \acrfull{foss} project \acrfull{eden} as its object of study and site of intervention. I chose \acrshort{eden} because it is, as far as I can tell, a well-made piece of software that fulfills its role in the world. Others will engage socially problematic software and try to fix it, but the goal of my work here is to engage with a piece of software which is not obviously harmed by the standpoints of its developers. This does not aim to be a critical engagement, but an engagement whose successes and failures may be informative for future critical projects. The result of the modifications to \acrshort{eden} can be seen at <url>. The changes to \acrshort{eden} highlight where \acrshort{eden} crosses blackbox boundaries, though my efficacy in making those crossings apparent is limited to particular \textit{kinds} of boundaries. As we will discuss, software is deeply heterogeneous and that heterogeneity impacts the work that is needed to engage with it. 

   \section*{What can we Say about Software?}
   Wider social science studies have consistently argued that software gathers particular social assemblages around it and that different kinds of software attract different kinds of social groups. Chris Kelty argues that the social structure of the groups of people who work on \acrshort{foss} software have been co-constituted with the philosophy of openness and sharing that \acrshort{foss} culture hopes to promote \citep{**kelty}. Gabrielle Coleman has similarly documented how software communities experience social connection and 

   
   
  
   \printnoidxglossaries
   \setlength{\parindent}{4em}
   \bibliography{final}

\end{document}