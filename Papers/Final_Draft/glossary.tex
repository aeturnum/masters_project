\newacronym{foss}{FOSS}{Free Open Source Software}
\newacronym{sts}{STS}{Science, Technology \& Society}
\newacronym{ict}{ICT}{Information and Communications Technology}
\newacronym{eden}{EDEN}{Emergency Development ENvironment}
\newacronym{ict4d}{ICT4D}{Information and Communication Technologies for Development}


\newglossaryentry{ICT}
{
    name=Information and communications technology,
    description={Information and communications technology: A term for the mix of physical devices and software that mix analog and digital techniques to allow communication, recall and storage of data.}
}
 
\newglossaryentry{FOSS}
{
    name=Free Open Source Software,
    description={Free Open Source Software: A method of developing software that does not charge for use of the software. The software's source code is also freely distributed, allowing users to modify their copy of the software or suggest changes to the organization developing it.}
}

\newglossaryentry{python}
{
    name=python,
    description={Python is an interpreted, high-level, general-purpose programming language.}
}

\newglossaryentry{OpenStreetMap}
{
    name=OpenStreetMap,
    description={An open community of volunteers that maintain data about roads, trails, cafés, railway stations, and much more, all over the world.}
}
\newglossaryentry{web framework}
{
    name=web framework,
    description={A heterogeneous set of software that allow a programmer to efficiently write and manage providing a service over the Internet. This could be a website or a mobile app (often it is both) or a go-between for other services. Examples include web2py and Django in Python or Phoenix in Elixir.}
}

\newglossaryentry{PostgreSQL}
{
   name=PostgreSQL,
   description={PostgreSQL is a powerful, open source object-relational database system with over 30 years of active development that has earned it a strong reputation for reliability, feature robustness, and performance.}
   % https://www.postgresql.org/
}

% https://en.wikipedia.org/wiki/MySQL
%  MySQL is a component of the LAMP web application software stack (and others), which is an acronym for Linux, Apache, MySQL, Perl/PHP/Python. MySQL is used by many database-driven web applications, including Drupal, Joomla, phpBB, and WordPress
\newglossaryentry{MySQL}
{
   name=MySQL,
   description={MySQL is an open-source relational database management system (RDBMS). Its name is a combination of "My", the name of co-founder Michael Widenius's daughter, and "SQL", the abbreviation for Structured Query Language}
}

% http://eden.sahanafoundation.org/wiki/WikiStart#WhatisSahanaEden
\newglossaryentry{EDEN}
{
   name=EDEN,
   description={The Emergency Development ENvironment, developed by the Sahana Foundation, is an Open Source Humanitarian Platform which can be used to provide solutions for Disaster Management, Development, and Environmental Management sectors}
}

\newglossaryentry{ICT4D}
{
   name=Information and Communication Technologies for Development,
   description={The movement to use digital technologies to serve the needs of the underprivileged and those living in the developing world. Criticized for being directed more at the developed worlds' love of technology than the actual needs of those it purports to serve. Also called digital humanitarianism}
}

\newglossaryentry{capta}
{
   name=capta,
   description={If data is every piece of information about an entity, capta are the semantically meaningfull elements within that data (Kitchin \& Dodge, 2011). The data of DNA is the entire sequence and the capta would be (we think) the genes that encode for particular traits and proteins}
}

\newglossaryentry{srcd}
{
   name=source code,
   description={Human recognizable language, generally written in primaraly latin text in sets of files. Examples include C, Python (the language of EDEN), LaTeX (the language this paper is assembled by).}
}

\newglossaryentry{mscd}
{
   name=machine code,
   description={A langauge understood by a particular kind of computer chip. Can be generally understoof as a list of specific instructions to the computer chip to do digital operations. I.e. add two numbers, load a value into memory, write a value to disk or send a value over the internet. Common machine languages would be x86 (most laptops and desktops) and ARM (most phones).}
}

\newglossaryentry{so}
{
    name=Software-Object,
    description={A generic term that describes the entire process that leads to the production of a series (or a single) software artifact. This includes the economic and culutral position of the organizing project, the social organization of those who write the code for the piece of software, the social positions of the software artifacts' users, and the software artifacts themselves. This term is used in opposition to the common "software" because of the potential confusion between a "software project" (the organizing structure of a Software-Object), a "program" (a Software-Artifact), and other linguistically confusing ways of speaking about the multi-layered process that produces "software."}
}

\newglossaryentry{sa}
{
    name=Software-Artifact,
    description={An entity of uncertain composition that behaves, for common users (as opposed to technical staff) as a single entity. A software-artifact could be a single binary with all its functionality provided by machine code included in that single location in memory. It could also be a binary that makes calls to various libraries. In the case of EDEN, it's an expensive set of Python files - some of which are human readable and some of which have been modified to make them faster to read. In this case a seperate program, the Python binary itself, reads and executes the EDEN program. Even though EDEN is many seperate files and contains no machine code, it is properly considered a software-artifact because users interact with it as a single entity.  Software-artifacts are produced periodically by software projects, but are not themselves entirely "software." The idea of software must be large enough to allow multiple software-artifacts to exist within the same project: the current version of your phone OS and the next version, the Facebook app on your phone and the webserver that tells it what has happened. Nevertheless, when we are using "software" we are always directly interacting with one or more software-artifact(s). They are the material reality of software projects.}
}

\newglossaryentry{webframework}
{
  name=Web Framework,
  description={A library (or set of libraries) that handles translation from and to the language of HTTP and other web technologies. The library of functionality they provide can range from wrapping basic text in a format a web browser will understand to providing functionality that allows users to log in, recieve realtime messages and interact with the same web site from multiple clients simultaniously (Statz, 2010). Each Web framework also has its own philosophy about what information the developer is expected to provide and how much the developer needs to set. Some frameworks attempt to provide everything that is required to build almost any website, while others focus on providing a core of functionality and may not include support for connecting to a database or managing logins. Generally, any web framework may be used to make any website, but framework selection will dramatically change the experience of the software engineers developing that web site.}
}



% \newglossaryentry{}
% {
%    name=,
%    description={}
% }