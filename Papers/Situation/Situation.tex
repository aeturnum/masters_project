%  ____  _ _               _   _               __  __                _  __ 
% / ___|(_) |_ _   _  __ _| |_(_)_ __   __ _  |  \/  |_   _ ___  ___| |/ _|
% \___ \| | __| | | |/ _` | __| | '_ \ / _` | | |\/| | | | / __|/ _ \ | |_ 
%  ___) | | |_| |_| | (_| | |_| | | | | (_| | | |  | | |_| \__ \  __/ |  _|
% |____/|_|\__|\__,_|\__,_|\__|_|_| |_|\__, | |_|  |_|\__, |___/\___|_|_|  
%                                      |___/          |___/                
% -------------------------------------------------------------------------
% I've moved beyond justifying latex and am not just living in it.


% Use APA module
% Based on template from https://www.overleaf.com/latex/templates/your-apa6-style-manuscript/kngbbqpypjcq
\documentclass[a4paper,man,natbib]{apa6}
% Imports natbib
% Ref Sheet: https://gking.harvard.edu/files/natnotes2.pdf

% Packages
\usepackage[english]{babel}
\usepackage[utf8x]{inputenc}
\usepackage{amsmath}
\usepackage{graphicx}
% set single spacing
\usepackage{setspace}
% glossary
\usepackage[acronym]{glossaries}
% fancy quotes
\usepackage{epigraph, varwidth}
% For formatting of potential bibliography
\usepackage{enumitem}
% html links
\usepackage{hyperref}
%\usepackage[colorinlistoftodos]{todonotes}

% This section was found in a stack overflow comment about making the epigraph length different and I include it here as an incantation against bad formatting
\renewcommand{\epigraphsize}{\small}
\setlength{\epigraphwidth}{0.6\textwidth}
\renewcommand{\textflush}{flushright}
\renewcommand{\sourceflush}{flushright}
% A useful addition
\newcommand{\epitextfont}{\itshape}
\newcommand{\episourcefont}{\scshape}

\makeatletter
\newsavebox{\epi@textbox}
\newsavebox{\epi@sourcebox}
\newlength\epi@finalwidth
\renewcommand{\epigraph}[2]{%
  \vspace{\beforeepigraphskip}
  {\epigraphsize\begin{\epigraphflush}
   \epi@finalwidth=\z@
   \sbox\epi@textbox{%
     \varwidth{\epigraphwidth}
     \begin{\textflush}\epitextfont#1\end{\textflush}
     \endvarwidth
   }%
   \epi@finalwidth=\wd\epi@textbox
   \sbox\epi@sourcebox{%
     \varwidth{\epigraphwidth}
     \begin{\sourceflush}\episourcefont#2\end{\sourceflush}%
     \endvarwidth
   }%
   \ifdim\wd\epi@sourcebox>\epi@finalwidth 
     \epi@finalwidth=\wd\epi@sourcebox
   \fi
   \leavevmode\vbox{
     \hb@xt@\epi@finalwidth{\hfil\box\epi@textbox}
     \vskip1.75ex
     \hrule height \epigraphrule
     \vskip.75ex
     \hb@xt@\epi@finalwidth{\hfil\box\epi@sourcebox}
   }%
   \end{\epigraphflush}
   \vspace{\afterepigraphskip}}}
\makeatother
% End epigraph modifications

\makenoidxglossaries

\singlespacing

\loadglsentries{glossary}

\renewcommand{\bibsection}{\section*}

\title{Mapping EDEN: Revealing the territory created by society and software}
\shorttitle{Mapping EDEN}
\author{Daniel "Drex" Drexler}
\affiliation{Center for Science, Technology and Society at Drexel University}
\date{January 2020}
\begin{document}
   \maketitle
   % This is a progressive work that's trying to explain my situatedness
   \section*{Casting My Lots}
   \epigraph{The point is to make a difference in the world, to cast our lot for some ways of life and not others}{\textit{Donna Haraway \\ Modest\_Witness p. 36}}

   \citet{Harding1992-od}'s strong objectivity contains within it two core objectives: to make meaningful claims and to meaningfully situate those claims. When I hold a mirror up to myself and give an accounting of who I am and why someone might want to listen to me, it gives the reader an opportunity to consider my judgment in a different domain than the subject of this project. Before you must consider how seriously to take my research, you have an opportunity to decide how seriously to take me.

   Software is tricky to speak about. One is frequently talking about more than one piece of software. \acrshort{eden} is often encountered as one thing, but it is a composite of many things. The map of its components quickly becomes fractal. The Sahana Foundation's code is divided into many modules with various purposes and various authors while software frameworks it draws functionality from (incompletely: the Python language, relational databases, the web2py web framework) are themselves separated into various components with various authors, goals and functions. To assess how much any of these individual kernels of functionality contribute to \acrshort{eden}'s functionality \textit{on a programmatic level} would be challenging and is beyond the scope of this project. To characterize the relationship between the code/space created in the actual world by \acrshort{eden} and the \textit{software code} that makes up \acrshort{eden} would be yet another layer of complexity. The jump from code to the actual world is not a move I attempt.

   My experience and expertise comes from eight years of doing professional software engineering (what \citet{Ensmenger2012-kz} would call craft expertise) and two years of studying \acrlong{sts} at Drexel University. It does not include the design of software to help manage disaster response or the management of a project to produce such software. This project focuses on qualities found in the \textit{code} of \acrshort{eden} in the full knowledge that I am ill equipped to understand the impact of \acrshort{eden} in the real world.

   Throughout this project I speak about characteristics of the source code of \acrshort{eden} and how the experience of using \acrshort{eden} emerges out of a combination of code, tooling, design and apparent intent. I talk about places where I've found instances that seem to me like errors and omissions in the code of \acrshort{eden} and how those instances relate to the standpoint from which the program views the world. These discussions are not intended to imply that \acrshort{eden} would be better at its goal of providing effective disaster relief if the code was different. The impact of \acrshort{eden} as applied in disasters is something totally outside my experience and something I cannot make claims about. That I have found bugs that \textit{have} been fixed is not to imply that the \acrshort{eden} team \textit{should} have fixed them earlier.

   The goal of this project is not to offer improvements, but to imagine other ways of engaging with the technological ecosystem that \acrshort{eden} floats within. The hope is that such imaginings ("worldings" in \citet{Haraway2016-nc}'s language) will be generative for technologists and yield lessons for \acrlong{sts} scholars on how to apply situated knowledges to real tools. 
  
   \printnoidxglossaries
   \setlength{\parindent}{4em}
   \bibliography{situation}

\end{document}