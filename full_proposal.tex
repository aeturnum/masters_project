%
% ______          _           _    ______                                _ 
% | ___ \        (_)         | |   | ___ \                              | |
% | |_/ / __ ___  _  ___  ___| |_  | |_/ / __ ___  _ __   ___  ___  __ _| |
% |  __/ '__/ _ \| |/ _ \/ __| __| |  __/ '__/ _ \| '_ \ / _ \/ __|/ _` | |
% | |  | | | (_) | |  __/ (__| |_  | |  | | | (_) | |_) | (_) \__ \ (_| | |
% \_|  |_|  \___/| |\___|\___|\__| \_|  |_|  \___/| .__/ \___/|___/\__,_|_|
%               _/ |                              | |                      
%              |__/                               |_|                      
% -------------------------------------------------------------------------
% It is generally worthwhile, I think, to justfiy the use of LaTeX. It's obtuse 
% and it's difficult and it's unclear that it's any better than using Word for
% most projects. However, it has one advantage: it is generated from plain text 
% and this project is using Git, a tool that talks plain text. 
% 
% Using LaTeX will let me create a relaibly formatted document and also have an
% intelligable history of changes. It will also let me work on this document (and
% presumably my final paper which will also use LaTeX) in any environment. It's also
% nice to have a language with comments that won't render into the final paper for
% items like this.
% 
% Docx conversion is achieved through Adobe Acrobat DC. Pandoc was investigated, but it
% ignores much of the tex style and formatting


% Use APA module
% Based on template from https://www.overleaf.com/latex/templates/your-apa6-style-manuscript/kngbbqpypjcq
\documentclass[a4paper,man,natbib]{apa6}
% Imports natbib
% Ref Sheet: https://gking.harvard.edu/files/natnotes2.pdf

% Packages
\usepackage[english]{babel}
\usepackage[utf8x]{inputenc}
\usepackage{amsmath}
\usepackage{graphicx}
% for cite title
% \usepackage{biblatex}
% fancy quotes
\usepackage{epigraph}
% For formatting of potential bibliography
\usepackage{enumitem}
% html links
\usepackage{hyperref}
%\usepackage[colorinlistoftodos]{todonotes}

% \bibliography{full_proposal}

\title{Mapping EDEN Proposal}
\shorttitle{Final Masters Project Proposal}
\author{Daniel "Drex" Drexler}
\affiliation{Center for Science, Technology and Society at Drexel University}
\date{August 2019}
\begin{document}
   \maketitle
   % This document outline based on the suggested outline in the Center for STS Student Handbook:
   % https://drexel.edu/~/media/Files/sts/2015-graduate-student-handbook.ashx?la=en

   \section*{Overview}
   % Overview (0.5-1 page)
   % A summary of other sections of the proposal
   \subsection*{Description}
   Mapping EDEN is an experiment in wedding sociological critique with a craft-tradition of Computer Science seeking artifacts and ways of making and doing that suggest fresh approaches to multiple disciplines.
   % A one line description of the project
   \subsection*{Background}
   % Background / context for project / articulation of contemporary problem being addressed
   Much of the groundbreaking work in the social sciences can appear banal. When they articulated strong objectivity and situated knowledges \citet{Harding1992-od} and \citet{Haraway1993-kw} respectively were working to undo myths of scientific objectivity with the deeply mundane but somehow elusive truth that our human experience impacts human judgment and that scientists are also human.

   Ideally, the goal of Science, Technology and Society is to do work that is fully engaged with the practical details of human experience and effort and uses that engagement to develop theoretical understandings that reveal new points of contact and analyses. \citet{Sismondo2008-xz} describes this ideal as an “engaged program” that seeks to use pertinent topics in theoretically interesting ways. This project attempts to realize these qualities by simultaneously applying theoretical critiques to a situated and particular software tool and modifying that tool in conversation with those critiques. The end goal is a process-object that exists both as a fully functional piece of software and a record of the progress towards analyzing and creating that software from an existing base. I chose this particular form because it draws from both the traditions of software development and Science, Technology and Society scholarship but sets neither above the other. An engaged product must strive to be fully available to more than just the scholars of the social sciences.
   \subsection*{Merits}
   % 2-3 intellectual merits (informs research questions)
   Primarily, this project grew out of a sense of the disconnect between the making and doing of Science, Technology and Society and the making and doing of engineering. I was inspired by the work of feminist scientists (primarily but not exclusively \citet{Roy2018-ma} and \citet{Subramaniam2014-wg}) in working to learn more through their combination of the scientific method with feminist critique. My project is not one centered on feminist critique, but within my professional background of craft Computer Science \citep[see][]{Ensmenger2012-kz}. In \citet[p. 63]{Haraway2016-nc} Haraway says "a model is worked, and it does work." I want to work the models of critical analsis and software development and see what work I can do through working them.
   Another focus of this project will EDEN's view of the world. In Modest\_Witness@Second\_Millennium.FemaleMan\copyright\_Meets\_OncoMouse\texttrademark Haraway works to describe the world as seen through the advertisements of genetic technologies companies. In the same tradition \citet{Subramaniam2014-wg} shows how the eugenic roots of biometric tools influence the kinds of results they can discover. EDEN did not spring from nowhere and the traditions it draws on will bias the views it renders of the world. I will search for the standpoint \citet{Haraway1993-kw} EDEN views the world from within.
   \section*{Background}
   % Background	description	of	problem	and	context (1 page)
   Sociology, Media Studies and Science, Technology and Society have a long history of critiquing the artifacts and processes produced by technology companies \citep{Cheney-Lippold2018-lw,Dean2010-lk,Harraway1997-va}. 

   \section*{Intellectual Heritage}
   % Intellectual	Merits (1-2 pages)
   % 2 to 3 scholarly conversations your research contributes to
   \subsection*{Situated Knowledges}

   \epigraph{A model is worked, and it does work}{\textit{Donna Haraway \\ Staying with the Trouble: Making Kin in the Chthulucene p. 63}}

   Our judgments are influenced by our experiences. This basic human quality is at the center how we understand others and how we learn, among other things. Within the social sciences, scholars have been working to demonstrate that this quality exists both within people and within processes and tools. This project is most directly inspired by the vein of work descended from critical feminism in the eighties around scientific epistemologies. Specifically \citet{Haraway1988-nh} introduces the term "Situated Knowledges" to talk about the power of speaking 'from' a location, which allows others to speak on the same topic from another location without needing to directly contradict the first speaker. It is a practice that strives to go beyond the flaws of scientific epistemologies and speak about a better system. Haraway was responding to work from Sandra Harding, who later returned to the topic to outline what \citet{Harding1992-od} calls "Strong Objectivity." In conversation with the idea of situated knowledges, Harding critiques the traditional scientific practice of excising the background of the producers of scientific knowledge by pointing out that retaining how knowledge-finders are situated in the world adds to our understanding of their work. Out of the conversation between these two feminist thinkers has emerged a productive lineage of feminist thinkers who take the question of identifying how a 'knower' (either themselves or a third party) is situated in the world and how that situation has impacted the knowledge they produce. A modern vision of this analysis can be seen in \citet{Harraway1997-va}'s Modest\_Witness@Second\_Millennium.FemaleMan\copyright\_Meets\_OncoMouse\texttrademark, which breaks down the rhetoric and framing around genetics research fetishizes the particular element of the gene because it is useful to the industry and particular narratives. Haraway advocates a view of the gene that centers it in a wide web of biological mechanisms that interact and combine to create living creatures. Beyond individuals and rhetoric, the design and function of tools also encode a perspective and a view of the world. Some tools, like telescopes, do this in obvious ways. Others, like the tools used to analyze populations, contain intents that are less obvious. \citet{Subramaniam2014-wg} describes how the eugenic legacy of population level analysis makes some findings easier to see than others. How tools developed with the assumption that they would reveal differences between races are disappointingly easy to misuse, purposefully or accidentally, and create results that suggest racial difference. That much of the bias in a standpoint comes from what is hard to see or invisible with a particular tool and that this is only a flaw (for tools are meant to make select things more visible) when that tendency goes unacknowledged. That a tool contains a bias is desirable (what use is a telescope that does not magnify one area of the world?) and only becomes problematic when that bias is mistaken for the nature of the world.

   \subsection*{Software Studies}

   \epigraph{[S]haring produces its own kind of moral and technical order”}{\textit{Christopher Kelty \\ Two Bits: The Cultural Significance of Free Software p. 118}}

   Software is a tool that emerges out of a sea of tools. The methods of its construction, verification, distribution and maintenance are varied. This variance of practice arises at least in part from the lack of universally agreed upon standards for the goals of software or the appropriate metrics for success \citep{Ensmenger2012-kz}. A diversity of practice does not create a homologous blend of cultures and practices. Instead, each software community has its own set of values and goals in addition to wider ethics that other projects and cultures share. Software Studies, broadly, attempts to connect the these hyper-local values that emerge from culture and circumstance to the technical and deterministic qualities of a particular piece of software or a particular piece of code. Works in the field often focus on a particular software community or a set of communities with shared interests. \citet{Kelty2008-jm}'s work focuses on highlighting common social structures among particular software communities that both enable and emerge from the environment of open source software. There is also work that describes "code/space": places where the fixed rules of code and the fuzzy rules of social understanding combine to create hybrid spaces \citep{Bridle2018-nd,Kitchin2011-af}. \citet{Bivens2017-tc} represents our final influence as she compares how social qualities emerge with underlying technical qualities. Are social limitations matched by technical limitations? How do these two qualities co-create one another? 

   \section*{Design}
   % Study	Design (1-2 pages)
   This project will use sociological analysis to explore and highlight the standpoint of a piece of software. While I break down the particular qualities of the software package -  its goals, blind spots and distortions, I will also be looking at its source code to compare the technical system to the external interface. What aspects seem constrained by the technical tools used to structure the system? What qualities are unrelated to the technical details?

   \subsection*{Fieldsite}
   % Description and justification of data sources or fieldsites
   \epigraph{That virtual worlds are places means they can be fieldsites;}{\textit{Tom Boellstorff \\ Two Bits: The Cultural Significance of Free Software p. 107}}

   The EDEN software package, developed by the Sahana Foundation (a non-profit) and deployed to manage disaster response is the focus of this project. It is designed with a purpose and with the intent to save lives. It has good reason to simplify and focus perspectives of those using the tool and to problematize particular aspects of the situation while allowing others to remain innocuous. Its lack of commercial motivation will also likely simply the process of decomposing whom its perspectives benefit.

   The software's homepage can be found here: \url{https://sahanafoundation.org/}. A demo version of it running on the Sahana foundation can be found here: \url{http://demo.sahanafoundation.org/eden/}. Its source code can be found here: \url{https://github.com/sahana/eden}

   \subsection*{Methods}
   \subsubsection*{Phase One}
   The first phase of this project will be a wide survey of literature. Focusing on Science, Technology and Society works that analyze the standpoint and biases of other technical tools and moving into analysis of disaster 
   The aim of this project is to put academic work in conversation with practical tools as well as put academic criticism in conversation with work to  transform the same tools. This will draw on Science, Technology and Society's tradition of Making and Doing as well as traditional methods for analysis and critique.
   % Description of methods
   A PDF of this document can be found at \url{https://github.com/aeturnum/masters_project/blob/master/full_proposal.pdf}.

   The \LaTeX source file can be found at \url{https://github.com/aeturnum/masters_project/blob/master/full_proposal.tex}.
   
   \subsection*{Inspiration}

   \subsection*{Research Questions}
   What ways of knowing and doing are assisted by the design of EDEN?

   What epistemological traditions are centered in EDEN's methods for storing data? Whose needs are attended to first in this particular product?
   \subsection*{Schedule}
   % Schedule of work
   % Much thanks to: 
   % https://www.tablesgenerator.com/
   % https://en.wikibooks.org/wiki/LaTeX/Tables#Text_wrapping_in_tables
   % https://stackoverflow.com/questions/790932/how-to-wrap-text-in-latex-tables
   \begin{center}
   \begin{tabular}{|l|p{0.1\linewidth}|p{0.1\linewidth}|p{0.7\linewidth}|}
   \hline
   Quarter & Quality & Time Allocation & Goal                         \\ \hline
   Fall    & Reading & 80\%            & 800 pages read with notes    \\ \cline{2-4} 
           & Writing & 10\%            & Notes and planning documents \\ \cline{2-4} 
           & Coding  & 10\%            & Notes on project structure   \\ \hline 
   Winter  & Reading & 40\%            & Final selection of bibliography with additions based on notes and readings.   \\ \cline{2-4} 
           & Writing & 30\%            & Outline for final paper with ~20\% of content finished \\ \cline{2-4} 
           & Coding  & 30\%            & Rough modifications completed, but in need of polish  \\ \hline
   Fall    & Reading & 10\%            & Unexpected additions to literature and gathering specific quotes from previously completed elements.   \\ \cline{2-4} 
           & Writing & 40\%            & Final paper \\ \cline{2-4}
           & Coding  & 40\%            & Full branch of EDEN project with documentation and demonstration server. \\ \hline
   \end{tabular}
   \end{center}
   \section*{Impacts}
   % Broader	Impacts (.5-.75 page)
   \subsection*{Artifacts}
   The project will produce three artifacts: a paper, a piece of software and a git source-tracking repository.
   \subsubsection*{The Paper}
   The paper will provide formal analysis of the original form of the software being considered. It will engage with social science literatures from traditions such as feminist critiques of technologies, critiques of algorithmic technologies, critiques of (often but not exclusively scientific) epistemological processes and software studies. Its discussion and conclusion will describe the particular view the software renders of the world and also argue how alterations to that piece of software might change its purpose, its views or its impact.
   \subsubsection*{New EDEN}
   The software artifact (with the tongue-in-cheek name of New EDEN) will be a modified version of the selected software whose modifications will be driven by the suggested alterations in the paper. These two software products (modified and unmodified) can serve to contextualize the paper and bound its theoretical claims to what is pragmatically possible for a software developer.
   \subsubsection*{The Git Repository}
   The two primary artifacts and all the other ephemera (including this proposal) involved in producing them will be archived in a git repository. Git will allow the process to be recorded and observed both in real time and in hindsight, giving the project a shape and an arc and a character outside of its final results. 

   The reference Git repository for this project is available at \url{https://github.com/aeturnum/masters_project}.
   % Output (e.g., conference presentations, results to informants, white paper, thesis, etc.)
   \subsection*{Goals}
   % How it will help us understand X (and Y)
   \clearpage
   \section{Suggested Bibliography}

   % Much thanks to https://tex.stackexchange.com/questions/99344/indent-text-after-line-1
   \begin{itemize}[label={},itemindent=-2em,leftmargin=2em]
   \item Ananny, M., \& Crawford, K. (2018). Seeing without knowing: Limitations of the transparency ideal and its application to algorithmic accountability. New Media \& Society, 20(3), 973–989.

   \item Barocas, S., \& Boyd, D. (2017). Engaging the Ethics of Data Science in Practice. Communications of the ACM, 60(11), 23–25.

   \item Benjamin, R. (2014). Race for cures: Rethinking the racial logics of “trust”in biomedicine. Sociology Compass, 8(6), 755–769.

   \item Boellstorff, T. (2016). For Whom the Ontology Turns: Theorizing the Digital Real. Current Anthropology, 57(4), 387–407.

   \item Bowker, G. C., \& Star, S. L. (2000). Sorting Things Out: Classification and Its Consequences. MIT Press.

   \item Cheney-Lippold, J. (2018). We Are Data: Algorithms and the Making of Our Digital Selves. NYU Press.

   \item Cooke, N. A., Sweeney, M. E., \& Noble, S. U. (2016). Social Justice as Topic and Tool: An Attempt to Transform an LIS Curriculum and Culture. The Library Quarterly, 86(1), 107–124.

   \item Crawford, K., \& Finn, M. (2015). The limits of crisis data: analytical and ethical challenges of using social and mobile data to understand disasters. GeoJournal, 80(4), 491–502.

   \item Deleuze, G. (1992). Postscript on the Societies of Control. October, 59, 3–7.

   \item Dourish, P., Anderson, K., \& Nafus, D. (2007). Cultural mobilities: Diversity and agency in urban computing. IFIP Conference on Human-Computer. Retrieved from https://link.springer.com/chapter/10.1007/978-3-540-74800-7\_8

   \item Eckhouse, L., Lum, K., Conti-Cook, C., \& Ciccolini, J. (2019). Layers of Bias: A Unified Approach for Understanding Problems With Risk Assessment. Criminal Justice and Behavior, 46(2), 185–209.

   \item Elish, M. C., \& Boyd, D. (2018). Situating methods in the magic of Big Data and AI. Communication Monographs, Vol. 85, pp. 57–80. https://doi.org/10.1080/03637751.2017.1375130

   \item Eubanks, V. (2018). Automating Inequality: How High-Tech Tools Profile, Police, and Punish the Poor. St. Martin’s Press.

   \item Gabriella Coleman, E. (2012). Coding Freedom: The Ethics and Aesthetics of Hacking (pp. 69–70). Princeton University Press.

   \item Gangadharan, S. P. (2017). The downside of digital inclusion: Expectations and experiences of privacy and surveillance among marginal Internet users. New Media \& Society, 19(4), 597–615.

   \item Gillespie, T. (2018). Custodians of the Internet: Platforms, Content Moderation, and the Hidden Decisions That Shape Social Media. Yale University Press.

   \item Haraway, D. (1988). Situated Knowledges: The Science Question in Feminism and the Privilege of Partial Perspective. Feminist Studies: FS, 14(3), 575–599.

   \item Haraway, D. J. (2016). Staying with the Trouble: Making Kin in the Chthulucene. Duke University Press.

   \item Harding, S. (1992). Rethinking standpoint epistemology: What is“ strong objectivity?” The Centennial Review, 36(3), 437–470.

   \item Harraway, D. (1997). Modest\_Witness@Second\_Millennium.FemaleMan\copyright\_Meets\_OncoMouse\texttrademark. Routledge New York.

   \item Howard, P. N., Anderson, K., \& Busch, L. (2009). Sizing up information societies: Toward a better metric for the cultures of ICT adoption. Information. An International Interdisciplinary Journal. Retrieved from https://www.tandfonline.com/doi/abs/10.1080/01972240902848948

   \item Kelty, C. M. (2008). Two Bits: The Cultural Significance of Free Software. Duke University Press.

   \item Kitchin, R., \& Dodge, M. (2011). Code/space: Software and Everyday Life. MIT Press.

   \item Knowles, S. G. (2014). Learning from Disaster?: The History of Technology and the Future of Disaster Research. Technology and Culture, 55(4), 773–784.

   \item Latonero, M., \& Kift, P. (2018). On Digital Passages and Borders: Refugees and the New Infrastructure for Movement and Control. Social Media + Society, 4(1), 2056305118764432.

   \item Mackenzie, A. (2010). Wirelessness: Radical Empiricism in Network Cultures. MIT Press.

   \item Nafus, D. (2014). Stuck data, dead data, and disloyal data: the stops and starts in making numbers into social practices. Distinktion: Scandinavian Journal of Social Theory. Retrieved from https://www.tandfonline.com/doi/abs/10.1080/1600910X.2014.920266

   \item Noble, S. U. (2017). Google and the misinformed public. The Chronicle of Higher Education.

   \item Ponte, A. (2014). The House of Light and Entropy. Architectural Association.

   \item Roy, D. (2018). Molecular Feminisms: Biology, Becomings, and Life in the Lab. University of Washington Press.

   \item Star, S. L. (ed ). (1995). The cultures of computing. Retrieved from http://agris.fao.org/agris-search/search.do?recordID=XF2015029404

   \item Wolske, M., Williams, N. S., Noble, S., Johnson, E. O., \& Duple, R. Y. (2010). Effective ICT use for Social Inclusion. Retrieved from http://www.ideals.illinois.edu/handle/2142/14952

   \end{itemize}

   \setlength{\parindent}{4em}
   \bibliography{full_proposal}

\end{document}