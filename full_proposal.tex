%
% ______          _           _    ______                                _ 
% | ___ \        (_)         | |   | ___ \                              | |
% | |_/ / __ ___  _  ___  ___| |_  | |_/ / __ ___  _ __   ___  ___  __ _| |
% |  __/ '__/ _ \| |/ _ \/ __| __| |  __/ '__/ _ \| '_ \ / _ \/ __|/ _` | |
% | |  | | | (_) | |  __/ (__| |_  | |  | | | (_) | |_) | (_) \__ \ (_| | |
% \_|  |_|  \___/| |\___|\___|\__| \_|  |_|  \___/| .__/ \___/|___/\__,_|_|
%               _/ |                              | |                      
%              |__/                               |_|                      
% -------------------------------------------------------------------------
% It is generally worthwhile, I think, to justfiy the use of LaTeX. It's obtuse 
% and it's difficult and it's unclear that it's any better than using Word for
% most projects. However, it has one advantage: it is generated from plain text 
% and this project is using Git, a tool that talks plain text. 
% 
% Using LaTeX will let me create a relaibly formatted document and also have an
% intelligable history of changes. It will also let me work on this document (and
% presumably my final paper which will also use LaTeX) in any environment. It's also
% nice to have a language with comments that won't render into the final paper for
% items like this.
% 
% Docx conversion is achieved through Adobe Acrobat DC. Pandoc was investigated, but it
% ignores much of the tex style and formatting


% Use APA module
% Based on template from https://www.overleaf.com/latex/templates/your-apa6-style-manuscript/kngbbqpypjcq
\documentclass[a4paper,man,natbib]{apa6}
% Imports natbib
% Ref Sheet: https://gking.harvard.edu/files/natnotes2.pdf

% Packages
\usepackage[english]{babel}
\usepackage[utf8x]{inputenc}
\usepackage{amsmath}
\usepackage{graphicx}
% set single spacing
\usepackage{setspace}
% glossary
\usepackage[acronym]{glossaries}
% fancy quotes
\usepackage{epigraph}
% For formatting of potential bibliography
\usepackage{enumitem}
% html links
\usepackage{hyperref}
%\usepackage[colorinlistoftodos]{todonotes}

\makenoidxglossaries

\singlespacing

\loadglsentries{glossary}

\title{Mapping EDEN: Revealing the territory created by society and software}
\shorttitle{Mapping EDEN}
\author{Daniel "Drex" Drexler}
\affiliation{Center for Science, Technology and Society at Drexel University}
\date{October 2019}
\begin{document}
   \maketitle
   % This document outline based on the suggested outline in the Center for STS Student Handbook:
   % https://drexel.edu/~/media/Files/sts/2015-graduate-student-handbook.ashx?la=en
   %
   % The October revision to this document is based on a Thesis proposal that my Graduate Advisor
   % Vincent Duclos suggested as a good example: The one written by Jim Gall. Though his thesis 
   % porposal is not included here, an outline is reproduced below:
   % 
   % Overview: sucscinct, opinionated description of the project and the problem it is looking to 
   %    address. Ending with a few sentances on literatures engaged and research questions
   %
   % Conceptual Framework: Citation-heavy section citing the specific theories you expect to be 
   %    using in your work, continue to describe the situation as it is useful 
   %
   % Methods & Timeline: How you will do the work and the time in which you suggest to do it.
   %
   % Perparation of Researcher: Brief explanation about why you would be good at this and any previous
   %    experiences that are relevant to this work
   %
   % Preliminary List of Sources: Sources that are expected to be used in the project including all 
   %    sources cited in the proposal.
   % 
   \section*{Overview}
   % Overview (0.5-1 page)
   % A summary of other sections of the proposal
   \subsection*{Description}
   Mapping EDEN is an experiment in wedding sociological critique with a craft-tradition of Computer Science seeking artifacts and ways of making and doing that suggest fresh approaches to multiple disciplines. It comes from a sense that we cannot sufficiently articulate why software has such an substantial impact on our world and why our responses to software are so insufficient in the face of the social footprint of the impact of software. Software is made by people who live and breath in the same world that we live in. We must struggle to find the aspect within software development that makes it so disruptive to our social worlds. Perhaps this project can help.
   % A one line description of the project
   \subsection*{Background}
   % Background / context for project / articulation of contemporary problem being addressed
   This project is inspired by a personal feeling that there is a wide gap between the practical realities of creating software and the impact of software on our shared social world. 
   \subsection*{Merits}
   % 2-3 intellectual merits (informs research questions)
   This project begins from a place inspired by feminist thinkings' Situated Knowledges and proceeds from there. It hopes to highlight the perspectives contained within a particular technical tool, the qualities of those perspectives, who those perspectives are helpful and hurtful to. It also hopes to closely examine the quality of the tool and what things emerge particularly easily from that tool as well as what is unusually well hidden.

   It will show how the particular source code of the \acrshort{eden} project co-creates its shadow in social space. How the concerns of the technical and the constraints of the tools that are used to create the tool drive the qualities that emerge from the techne ecosystem. 
   \pagebreak
   \section*{Conceptual Framework}
   % Background	description	of	problem	and	context (1 page)
   % Intellectual	Merits (1-2 pages)
   % 2 to 3 scholarly conversations your research contributes to
   \subsection*{Situated Knowledges}

   \epigraph{A model is worked, and it does work}{\textit{Donna Haraway \\ Staying with the Trouble: Making Kin in the Chthulucene p. 63}}

   Our judgments are influenced by our experiences. This basic human quality is at the center how we understand others and how we learn, among other things. Within the social sciences, scholars have been working to demonstrate that this quality exists both within people and within processes and tools. This project is most directly inspired by the vein of work descended from critical feminism in the eighties around scientific epistemologies. Specifically \citet{Haraway1988-nh} introduces the term "Situated Knowledges" to talk about the power of speaking 'from' a location, which allows others to speak on the same topic from another location without needing to directly contradict the first speaker. It is a practice that strives to go beyond the flaws of scientific epistemologies and speak about a better system. Haraway was responding to work from Sandra Harding, who later returned to the topic to outline what \citet{Harding1992-od} calls "Strong Objectivity." In conversation with the idea of situated knowledges, Harding critiques the traditional scientific practice of excising the background of the producers of scientific knowledge by pointing out that retaining how knowledge-finders are situated in the world adds to our understanding of their work. Out of the conversation between these two feminist thinkers has emerged a productive lineage of feminist thinkers who take the question of identifying how a 'knower' (either themselves or a third party) is situated in the world and how that situation has impacted the knowledge they produce. A modern vision of this analysis can be seen in \citet{Harraway1997-va}'s Modest\_Witness@Second\_Millennium.FemaleMan\copyright\_Meets\_OncoMouse\texttrademark, which breaks down the rhetoric and framing around genetics research fetishizes the particular element of the gene because it is useful to the industry and particular narratives. Haraway advocates a view of the gene that centers it in a wide web of biological mechanisms that interact and combine to create living creatures. Beyond individuals and rhetoric, the design and function of tools also encode a perspective and a view of the world. Some tools, like telescopes, do this in obvious ways. Others, like the tools used to analyze populations, contain intents that are less obvious. \citet{Subramaniam2014-wg} describes how the eugenic legacy of population level analysis makes some findings easier to see than others. How tools developed with the assumption that they would reveal differences between races are disappointingly easy to misuse, purposefully or accidentally, and create results that suggest racial difference. That much of the bias in a standpoint comes from what is hard to see or invisible with a particular tool and that this is only a flaw (for tools are meant to make select things more visible) when that tendency goes unacknowledged. That a tool contains a bias is desirable (what use is a telescope that does not magnify one area of the world?) and only becomes problematic when that bias is mistaken for the nature of the world.

   \subsection*{Software Studies}

   \epigraph{[S]haring produces its own kind of moral and technical order”}{\textit{Christopher Kelty \\ Two Bits: The Cultural Significance of Free Software p. 118}}

   Software is a tool that emerges out of a sea of tools. The methods of its construction, verification, distribution and maintenance are varied. This variance of practice arises at least in part from the lack of universally agreed upon standards for the goals of software or the appropriate metrics for success \citep{Ensmenger2012-kz}. A diversity of practice does not create a homologous blend of cultures and practices. Instead, each software community has its own set of values and goals in addition to wider ethics that other projects and cultures share. Software Studies, broadly, attempts to connect the these hyper-local values that emerge from culture and circumstance to the technical and deterministic qualities of a particular piece of software or a particular piece of code. Works in the field often focus on a particular software community or a set of communities with shared interests. \citet{Kelty2008-jm}'s work focuses on highlighting common social structures among particular software communities that both enable and emerge from the environment of open source software. There is also work that describes "code/space": places where the fixed rules of code and the fuzzy rules of social understanding combine to create hybrid spaces \citep{Bridle2018-nd,Kitchin2011-af}. \citet{Bivens2017-tc} represents our final influence as she compares how social qualities emerge with underlying technical qualities. Are social limitations matched by technical limitations? How do these two qualities co-create one another? 

   \section*{Design}
   % Study	Design (1-2 pages)
   This project will use sociological analysis to explore and highlight the standpoint of a piece of software. While I break down the particular qualities of the software package -  its goals, blind spots and distortions, I will also be looking at its source code to compare the technical system to the external interface. What aspects seem constrained by the technical tools used to structure the system? What qualities are unrelated to the technical details?

   \subsection*{Fieldsite}
   % Description and justification of data sources or fieldsites
   \epigraph{That virtual worlds are places means they can be fieldsites;}{\textit{Tom Boellstorff \\ Two Bits: The Cultural Significance of Free Software p. 107}}

   The \acrfull{EDEN} software package, developed by the Sahana Foundation (a non-profit) and deployed to manage disaster response is the focus of this project. It was initially developed and deployed by an alliance of companies in the Sri Lankan \acrfull{ict} sector in response to the Indian Ocean Earthquake \& Tsunami. In 2009 development was transferred to the Sahana Foundation, an American non-profit. The Foundation lists its mission as "Saving lives through information management solutions." They describe their software as part of a wave of change in how humanitarian responses can use \acrshort{ict} to improve the quality of their response. They situate themselves in the middle of other \acrfull{foss} \acrshort{ict} projects such as \Gls{OpenStreetMap} which focus on community engagement and incorporating user data into digital platforms. \acrshort{eden} has been used in response to a number of disasters in the developing world as well as ongoing supply management programs in the developed world \citep{Sahana_Foundation_undated-hl}.

   The software's homepage can be found here: \url{https://sahanafoundation.org/}. \linebreak 
   A demo version of it running on the Sahana Foundation can be found here: \url{http://demo.sahanafoundation.org/\acrshort{eden}/}. \linebreak 
   \acrshort{eden}'s source code can be found here: \url{https://github.com/sahana/\acrshort{eden}} \linebreak 

   EDEN's identity as a piece of software is enmeshed in an \acrshort{foss} ecosystem. It is written in \Gls{python}, an open source language and uses the \Gls{python} \gls{web framework} web2py as the foundation its functionality is built on. Its functionality is broken up into many different web modules, which are organized around nodes of shared functionality that allow the individual modules to communicate and avoid duplicating functionality. \acrshort{eden} generally uses other \acrshort{foss} tools wherever possible: both of its database options (\Gls{MySQL} and \Gls{PostgreSQL}) are \acrshort{foss} databases that are also available for free. This use of \acrlong{foss} technologies helps ensure that users pay minimal fees to use \acrshort{eden}, as it and all of its resources are available free of charge. However, because \acrshort{foss} products are developed with many audiences in mind, their approach to problems is general and those projects will make governance decisions that may force you to leave them.

   \subsection*{The Project}
   \subsubsection*{Phase One}
   The first phase of this project will be a wide survey of literature. Focusing on Science, Technology and Society works that analyze the standpoint and biases of other technical tools and moving into analysis of disaster 
   The aim of this project is to put academic work in conversation with practical tools as well as put academic criticism in conversation with work to transform the same tools. This will build a set of analytical frames that will be essential for the second phase of the project.  
   \subsubsection*{Phase Two}
   The second phase will be be a co-constitutive process between social critique and technical analysis. I'll begin by examining a standpoint that \acrshort{eden} adopts, writing about what that standpoint intends to do and what side effects it might have. Those outcomes will be compared to and put in conversation with, the technical systems used to realize the standpoint. Over time, a paper that holds the technical choices, social goals and impacts, blind spots and clarifying qualities of \acrshort{eden} will emerge from this work. Some standpoints will prove uninteresting and some will be difficult to match with sociological analysis. 

   As I find and characterize these points of social-technical tension, they will guide modifications to the \acrshort{eden} software package itself. These modifications might attempt to address weaknesses in the particular standpoint the software addresses. They might be used to 'queer' the original intent of the \acrshort{eden} module being modified. Such 'queered' modules might highlight how narrowly \acrshort{eden} sees the world, how much about lived experience is irrelevant to \acrshort{eden}, or highlight conditions that would make it impossible to continue to use \acrshort{eden} and how unrelated they are to the circumstances that would lead people to use \acrshort{eden}. Some number of these modifications will be made to the \acrshort{eden} source code, but some more may highlight the technically unavailable resources that might enable software and society to interact differently.

   % Description of methods
   \subsection*{Inspiration}
   I don't know of anything else quite like this project.

   The praxis of this project is heavily inspired by Rena Bivens' stellar work that smashes together the technical realities of a product with the social understanding of it to problematize it. Its perspective is strongly influenced by Donna Haraway's attention to the specific qualities  that structure spaces where sociological work happens. It is hoping to recall Banu Subramaniam's powerful tools for thinking through how and why a particular thing impacts the world in a particular way.

   \subsection*{Research Questions}
   What ways of knowing and doing are assisted with the design of \acrshort{eden}?

   What epistemological traditions are attended to by \acrshort{eden}'s methods for storing data? 

   How do the social qualities of the social spaces within \acrshort{eden} emerge from within its data technologies?

   What qualities of \acrshort{eden} are co-creations between social space and the technical tools used to constitute the software?
   \subsection*{Schedule}
   % Schedule of work
   % Much thanks to: 
   % https://www.tablesgenerator.com/
   % https://en.wikibooks.org/wiki/LaTeX/Tables#Text_wrapping_in_tables
   % https://stackoverflow.com/questions/790932/how-to-wrap-text-in-latex-tables
   \begin{center}
   \begin{tabular}{|l|l|p{0.1\linewidth}|p{0.1\linewidth}|p{0.6\linewidth}|}
   \hline
   Phase & Quarter & Quality & Time Allocation & Goal                         \\ \hline
   One   & Fall    & Reading & 80\%            & 800 pages read with notes    \\ \cline{3-5} 
         &         & Writing & 10\%            & Notes and planning documents \\ \cline{3-5} 
         &         & Coding  & 10\%            & Notes on \acrshort{eden} program structure   \\ \hline 
   Two   & Winter  & Reading & 40\%            & Final selection of bibliography with additions based on notes and readings.   \\ \cline{3-5} 
         &         & Writing & 30\%            & Outline for final paper with ~20\% of content finished \\ \cline{3-5} 
         &         & Coding  & 30\%            & Rough modifications completed, but in need of polish  \\ \cline{2-5}
         & Spring  & Reading & 10\%            & Unexpected additions to literature and gathering specific quotes from previously completed elements.   \\ \cline{3-5} 
         &         & Writing & 40\%            & Final paper \\ \cline{3-5}
         &         & Coding  & 40\%            & Full branch of \acrshort{eden} project with documentation and demonstration server. \\ \hline
   \end{tabular}
   \end{center}
   \newpage
   \section*{Outcomes}
   The project will produce three artifacts: a paper, a piece of software and a git source-tracking repository.

   The reference Git repository for this project is available at \url{https://github.com/aeturnum/masters_project}.
   % Broader	Impacts (.5-.75 page)
   % Output (e.g., conference presentations, results to informants, white paper, thesis, etc.)
   \subsubsection*{The Git Repository}
   The two primary artifacts and all the other ephemera (including this proposal) involved in producing them will be archived in a git repository. Git will allow the process to be recorded and observed both in real time and in hindsight, giving the project a shape and an arc and a character outside of its final results. 

   A PDF of this document can be found at \url{https://github.com/aeturnum/masters_project/blob/master/full_proposal.pdf}.

   The \LaTeX source file can be found at \url{https://github.com/aeturnum/masters_project/blob/master/full_proposal.tex}.

   \subsubsection*{The Paper}
   The paper will provide formal analysis of the original form of the software being considered. It will engage with social science literatures from traditions such as feminist critiques of technologies, critiques of algorithmic technologies, critiques of (often but not exclusively scientific) epistemological processes and software studies. Its discussion and conclusion will describe the particular view the software renders of the world and also argue how alterations to that piece of software might change its purpose, its views or its impact.

   \subsubsection*{New EDEN}
   The software artifact (with the tongue-in-cheek name of New \acrshort{eden}) will be a modified version of the selected software whose modifications will be driven by the suggested alterations in the paper. These two software products (modified and unmodified) can serve to contextualize the paper and bound its theoretical claims to what is pragmatically possible for a software developer.

   \subsection*{Goals}
   % How it will help us understand X (and Y)
   This project will create a new space where the intersections between technology and society are hyper-visible. It will help to highlight the qualities of technical objects that give rise to their socially visible qualities. It will hopefully inspire others to further explore the power of modifying software as a way of finding and highlighting ways in which that software intrudes into our social lives. It may even provoke more engaged and practical work in software studies that struggle the daily challenges of writing software.

   \newpage
   \printnoidxglossaries

   \newpage
   \setlength{\parindent}{4em}
   \bibliography{full_proposal}

\end{document}