%
% ______          _           _    ______                                _ 
% | ___ \        (_)         | |   | ___ \                              | |
% | |_/ / __ ___  _  ___  ___| |_  | |_/ / __ ___  _ __   ___  ___  __ _| |
% |  __/ '__/ _ \| |/ _ \/ __| __| |  __/ '__/ _ \| '_ \ / _ \/ __|/ _` | |
% | |  | | | (_) | |  __/ (__| |_  | |  | | | (_) | |_) | (_) \__ \ (_| | |
% \_|  |_|  \___/| |\___|\___|\__| \_|  |_|  \___/| .__/ \___/|___/\__,_|_|
%               _/ |                              | |                      
%              |__/                               |_|                      
% -------------------------------------------------------------------------
% It is generally worthwhile, I think, to justfiy the use of LaTeX. It's obtuse 
% and it's difficult and it's unclear that it's any better than using Word for
% most projects. However, it has one advantage: it is generated from plain text 
% and this project is using Git, a tool that talks plain text. 
% 
% Using LaTeX will let me create a relaibly formatted document and also have an
% intelligable history of changes. It will also let me work on this document (and
% presumably my final paper which will also use LaTeX) in any environment. It's also
% nice to have a language with comments that won't render into the final paper for
% items like this.
% 
% Docx conversion is achieved through Adobe Acrobat DC. Pandoc was investigated, but it
% ignores much of the tex style and formatting


% Use APA module
% Based on template from https://www.overleaf.com/latex/templates/your-apa6-style-manuscript/kngbbqpypjcq
\documentclass[a4paper,man,natbib]{apa6}
% Imports natbib
% Ref Sheet: https://gking.harvard.edu/files/natnotes2.pdf

% Packages
\usepackage[english]{babel}
\usepackage[utf8x]{inputenc}
\usepackage{amsmath}
\usepackage{graphicx}
% set single spacing
\usepackage{setspace}
% glossary
\usepackage[acronym]{glossaries}
% fancy quotes
\usepackage{epigraph, varwidth}
% For formatting of potential bibliography
\usepackage{enumitem}
% html links
\usepackage{hyperref}
%\usepackage[colorinlistoftodos]{todonotes}

\renewcommand{\epigraphsize}{\small}
\setlength{\epigraphwidth}{0.6\textwidth}
\renewcommand{\textflush}{flushright}
\renewcommand{\sourceflush}{flushright}
% A useful addition
\newcommand{\epitextfont}{\itshape}
\newcommand{\episourcefont}{\scshape}

\makeatletter
\newsavebox{\epi@textbox}
\newsavebox{\epi@sourcebox}
\newlength\epi@finalwidth
\renewcommand{\epigraph}[2]{%
  \vspace{\beforeepigraphskip}
  {\epigraphsize\begin{\epigraphflush}
   \epi@finalwidth=\z@
   \sbox\epi@textbox{%
     \varwidth{\epigraphwidth}
     \begin{\textflush}\epitextfont#1\end{\textflush}
     \endvarwidth
   }%
   \epi@finalwidth=\wd\epi@textbox
   \sbox\epi@sourcebox{%
     \varwidth{\epigraphwidth}
     \begin{\sourceflush}\episourcefont#2\end{\sourceflush}%
     \endvarwidth
   }%
   \ifdim\wd\epi@sourcebox>\epi@finalwidth 
     \epi@finalwidth=\wd\epi@sourcebox
   \fi
   \leavevmode\vbox{
     \hb@xt@\epi@finalwidth{\hfil\box\epi@textbox}
     \vskip1.75ex
     \hrule height \epigraphrule
     \vskip.75ex
     \hb@xt@\epi@finalwidth{\hfil\box\epi@sourcebox}
   }%
   \end{\epigraphflush}
   \vspace{\afterepigraphskip}}}
\makeatother

\makenoidxglossaries

\singlespacing

\loadglsentries{glossary}

\title{Mapping EDEN: Revealing the territory created by society and software}
\shorttitle{Mapping EDEN}
\author{Daniel "Drex" Drexler}
\affiliation{Center for Science, Technology and Society at Drexel University}
\date{October 2019}
\begin{document}
   \maketitle
   % This document outline based on the suggested outline in the Center for STS Student Handbook:
   % https://drexel.edu/~/media/Files/sts/2015-graduate-student-handbook.ashx?la=en
   %
   % The October revision to this document is based on a Thesis proposal that my Graduate Advisor
   % Vincent Duclos suggested as a good example: The one written by Jim Gall. Though his thesis 
   % porposal is not included here, an outline is reproduced below:
   % 
   % Overview: sucscinct, opinionated description of the project and the problem it is looking to 
   %    address. Ending with a few sentances on literatures engaged and research questions
   %
   % Conceptual Framework: Citation-heavy section citing the specific theories you expect to be 
   %    using in your work, continue to describe the situation as it is useful 
   %
   % Methods & Timeline: How you will do the work and the time in which you suggest to do it.
   %
   % Perparation of Researcher: Brief explanation about why you would be good at this and any previous
   %    experiences that are relevant to this work
   %
   % Preliminary List of Sources: Sources that are expected to be used in the project including all 
   %    sources cited in the proposal.
   % 
   \section*{Overview}
   % Overview (0.5-1 page)
   % A summary of other sections of the proposal
   \epigraph{[T]he "writing of technology" is by by no means universal; the opaque and stubborn places do not lie simply beneath technology, but are wrapped around and in it}{\textit{Adrian Mackenzie \\ Cutting Code: Software and Sociality p. 181}}

   This project will examine the \acrfull{eden} software package, developed by the Sahana Foundation (a non-profit) and deployed to manage disaster response. The Foundation lists its mission as "Saving lives through information management solutions." They see their software as part of a wave of change in how humanitarian responses can use \acrshort{ict} to improve the quality of their response. The Sahana foundation would like to be seen as peers to other \acrfull{foss} \acrshort{ict} projects such as \Gls{OpenStreetMap} which focus on "community engagement" and incorporating user data into digital platforms \citep{Sahana_Foundation_undated-hl}. \acrshort{eden} is also part of the \acrfull{ict4d} movement - seeking to turn technology to humanitarian ends. 

   Both \acrshort{ict4d} and "community engagement" applications run the risk of benefiting users only in ways that also can benefit the makers of the technology. Grappling with these pitfalls will be essential to spreading the bounties of technology widely.
   % They can also inadvertently foist western ideas about appropriate action and response onto groups that have no other tool to solve their particular problem.

   This project will look at \acrshort{eden} from both the outside and the inside. It will examine the ways in which \acrshort{eden} relates to its users, its referents and to the digital services it uses. What is easy to see from \acrshort{eden}, what is difficult? What perspectives does it contain?
   \medskip
   Generative encounters are as important as critical analysis. Can the standpoints encoded in \acrshort{eden} be changed? How are they driven by the technical constraints of the system? How does working with the code and on the code change how the code can be understood? I'll investigate by engaging with the perspectives I've discovered and trying to modify \acrshort{eden} with them in mind.
   \medskip
   This project proceeds from a Haraway-driven understanding of standpoint theory into a software-studies informed engagement with an a particular piece of software made for a particular purpose. It draws inspiration from the making and doing traditions of Science, Technology \& Society and the craft tradition of software engineering.
   \par
   \begin{flushleft}
   I am asking four questions:
   \end{flushleft}
   \begin{enumerate}
     \item What standpoints can be find within \acrshort{eden} by examining how it renders the world?

     \item How are these standpoints influenced by \acrshort{eden}'s constitutive technologies?

     \item Can these standpoints be changed or "queer"d through changing the code of \acrshort{eden}?

     \item Is the cycle of sociological analysis followed by technical modification useful and generative?
   \end{enumerate}

   
   % \pagebreak
   \section*{Conceptual Framework}
   % Background	description	of	problem	and	context (1 page)
   % Intellectual	Merits (1-2 pages)
   % 2 to 3 scholarly conversations your research contributes to
   \epigraph{A model is worked, and it does work}{\textit{Donna Haraway \\ Staying with the Trouble: Making Kin in the Chthulucene p. 63}}

   In his famous work on Pasteur, \citet{Latour1983-bs} describes how Pasteur's work was as much about changing the world to be more lab-like as it was about the products of the lab. Science appears magic because it distracts us from the work we do to make it possible to use its products. This focus on the perspective of engagement and knowing is what \citet{Haraway1988-nh} calls "Situated Knowledges." The term evokes the idea of speaking 'from' a location, with a particular view of the world. Others may speak about the same thing from other places and everyone may correctly describe their view. Sandra \citet{Harding1992-od}, another feminist epistemological thinker, later returned scientific epistemology to define "Strong Objectivity." Harding critiques returns to the traditional scientific epistemological practice Latour described of deemphasizing the social changes needed to use science. She points out that hiding this information leaves us knowing less. That a stronger practice would be to engage directly with the social changes we need to use science.
   \medskip
   Further works have expanded how knowing can be situated or encoded into social-material constructs. \citet{Harraway1997-va} returns to the subject to illustrate how the construct of "the gene" is encodes views from the biotech industry. Like Pasteur, the industry has constructed "the gene," as something that can only be seen through their expensive methods. Where Haraway was writing about an emerging technology, \citet{Subramaniam2014-wg} writes about the forgotten perspectives of tools. She describes how the eugenic legacy of population level analysis tools makes those tools better at seeing certain kinds of differences. The danger in a tool whose biases are hidden are that we miss its finger on the scale. We build tools to show us the world from a particular perspective, but we must interrogate for what purposes that perspective was shaped.
   \bigskip

   % \epigraph{[S]haring produces its own kind of moral and technical order”}{\textit{Christopher Kelty \\ Two Bits: The Cultural Significance of Free Software p. 118}}

   \citet{Mackenzie2006-hb} describes software not as a tool, but as a process-tool, whose current state is always inescapably intertwined with socially understood future expectations about its purposes and needs. The sociality of software, as Mackenzie calls it, reflects that software makers are unusually aware of the encoding of opinions in tools. The field has always been home to a huge diversities of practices all of which are justified as being, in some way, the 'right' way to make software \citep{Ensmenger2012-kz}. A diversity of practice does not create a homologous blend of cultures and practices. Instead, each software community has its own set of values and goals in addition to wider ethics that other projects and cultures share. Scholars have engaged at different levels of scale to understand the impacts of software. \citet{Kelty2008-jm} looks at the qualities of the communities around open source projects. \citet{Bivens2017-tc} describes how the demands of Facebook's tools and the changing social understanding around gender have constrained how they represent gender in their system. \citep{Kitchin2011-af} describe how software and social convention have begun to co-determine the rules of how places work, an idea they call "code/space." Software is not a thing that is ever actually finished. It is a continuing process which, through its periodic production of artifacts, is co-constitutive of the space(s) it manifests within. To understand how it changes the things it interacts with requires both examining the representational choices made by the software and the technical constraints of the systems it relies on.

   The design of the project is heavily inspired by Science, Technology \& Society practices of making \& doing. It also draws on \citet{Bivens2017-tc} work on the material qualities of software, \citet{Mackenzie2006-hb}'s Sociality and the womens studies concept of feminist epistemologies.

   \section*{Project Structure}
   % Study	Design (1-2 pages)
   
   % Description and justification of data sources or fieldsites
   \epigraph{That virtual worlds are places means they can be fieldsites;}{\textit{Tom Boellstorff \\ Two Bits: The Cultural Significance of Free Software p. 107}}

   \subsection*{Outcomes}
   The project will produce three artifacts: a paper detailing some perspectives within \acrshort{eden} and reflecting on the attempt to engage them, New \acrshort{eden} (a modified version of \acrshort{eden}) and a git repository that will have a history of the project..
   % Broader   Impacts (.5-.75 page)
   % Output (e.g., conference presentations, results to informants, white paper, thesis, etc.)
   \subsubsection*{The Git Repository}
   The two primary artifacts and all the other ephemera (including this proposal) involved in producing them will be publicly available in a git repository.

   The Git repository for this project is available at \url{https://github.com/aeturnum/masters_project}.

   \subsection*{Phases}
   \subsubsection*{One}
   The first phase of this project will be a survey of literature. Focusing on Science, Technology and Society works that analyze the standpoint and biases of other technical tools and relevant work on digital humanitarianism and disaster recovery. I will also be familiarizing myself with the implementation and operation of EDEN itself. These theoretical tools are what I will engage the "text" of EDEN with in the second phase. 
   \begin{flushleft}
   The git repository will begin having content added to it immediately.
   \end{flushleft}
   \subsubsection*{Two}
   The second phase will be be a co-constitutive process between social critique and technical analysis. I'll search for perspectives within \acrshort{eden}, analyzing how that perspective emerges from \acrshort{eden} and impacts the world around \acrshort{eden}. My focus will be perspectives that seem driven by the material technical realities of \acrshort{eden}. These findings will be the center of the project. This will be a filtering process - some number of perspectives will need to be investigated and found uninteresting. 
   \begin{flushleft}
   This phase will produce the first half of the paper.
   \end{flushleft}
   \subsubsection*{Three}
   The perspectives found in phase two will guide modifications to the \acrshort{eden} software package itself. These modifications will be made with the goal of engaging with \acrshort{eden} on a material level and understanding the practicalities of trying to address encoded perspectives. The modified software, called "New \acrshort{eden}", will be the second product of this project.
   \begin{flushleft}
   This phase will produce the second half of the paper and New \acrshort{eden}.
   \end{flushleft}
   \subsection*{Schedule}
   % Schedule of work
   % Much thanks to: 
   % https://www.tablesgenerator.com/
   % https://en.wikibooks.org/wiki/LaTeX/Tables#Text_wrapping_in_tables
   % https://stackoverflow.com/questions/790932/how-to-wrap-text-in-latex-tables
   \begin{center}
   \begin{tabular}{|l|l|p{0.1\linewidth}|p{0.1\linewidth}|p{0.6\linewidth}|}
   \hline
   Phase & Quarter & Activity& Time Allocation & Goal                         \\ \hline
   One   & Fall    & Reading & 80\%            & 800 pages read with notes    \\ \cline{3-5} 
         &         & Writing & 10\%            & Notes and planning documents \\ \cline{3-5} 
         &         & Coding  & 10\%            & Notes on \acrshort{eden} program structure   \\ \hline 
   Two   & Winter  & Reading & 40\%            & Final selection of bibliography with additions based on notes and readings.   \\ \cline{3-5} 
         &         & Writing & 30\%            & Outline for final paper with ~20\% of content finished \\ \cline{3-5} 
         &         & Coding  & 30\%            & Rough modifications completed, but in need of polish  \\ \hline
   Three & Spring  & Reading & 10\%            & Unexpected additions to literature and gathering specific quotes from previously completed elements.   \\ \cline{3-5} 
         &         & Writing & 40\%            & Final paper \\ \cline{3-5}
         &         & Coding  & 40\%            & Full branch of \acrshort{eden} project with documentation and demonstration server. \\ \hline
   \end{tabular}
   \end{center}
   % \newpage

   \subsection*{Fieldsite Details}
   % left over
   \acrshort{eden} was initially developed and deployed by an alliance of companies in the Sri Lankan \acrfull{ict} sector in response to the Indian Ocean Earthquake \& Tsunami. \acrshort{eden} has been used in response to a number of disasters in the developing world as well as ongoing supply management programs in the developed world \citep{Sahana_Foundation_undated-hl}. It is written in the \acrshort{foss} language \Gls{python} and uses the \acrshort{foss} \gls{web framework} web2py to build its components. Its functionality is broken up into many different modules which group common functionality within a layer of access conventions. \acrshort{eden} also uses \acrshort{foss} tools wherever possible: its databases of choice (\Gls{MySQL} and \Gls{PostgreSQL}) are \acrshort{foss} databases. The focus on \acrlong{foss} technologies is an obvious structuring perspective for the \acrshort{eden} project and ones whose impacts are worth investigating.

   The software's homepage can be found here: \url{https://sahanafoundation.org/}. \linebreak 
   % A demo version of it running on the Sahana Foundation can be found here: \url{http://demo.sahanafoundation.org/\acrshort{eden}/}. \linebreak 
   \acrshort{eden}'s source code can be found here: \url{https://github.com/sahana/eden} \linebreak 

   % Description of methods
   \subsection*{Preparation of Researcher}
   As a Science, Technology \& Society graduate student I've been engaging with analytical questions around the impact of tools and their power to structure space. A previous project of mine used a energy pricing tool as a jumping off point for a critique of approaching the question of energy supply financially and suggested an alternative or additive perspective and gave me experience engaging with the social implication of a technical system.

   Previous to being a graduate student I obtained an undergraduate degree in computer science and worked professionally as a programmer for about eight years. I did most of my work writing and designing web services and generally used python for this purpose (like \acrshort{eden} in both cases). I have experience using most of the tools and environments that \acrshort{eden} utilizes and expect that this background will be an asset in smoothly 

   \subsection*{Goals}
   % How it will help us understand X (and Y)
   This project will create a new space where the intersections between technology and society are hyper-visible. It will help to highlight the qualities of technical objects that give rise to their socially visible qualities. It will hopefully inspire others to further explore the power of modifying software as a way of finding and highlighting ways in which that software intrudes into our social lives. It may even provoke more engaged and practical work in software studies that struggle the daily challenges of writing software.

   \newpage
   \printnoidxglossaries

   \newpage
   \setlength{\parindent}{4em}
   \bibliography{full_proposal}

\end{document}