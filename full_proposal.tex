%
% ______          _           _    ______                                _ 
% | ___ \        (_)         | |   | ___ \                              | |
% | |_/ / __ ___  _  ___  ___| |_  | |_/ / __ ___  _ __   ___  ___  __ _| |
% |  __/ '__/ _ \| |/ _ \/ __| __| |  __/ '__/ _ \| '_ \ / _ \/ __|/ _` | |
% | |  | | | (_) | |  __/ (__| |_  | |  | | | (_) | |_) | (_) \__ \ (_| | |
% \_|  |_|  \___/| |\___|\___|\__| \_|  |_|  \___/| .__/ \___/|___/\__,_|_|
%               _/ |                              | |                      
%              |__/                               |_|                      
% -------------------------------------------------------------------------
% It is generally worthwhile, I think, to justfiy the use of LaTeX. It's obtuse 
% and it's difficult and it's unclear that it's any better than using Word for
% most projects. However, it has one advantage: it is generated from plain text 
% and this project is using Git, a tool that talks plain text. 
% 
% Using LaTeX will let me create a relaibly formatted document and also have an
% intelligable history of changes. It will also let me work on this document (and
% presumably my final paper which will also use LaTeX) in any environment. It's also
% nice to have a language with comments that won't render into the final paper for
% items like this.
% 
% If there is demand for .docx files, we can convert, but until then...


% Use APA module
% Based on template from https://www.overleaf.com/latex/templates/your-apa6-style-manuscript/kngbbqpypjcq
\documentclass[a4paper,man,natbib]{apa6}
% Imports natbib
% Ref Sheet: https://gking.harvard.edu/files/natnotes2.pdf

% Packages
\usepackage[english]{babel}
\usepackage[utf8x]{inputenc}
\usepackage{amsmath}
\usepackage{graphicx}
% For formatting of potential bibliography
\usepackage{enumitem}
% html links
\usepackage{hyperref}
%\usepackage[colorinlistoftodos]{todonotes}

\title{Mapping EDEN Proposal}
\shorttitle{Final Masters Project Proposal}
\author{Daniel "Drex" Drexler}
\affiliation{Center for Science, Technology and Society at Drexel University}
\date{August 2019}
\begin{document}
   \maketitle
   \clearpage
   % This document outline based on the suggested outline in the Center for STS Student Handbook:
   % https://drexel.edu/~/media/Files/sts/2015-graduate-student-handbook.ashx?la=en

   \section*{Overview}
   % Overview (0.5-1 page)
   % A summary of other sections of the proposal
   \subsection*{Description}
   Mapping EDEN is an experiment in wedding sociological critique with a craft-tradition of Computer Science seeking artifacts and ways of making and doing that suggest fresh approaches to multiple disciplines.
   % A one line description of the project
   \subsection*{Background}
   % Background / context for project / articulation of contemporary problem being addressed
   Ideally, the goal of Science, Technology and Society is to do work that is fully engaged with the practical details of human experience and effort and uses that engagement to develop theoretical understandings that reveal new points of contact and analyses. \citet{Sismondo2008-xz} describes this ideal as an “engaged program” that seeks to use pertinent topics in theoretically interesting ways. This project attempts to realize these qualities by simultaneously applying theoretical critiques to a situated and particular software tool and modifying that tool in conversation with those critiques. The end goal is a process-object that exists both as a fully functional piece of software and a record of the progress towards analyzing and creating that software from an existing base. I chose this particular form because it draws from both the traditions of software development and Science, Technology and Society scholarship but sets neither above the other. An engaged product must strive to be fully available to more than just the scholars of the social sciences.
   \subsection*{Merits}
   % 2-3 intellectual merits (informs research questions)
   Primarily, this project grew out of a sense of the disconnect between the making and doing of Science, Technology and Society and the making and doing of engineering. This approach was inspired the approaches of \citet{Roy2018-ma} and \citet{Subramaniam2014-wg} in working to learn more through their combination of the scientific method with feminist critique. My project is not one centered on feminist critique, but within my own background of craft Computer Science \citep[see][]{Ensmenger2012-kz}. 
   \section*{Background}
   % Background	description	of	problem	and	context (1 page)
   \section*{Intellectual Heritage}
   % Intellectual	Merits (1-2 pages)
   % 2 to 3 scholarly conversations your research contributes to
   \subsection*{Software Studies}
   \subsection*{Situated Knowledges}
   \section*{Design}
   % Study	Design (1-2 pages)
   \subsection*{Methods}
   The aim of this project is to put academic work in conversation with practical tools as well as put academic criticism in conversation with work to  transform the same tools. This will draw on Science, Technology and Society's tradition of Making and Doing as well as traditional methods for analysis and critique.
   % Description of methods
   A PDF of this document can be found at \url{https://github.com/aeturnum/masters_project/blob/master/full_proposal.pdf}.

   The \LaTeX source file can be found at \url{https://github.com/aeturnum/masters_project/blob/master/full_proposal.tex}.
   \subsection*{Fieldsites}
   % Description and justification of data sources or fieldsites

   The EDEN software package, developed by the Sahana Foundation (a non-profit) and deployed to manage disaster response, appears to be an excellent fit for this project. It is designed with a purpose and with the intent to save lives. It has good reason to simplify and focus perspectives of those using the tool and to problematize particular aspects of the situation while allowing others to remain innocuous. Its lack of commercial motivation will also likely simply the process of decomposing whom its perspectives benefit.

   The software's homepage can be found here: \url{https://sahanafoundation.org/}. A demo version of it running on the Sahana foundation can be found here: \url{http://demo.sahanafoundation.org/eden/}. Its source code can be found here: \url{https://github.com/sahana/eden}
   \subsection*{Research Questions}
   What ways of knowing and doing are assisted by the design of EDEN?

   What epistemological traditions are centered in EDEN's methods for storing data? Whose needs are attended to first in this particular product?
   \subsection*{Schedule}
   % Schedule of work
   % Much thanks to: 
   % https://www.tablesgenerator.com/
   % https://en.wikibooks.org/wiki/LaTeX/Tables#Text_wrapping_in_tables
   % https://stackoverflow.com/questions/790932/how-to-wrap-text-in-latex-tables
   \begin{center}
   \begin{tabular}{|l|p{0.1\linewidth}|p{0.1\linewidth}|p{0.7\linewidth}|}
   \hline
   Quarter & Quality & Time Allocation & Goal                         \\ \hline
   Fall    & Reading & 80\%            & 800 pages read with notes    \\ \cline{2-4} 
           & Writing & 10\%            & Notes and planning documents \\ \cline{2-4} 
           & Coding  & 10\%            & Notes on project structure   \\ \hline 
   Winter  & Reading & 40\%            & Final selection of bibliography with additions based on notes and readings.   \\ \cline{2-4} 
           & Writing & 30\%            & Outline for final paper with ~20\% of content finished \\ \cline{2-4} 
           & Coding  & 30\%            & Rough modifications completed, but in need of polish  \\ \hline
   Fall    & Reading & 10\%            & Unexpected additions to literature and gathering specific quotes from previously completed elements.   \\ \cline{2-4} 
           & Writing & 40\%            & Final paper \\ \cline{2-4}
           & Coding  & 40\%            & Full branch of EDEN project with documentation and demonstration server. \\ \hline
   \end{tabular}
   \end{center}
   \section*{Impacts}
   % Broader	Impacts (.5-.75 page)
   \subsection*{Artifacts}
   % Output (e.g., conference presentations, results to informants, white paper, thesis, etc.)
   \subsection*{Goals}
   % How it will help us understand X (and Y)
   \clearpage
   \section{Suggested Bibliography}

   % Much thanks to https://tex.stackexchange.com/questions/99344/indent-text-after-line-1
   \begin{itemize}[label={},itemindent=-2em,leftmargin=2em]
   \item Ananny, M., \& Crawford, K. (2018). Seeing without knowing: Limitations of the transparency ideal and its application to algorithmic accountability. New Media \& Society, 20(3), 973–989.

   \item Barocas, S., \& Boyd, D. (2017). Engaging the Ethics of Data Science in Practice. Communications of the ACM, 60(11), 23–25.

   \item Benjamin, R. (2014). Race for cures: Rethinking the racial logics of “trust”in biomedicine. Sociology Compass, 8(6), 755–769.

   \item Boellstorff, T. (2016). For Whom the Ontology Turns: Theorizing the Digital Real. Current Anthropology, 57(4), 387–407.

   \item Bowker, G. C., \& Star, S. L. (2000). Sorting Things Out: Classification and Its Consequences. MIT Press.

   \item Cheney-Lippold, J. (2018). We Are Data: Algorithms and the Making of Our Digital Selves. NYU Press.

   \item Cooke, N. A., Sweeney, M. E., \& Noble, S. U. (2016). Social Justice as Topic and Tool: An Attempt to Transform an LIS Curriculum and Culture. The Library Quarterly, 86(1), 107–124.

   \item Crawford, K., \& Finn, M. (2015). The limits of crisis data: analytical and ethical challenges of using social and mobile data to understand disasters. GeoJournal, 80(4), 491–502.

   \item Deleuze, G. (1992). Postscript on the Societies of Control. October, 59, 3–7.

   \item Dourish, P., Anderson, K., \& Nafus, D. (2007). Cultural mobilities: Diversity and agency in urban computing. IFIP Conference on Human-Computer. Retrieved from https://link.springer.com/chapter/10.1007/978-3-540-74800-7\_8

   \item Eckhouse, L., Lum, K., Conti-Cook, C., \& Ciccolini, J. (2019). Layers of Bias: A Unified Approach for Understanding Problems With Risk Assessment. Criminal Justice and Behavior, 46(2), 185–209.

   \item Elish, M. C., \& Boyd, D. (2018). Situating methods in the magic of Big Data and AI. Communication Monographs, Vol. 85, pp. 57–80. https://doi.org/10.1080/03637751.2017.1375130

   \item Eubanks, V. (2018). Automating Inequality: How High-Tech Tools Profile, Police, and Punish the Poor. St. Martin’s Press.

   \item Gabriella Coleman, E. (2012). Coding Freedom: The Ethics and Aesthetics of Hacking (pp. 69–70). Princeton University Press.

   \item Gangadharan, S. P. (2017). The downside of digital inclusion: Expectations and experiences of privacy and surveillance among marginal Internet users. New Media \& Society, 19(4), 597–615.

   \item Gillespie, T. (2018). Custodians of the Internet: Platforms, Content Moderation, and the Hidden Decisions That Shape Social Media. Yale University Press.

   \item Haraway, D. (1988). Situated Knowledges: The Science Question in Feminism and the Privilege of Partial Perspective. Feminist Studies: FS, 14(3), 575–599.

   \item Haraway, D. J. (2016). Staying with the Trouble: Making Kin in the Chthulucene. Duke University Press.

   \item Harding, S. (1992). Rethinking standpoint epistemology: What is“ strong objectivity?” The Centennial Review, 36(3), 437–470.

   \item Harraway, D. (1997). Modest\_Witness@Second\_Millennium.FemaleMan©Meet\_OncoMouseTM. Routledge New York.

   \item Howard, P. N., Anderson, K., \& Busch, L. (2009). Sizing up information societies: Toward a better metric for the cultures of ICT adoption. Information. An International Interdisciplinary Journal. Retrieved from https://www.tandfonline.com/doi/abs/10.1080/01972240902848948

   \item Kelty, C. M. (2008). Two Bits: The Cultural Significance of Free Software. Duke University Press.

   \item Knowles, S. G. (2014). Learning from Disaster?: The History of Technology and the Future of Disaster Research. Technology and Culture, 55(4), 773–784.

   \item Latonero, M., \& Kift, P. (2018). On Digital Passages and Borders: Refugees and the New Infrastructure for Movement and Control. Social Media + Society, 4(1), 2056305118764432.

   \item Mackenzie, A. (2010). Wirelessness: Radical Empiricism in Network Cultures. MIT Press.

   \item Roy, D. (2018). Molecular Feminisms: Biology, Becomings, and Life in the Lab. University of Washington Press.

   \item Nafus, D. (2014). Stuck data, dead data, and disloyal data: the stops and starts in making numbers into social practices. Distinktion: Scandinavian Journal of Social Theory. Retrieved from https://www.tandfonline.com/doi/abs/10.1080/1600910X.2014.920266

   \item Noble, S. U. (2017). Google and the misinformed public. The Chronicle of Higher Education.

   \item Ponte, A. (2014). The House of Light and Entropy. Architectural Association.

   \item Star, S. L. (ed ). (1995). The cultures of computing. Retrieved from http://agris.fao.org/agris-search/search.do?recordID=XF2015029404

   \item Subramaniam, B. (2014). Ghost Stories for Darwin: The Science of Variation and the Politics of Diversity. University of Illinois Press.

   \item Wolske, M., Williams, N. S., Noble, S., Johnson, E. O., \& Duple, R. Y. (2010). Effective ICT use for Social Inclusion. Retrieved from http://www.ideals.illinois.edu/handle/2142/14952

   \end{itemize}

   \setlength{\parindent}{4em}
   \bibliography{full_proposal}

\end{document}